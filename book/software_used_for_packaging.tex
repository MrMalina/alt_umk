\hypertarget{3}{\chapter{Программное обеспечение используемое для упаковки}}\label{software-used-for-packaging}

В разделе \hyperlink{1.3}{1.3 <<Установка необходимых пакетов для процесса сборки>>} 
упоминался перечень пакетов для работы со сборкой:
\begin{verbatim}
	gcc, rpm-build, rpmlint, make, python, git, gear, hasher, patch, rpmdevtools.
\end{verbatim}

Опишем основные инструменты для управления пакетами и их сборки. Технологическую 
базу репозитория Sisyphus составляют адаптированные к нуждам команды разработчиков 
программы и специально разработанные решения%
\footnote{\href{https://wiki.altlinux.ru/QuickStart/PkgManagment\#\%D0\%9E\%D1\%81\%D0\%BD\%D0\%BE\%D0\%B2\%D0\%BD\%D1\%8B\%D0\%B5_\%D0\%B8\%D0\%BD\%D1\%81\%D1\%82\%D1\%80\%D1\%83\%D0\%BC\%D0\%B5\%D0\%BD\%D1\%82\%D1\%8B_\%D0\%B4\%D0\%BB\%D1\%8F_\%D1\%83\%D0\%BF\%D1\%80\%D0\%B0\%D0\%B2\%D0\%BB\%D0\%B5\%D0\%BD\%D0\%B8\%D1\%8F_\%D0\%BF\%D0\%B0\%D0\%BA\%D0\%B5\%D1\%82\%D0\%B0\%D0\%BC\%D0\%B8}{https://wiki.altlinux.ru/QuickStart/PkgManagment}}. 
К ним можно отнести\footnote{\href{https://www.altlinux.org/\%D0\%A0\%D0\%B5\%D0\%BF\%D0\%BE\%D0\%B7\%D0\%B8\%D1\%82\%D0\%BE\%D1\%80\%D0\%B8\%D0\%B9_\%D0\%A1\%D0\%9F\%D0\%9E\#APT_\%D0\%B8_\%D1\%80\%D0\%B5\%D0\%BF\%D0\%BE\%D0\%B7\%D0\%B8\%D1\%82\%D0\%BE\%D1\%80\%D0\%B8\%D0\%B9_\%D0\%BF\%D0\%B0\%D0\%BA\%D0\%B5\%D1\%82\%D0\%BE\%D0\%B2}{https://www.altlinux.org/Репозиторий\_СПО\#APT\_и\_репозиторий\_пакетов}.}:
\begin{itemize}
	\item \Emph{RPM} (менеджер пакетов). Используется для просмотра, сборки, установки, инспекции, 
		проверки, обновления и удаления отдельных программных пакетов. Каждый такой пакет состоит 
		из набора файлов и информации о пакете, включающей название, версию, описание пакета и~т.\,д. 
		Отличительными особенностями \Sys{RPM} в Sisyphus являются: удобное поведение <<по умолчанию>> 
		для уменьшения количества шаблонного кода в \Sys{.spec}-файлах, обширный модульный набор 
		макросов для упаковки различных типов пакетов, развитые механизмы автоматического вычисления 
		межпакетных зависимостей при сборке пакетов, поддержка \Sys{set}-версий в зависимостях на 
		разделяемые библиотеки, автоматическое создание пакетов с отладочной информацией с поддержкой 
		зависимостей между такими пакетами. В контексте данной темы \Sys{RPM} рассматривается не только 
		как менеджер пакетов, но и как набор инструментов для их сборки (\Sys{rpmbuild}). 
	\item \Emph{APT} (усовершенствованная система управления программными пакетами). Умеет автоматически 
		разрешать зависимости при установке, обеспечивает установку из нескольких источников и целый 
		ряд других уникальных возможностей, включая получение последней версии списка пакетов из 
		репозитория и обновление системы.
	\item \Emph{Hasher} --- инструмент для безопасной, воспроизводимой и высокопроизводительной сборки 
		\Sys{RPM}-пакетов в контролируемой среде.
	\item \Emph{Gear} --- набор инструментов для поддержки совместной разработки \Sys{RPM}-пакетов в 
		системе контроля версий \Sys{git}. \Sys{Gear} интегрирован с \Sys{hasher} и \Sys{rpmbuild}. 
		\Sys{Gear} подготавливает \Sys{SRPM} из \Sys{git} репозитория, распаковывает его и запускает 
		\Sys{hasher} или \Sys{rpmbuild}.
\end{itemize}


\hypertarget{3.1}{\section{Описание RPM-пакета}}
\Emph{RPM-пакет} --- это специальный архив с файлами. Сам файл пакета состоит из четырёх секций --- 
начального идентификатора, сигнатуры, бинарного заголовка и \Sys{cpio}-архива с файлами проекта и деревом 
каталога\footnote{\href{https://www.opennet.ru/docs/RUS/rpm_guide/13.html}{https://www.opennet.ru/docs/RUS/rpm\_guide/13.html}}.

\Sys{RPM}-пакеты делятся на несколько категорий --- пакеты с исходным кодом, бинарные пакеты и платформо-независимые 
бинарные пакеты.
\begin{itemize}
	\item \Emph{RPM-пакет} (бинарные) --- это архив с расширением \Sys{.rpm}. Такой пакет содержит 
		скомпилированные под определённую процессорную архитектуру исполняемые файлы и библиотеки. 
		На системах с разной процессорной архитектурой не получится использовать один и тот же 
		скомпилированный бинарный \Sys{rpm}-пакет.
	\item \Emph{noarch-пакет} --- платформо-независимый бинарный пакет.
	\item \Emph{SRPM-пакет} (source RPM, пакет с исходным кодом) --- это архив с расширением \Sys{.src.rpm}. 
		\Sys{SRPM} содержит исходный код, патчи, если необходимо, и \Sys{SPEC}-файл. Эти пакеты содержат 
		всю информацию для сборки пакета.
\end{itemize}

По инструкциям из \Sys{SPEC} файла собирается бинарный \Sys{RPM}-пакет%
\footnote{\href{https://uneex.ru/static/RedHatRPMGuideBook/rpm_guide-linux.html\#16_html}{https://uneex.ru/static/RedHatRPMGuideBook/rpm\_guide-linux.html\#16\_html}}. 
На основе бинарных пакетов строится база данных в \Sys{/var/lib/rpm}. Вся информация о пакетах хранится в базе 
данных \Sys{Packages}. Инструкции содержат также информацию о правах доступа и их применении в процессе установки, 
скрипты, запускаемые при установке или удалении пакета.

Принято называть пакеты \Sys{RPM} по типу:
\begin{verbatim}
	имя-версия-релиз.процессорная_архитектура.rpm
\end{verbatim}

При этом в системе имя установленного пакета будет отличаться: в командной строке можно обратиться за информацией 
по пакету, указывая только его имя, если установлена одна версия пакета, и указывая \Sys{имя-версию-релиз}, если 
установлено больше версий\footnote{\href{https://www.opennet.ru/docs/RUS/rpm_guide/13.html}{https://www.opennet.ru/docs/RUS/rpm\_guide/13.html}}.


\section{Инструменты для сборки RPM-пакетов}
\Emph{Инструменты сборки \Sys{rpm}-пакета} --- это пакеты и программы, с помощью которых из набора исходных файлов 
можно получить специальный архив в формате \Sys{rpm}-пакета. Приведём список этих инструментов%
\footnote{\href{https://www.altlinux.org/\%D0\%A1\%D0\%B1\%D0\%BE\%D1\%80\%D0\%BA\%D0\%B0_\%D0\%BF\%D0\%B0\%D0\%BA\%D0\%B5\%D1\%82\%D0\%B0_\%D1\%81_\%D0\%BD\%D1\%83\%D0\%BB\%D1\%8F}%
{https://www.altlinux.org/Сборка\_пакета\_с\_нуля}}: \Sys{rpmdevtools}, \Sys{rpmdev-setuptree}, \Sys{rpmbuild}, \Sys{rpmspec}, \Sys{rpmlint}. 

\begin{itemize}
	\item \Emph{rpmdevtools} --- пакет с набором программ для сборки пакетов; 
	\begin{itemize}
		\item \Emph{rpmdev-setuptree} --- утилита для создания структуры рабочих каталогов;
		\item \Emph{rpmdev-newspec} --- утилита для создания \Sys{spec}-файла;
	\end{itemize}
	\item \Emph{rpmspec} --- утилита работы с файлами спецификации --- текстовыми файлами с расширением 
		\Sys{.spec}. Служит для проверки подготовленного \Sys{spec}-файла;
	\item \Emph{rpmbuild} --- утилита сборки \Sys{rpm}-пакета из набора подготовленных файлов;
	\item \Emph{rpmlint} --- утилита для тестирования собранного \Sys{rpm}-пакета;
	\item \Emph{rpm-utils} --- пакет с набором программ для работы с \Sys{rpm}-пакетами.
\end{itemize}


\section{Рабочее пространство для сборки RPM-пакетов}
\Emph{Рабочая область упаковки \Sys{RPM}-пакета} --- это структура файлов и каталогов. Эту структуру 
можно создать двумя способами --- вручную или через утилиту \Sys{rpmdev-setuptree}.

Для подготовки ручным способом структуры каталогов выполните команду:
\begin{verbatim}
	mkdir -p ~/RPM/{BUILD,SRPMS,RPMS,SOURCES,SPECS}
\end{verbatim}

Альтернативный способ подготовки рабочей среды --- утилита \Sys{rpmdev-setuptree}. Утилита входит в 
состав пакета \Sys{rpmdevtools} (см.\,\hyperlink{1.3}{раздел 1.3}). Для подготовки структуры каталогов 
через утилиту \Sys{rpmdev-setuptree} выполните команду:
\begin{verbatim}
	$ rpmdev-setuptree
\end{verbatim}

Утилита создаст базовую структуру каталогов и файл \Sys{~/.rpmmacros}, если его не существовало.

Для системы ALT расположение структуры каталогов%
%\footnote{\href{https://rpm-packaging-guide-ru.github.io/\#rpm-packaging-workspace}{https://rpm-packaging-guide-ru.github.io/\#rpm-packaging-workspace}} 
по умолчанию определятся в файле \Sys{~/.rpmmacros} и находится в каталоге \Sys{~/RPM}:
\begin{verbatim}
	$ tree ~/RPM
	/home/tefst/RPM
	├── BUILD
	├── RPMS
	├── SOURCES
	├── SPECS
	└── SRPMS
	
	5 directories, 0 files
\end{verbatim}

\begin{enumerate}
	\item \Emph{BUILD} --- в каталог попадают распакованные исходные файлы из 
		\Sys{SOURCES} c уже применёнными патчами --- стадия \Sys{\%prep}. 
		В каталоге \Sys{BUILD} происходит сборка программного обеспечения.
	\item \Emph{RPMS} --- в каталог \Sys{RPMS} бинарные \Sys{RPM}-пакеты после 
		сборки, в соответствии с подкаталогами для поддерживаемых архитектур.
	\item \Emph{SOURCES} --- в каталоге размещают архивы исходных данных и патчи.
	\item \Emph{SPECS} --- в каталоге размещают\Sys{spec}-файлы пакетов для сборки.
	\item \Emph{SRPMS} --- в каталог SRPMS попадают результаты сборки SRPM пакетов.
\end{enumerate}

Созданная структура каталогов становится рабочей областью упаковки \Sys{RPM}-пакета.

\marginalia{ex_sign_col}{В структуре сборочного окружения \Sys{RPM} существует понятие 
\Sys{buildroot}. Это каталог \Sys{/TMP}, в который попадают служебные файлы в ходе сборки 
пакета и уже подготовленные бинарные данные для упаковки в соответствии со структурой 
каталогов. По умолчанию создаётся во временном системном каталоге. Может быть переназначен.}


\section{Описание SPEC-файла}
\Emph{SPEC-файл} --- RPM Specification File --- это текстовый файл, который описывает процесс сборки и конфигурацию 
пакета, служит инструкцией для утилиты \Sys{rpmbuild}. Он содержит метаданные, такие как имя пакета, версию, 
лицензию, а также разделы с инструкциями для сборки, установки и упаковки программного обеспечения, журнал 
изменений пакета%
\footnote{\href{https://wiki.mageia.org.ru/index.php?title=\%D0\%A1\%D1\%82\%D1\%80\%D1\%83\%D0\%BA\%D1\%82\%D1\%83\%D1\%80\%D0\%B0_RPM_SPEC-\%D1\%84\%D0\%B0\%D0\%B9\%D0\%BB\%D0\%B0}%
{https://wiki.mageia.org.ru/index.php?title=Структура\_RPM\_SPEC-файла}}. %
\Sys{SPEC}-файл можно рассматривать как <<инструкцию>>, которую утилита \Sys{rpmbuild} использует для сборки \Sys{RPM}-пакета.

\Sys{Spec}-файл состоит из трёх разделов: Header (Заголовок/Преамбула), Body (Тело) и Сhangelog (Журнал изменений).
\begin{enumerate}
	\item \Emph{Header} (Заголовок) --- этот раздел содержит метаданные о пакете, такие как его имя (Name), 
		версия (Version), релиз (Release), краткое описание (Summary), лицензия (License) и другие 
		параметры, которые идентифицируют и характеризуют пакет.
	\item \Emph{Body} (Тело) --- этот раздел содержит инструкции для процесса сборки пакета. В нём 
		определяются различные секции, такие как \Sys{BuildRequires} (зависимости для сборки), 
		\Sys{\%build} (инструкции для сборки), \Sys{\%install} (инструкции для установки), 
		\Sys{\%files} (список файлов, включённых в пакет), и другие.	
	\item \Emph{Changelog} (Журнал изменений) --- этот раздел содержит историю изменений пакета. 
		Он включает записи о внесённых изменениях, включая дату изменения, автора и краткое 
		описание того, что было изменено.
\end{enumerate}

\Emph{Преамбула (Заголовок)}

Заголовок \Sys{SPEC}-файла содержит информацию о пакете: версию, исходный код, патчи, зависимости.

Рекомендуемый порядок заголовочных тэгов:
\begin{itemize}
	\item Name, Version, Release, Serial
	\item далее Summary, License, Group, Url, Packager, BuildArch
	\item потом Source*, Patch*
	\item далее PreReqs, Requires, Provides, Conflicts
	\item и, наконец, Prefix, BuildPreReqs, BuildRequires.
\end{itemize}

Ниже приведён пример части \Sys{SPEC}-файла \Sys{notepadqq}:
\begin{verbatim}
	Summary:	A Linux clone of Notepad++
	Name:		notepadqq
	Version:	1.4.8
	Release:	alt2
	License:	GPLv3
	Group:		Editors
	URL:		http://notepadqq.altervista.org/wp/
	Source0:	%name-%version.tar
	Source1:	codemirror.tar
\end{verbatim}


\section{Пример \Sys{.spec}-файла}
\Emph{SPEC-файл} --- RPM Specification File --- это текстовый файл, который описывает 
процесс сборки и конфигурацию пакета. \Sys{SPEC}-файл можно рассматривать как <<инструкцию>>, 
которую утилита \Sys{rpmbuild} использует для сборки \Sys{RPM}-пакета. Он содержит метаданные, 
такие как имя пакета, версию, лицензию, а также разделы с инструкциями для сборки, установки 
и упаковки программного обеспечения, журнал изменений пакета.

Представим образцы \Sys{SPEC}-файлов%
\footnote{\href{https://www.altlinux.org/SampleSpecs}{https://www.altlinux.org/SampleSpecs}} 
--- шаблонный образец, образец для программы на \Sys{autotools}, образец для программы на 
\Sys{cmake} и образец модуля для \Sys{Python 3}.

\subsection*{Примеры \Sys{SPEC}-файлов}

\Emph{Пример 1.} Пустой \Sys{SPEC}
\begin{Verbatim}[breaklines=true,breakanywhere=true,fontsize=\scriptsize]
Name: <имя-пакета>
Version: <версия-пакета>
Release: alt<релиз-пакета>
	
Summary: <однострочное описание>
License: <лицензия>
Group: <группа>

Url: <URL>
Source: %name-%version.tar
Patch1:
	
PreReq:
Requires:
Provides:
Conflicts:
	
BuildPreReq:
BuildRequires:
%{?!_without_test:%{?!_disable_test:%{?!_without_check:%{?!_disable_check:BuildRequires: }}}}
BuildArch:
	
%description
<многострочное
описание>
	
%prep
%setup
%patch1 -p1

%build
%configure
%make_build
		
%install
%makeinstall_std
	
%check
%make_build check
		
%files
%_bindir/*
%_man1dir/*
%doc AUTHORS NEWS README
		
%changelog
* <дата> <ваше имя> <$login@altlinux.org> <версия_пакета>-<релиз_пакета>
- initial build for ALT Linux Sisyphus
\end{Verbatim}

\Emph{Пример 2.} Для программы (на \Sys{autotools})
	
		Название \Sys{GNU Autotools} обычно относится к программным пакетам \Sys{Autoconf}, 
		\Sys{Automake}, \Sys{Libtool} и \Sys{Gnulib}. Вместе они составляют систему сборки 
		\Sys{GNU}. Этот \Sys{SPEC}-файл является примером пакета с программой.
	\begin{Verbatim}[breaklines=true,breakanywhere=true,fontsize=\scriptsize]
Name: sampleprog
Version: 1.0
Release: alt1
	
Summary: Sample program specfile
Summary(ru_RU.UTF-8): Пример spec-файла для программы
	
License: GPLv2+
Group: Development/Other
Url: http://www.altlinux.org/SampleSpecs/program
		
Source: %name-%version.tar
Patch0: %name-1.0-alt-makefile-fixes.patch
		
%description
This specfile is provided as sample specfile for packages with programs.
It contains most of usual tags and constructions used in such specfiles.
		
%description -l ru_RU.UTF-8
		
%prep
%setup
%patch0 -p1
	
%build
%configure
%make_build
		
%install
%makeinstall_std
%find_lang %name
		
%files -f %name.lang
%doc AUTHORS ChangeLog NEWS README THANKS TODO contrib/ manual/
%_bindir/*
%_man1dir/*
		
%changelog
* Sat Sep 33 3001 Sample Packager <sample@altlinux.org> 1.0-alt1
- initial build
\end{Verbatim}
	
	\Emph{Пример 3.} Для программы на \Sys{cmake}
	
	\Emph{CMake} --- это кроссплатформенный инструмент с открытым исходным кодом, который использует 
	независимые от компилятора и платформы файлы конфигурации для создания собственных файлов 
	инструментов сборки, специфичных для вашего компилятора и платформы. Является стандартом 
	де-факто для сборки кода на \Sys{C++}.
\begin{Verbatim}[breaklines=true,breakanywhere=true,fontsize=\scriptsize]
Name: sampleprog
Version: 1.0
Release: alt1
		
Summary: Sample program specfile
License: GPLv2+
Group: Development/Other
		
Url: http://www.altlinux.org/SampleSpecs/cmakeprogram
Source: %name-%version.tar.bz2
		
BuildRequires(pre): cmake rpm-macros-cmake
		
%description
This specfile is provided as a sample specfile
for a package built with cmake.
		
%prep
%setup
		
%build
%cmake_insource
%make_build # VERBOSE=1
		
%install
%makeinstall_std
%find_lang %name
		
%files -f %name.lang
%doc AUTHORS ChangeLog NEWS README THANKS TODO contrib/ manual/
%_bindir/*
%_man1dir/*
		
%changelog
* Sat Jan 33 3001 Example Packager <example@altlinux.org> 1.0-alt1
- initial build
\end{Verbatim}
	
\Emph{Пример 4.} Модуль для \Sys{Python 3}
	
	Макросы для сборки модулей \Sys{python3} содержатся в пакете \Sys{rpm-build-python3} и 
	аналогичны тем, что используются в ALT для \Sys{python}\footnote{\href{https://www.altlinux.org/Python3}{https://www.altlinux.org/Python3}}.
\begin{Verbatim}[breaklines=true,breakanywhere=true,fontsize=\scriptsize]
%define pypi_name @NAME@
%define mod_name %pypi_name
		
%def_with check
		
Name: python3-module-%pypi_name
Version: 0.0.0
Release: alt1
Summary: @TEMPLATE@
License: MIT
Group: Development/Python3
Url: https://pypi.org/project/@NAME@
Vcs: @SOURCE_GIT@
BuildArch: noarch
Source: %name-%version.tar
Source1: %pyproject_deps_config_name
Patch: %name-%version-alt.patch
		
# mapping of PyPI name to distro name
Provides: python3-module-%{pep503_name %pypi_name} = %EVR
		
%pyproject_runtimedeps_metadata
BuildRequires(pre): rpm-build-pyproject
%pyproject_builddeps_build
%if_with check
%pyproject_builddeps_metadata
%endif
		
%description
@DESCR@
	
%prep
%setup
%autopatch -p1
%pyproject_deps_resync_build
%pyproject_deps_resync_metadata
		
%build
%pyproject_build
		
%install
%pyproject_install
		
%check
%pyproject_run_pytest
		
%files
%doc README.*
%python3_sitelibdir/%mod_name/
%python3_sitelibdir/%{pyproject_distinfo %pypi_name}
		
%changelog
\end{Verbatim}

\section{Составляющие основной части}
\Emph{SPEC-файл} --- RPM Specification File --- это текстовый файл, который описывает 
процесс сборки и конфигурацию пакета, служит инструкцией для утилиты \Sys{rpmbuild}. 
Он содержит метаданные, такие как имя пакета, версию, лицензию, а также разделы с 
инструкциями для сборки, установки и упаковки программного обеспечения, журнал изменений 
пакета. \Sys{SPEC}-файл можно рассматривать как <<инструкцию>>, которую утилита \Sys{rpmbuild} 
использует для сборки \Sys{RPM}-пакета. 

\Sys{SPEC}-файл состоит из трёх разделов:
\begin{enumerate}
	\item Header (Заголовок/Преамбула);
	\item Body (Тело);
	\item Сhangelog(Журнал изменений).
\end{enumerate}

\Emph{Тело SPEC-файла} отвечает за выполнение сборки, установки или очистки пакета.

Описание структуры%\footnote{\href{https://alt-packaging-guide.github.io/\#body-items}{https://alt-packaging-guide.github.io/\#body-items}}: 
\begin{itemize}
	\item В секции \Sys{\%prep} производится распаковка архивов с исходными кодами и 
		формируется директория с исходниками\footnote{\href{https://www.opennet.ru/docs/RUS/rpm_guide/48.html}{https://www.opennet.ru/docs/RUS/rpm\_guide/48.html}}. 
	\begin{itemize}
		\item Макрос \Sys{\%setup} этой секции выполняет смену рабочего 
			каталога на каталог сборки и распаковку в него архивов с исходными кодами.
		\item Макрос \Sys{\%patch1} будет описывать применение патча.
	\end{itemize}
	\item В секции \Sys{\%build} внутри ранее подготовленной директории производится 
		сборка программы. Если это компилируемый язык, то исходники компилируются в бинарные 
		файлы. Если это интерпретируемый язык, то процесс может не подразумевать компиляцию. 
		Обычно за процесс сборки отвечают системы сборки, отличающиеся для разных языков 
		программирования. Для \Sys{C/C++} обычно используется \Sys{automake/autoconf} и макросы 
		\Sys{\%configure} и \Sys{\%make\_build}. Есть и другие системы сборки с другими 
		макросами --- \Sys{CMake}, \Sys{meson}, \Sys{pyproject} и~т.\,д.
	\item В секции \Sys{\%install} подготавливается новая директория с теми файлами, которые будут 
		помещены в \Sys{RPM} пакет в конце процесса сборки. Эта директория обозначается макросом 
		\Sys{\%buildroot}. Из текущей директории подготовленные на предыдущем этапе файлы (бинарные 
		файлы, файлы документации, конфигурационные файлы и~т.\,д.) нужно перенести в \Sys{\%buildroot}. 
		Например файл \Sys{build/application.bin} нужно перенести в \Sys{\%buildroot/usr/bin/application.bin}. 
		За это в некоторых случаях может также отвечать система сборки. Например, \Sys{automake/autoconf} 
		так умеет. Запускается через макрос \Sys{\%makeinstall\_std}. Для других систем сборки есть другие макросы.
	\item Возможно добавление секции \Sys{\%clean}. Её задача --- очистить дерево сборки и каталог установки.
	\item Если разработчик добавляет в \Sys{SPEC}-файл собственные скрипты, их следует распределять в секции: 
	\begin{itemize}
		\item \Sys{\%pre} (выполнение перед установкой)
		\item \Sys{\%post} (выполнение после установки)
		\item \Sys{\%preun} (перед удалением пакета)
		\item \Sys{\%postun} (после удаления пакета)
	\end{itemize}
	\item Секция \Sys{\%files} содержит список путей и файлов, которые будут упакованы в \Sys{RPM}-пакет 
		и в дальнейшем установлены в систему.
	\begin{itemize}
		\item В этой секции можно создать каталог (\Sys{\%dir}), отметить, что файл является 
			документацией (\Sys{\%doc}) или файлом конфигурации (\Sys{\%config}), или файл не 
			относится к пакету, но необходим в начале работы приложения (\Sys{\%gost}).
	\end{itemize} 
\end{itemize}

\marginalia{ex_sign_col}{Сборка \Sys{rpm}-пакета выполняет все инструкции, указанные в \Sys{SPEC}-файле. 
В процессе сборки утилита \Sys{rpmbuild} выводит на экран информацию о процессе сборки. 
С помощью этой информации можно отследить возможные ошибки описания \Sys{SPEC}-файла.}

\section{RPM макросы}
\Emph{Макрос RPM} --- это именованная переменная, которая напрямую подставляет текст в \Sys{SPEC}-файл 
во время сборки \Sys{rpm}-пакета. Имена макросов начинаются с символа \%. Имена макросов --- это 
сокращённые псевдонимы для часто используемых фрагментов текста. 
	
Ниже приведены примеры макросов\footnote{\href{https://www.altlinux.org/Spec/\%D0\%9F\%D1\%80\%D0\%B5\%D0\%B4\%D0\%BE\%D0\%BF\%D1\%80\%D0\%B5\%D0\%B4\%D0\%B5\%D0\%BB\%D0\%B5\%D0\%BD\%D0\%BD\%D1\%8B\%D0\%B5_\%D0\%BC\%D0\%B0\%D0\%BA\%D1\%80\%D0\%BE\%D1\%81\%D1\%8B}{https://www.altlinux.org/Spec/Предопределённые\_макросы}}:
\begin{enumerate}
	\item \Emph{Пример макроса, содержащего значение.}
	Если во время сборки некоторым командам необходимо передать имя собираемого пакета, 
		то можно передавать им макрос \Sys{\%name}. Во время сборки этот макрос подставляет 
		имя пакета, объявленное в начале \Sys{SPEC}-файла:
	\begin{verbatim}
		%define some_macro foo
		....
		Name: bar-%some_macro
		.....
		%build
		%dune_build -p %some_macro
	\end{verbatim} 
	
	\item \Emph{Пример макроса с набором команд.}
	\Sys{\%cmake\_build} --- макрос необходимый для сборки пакетов с помощью cmake. 
		Он подставляет следующую последовательность команд: 
	\begin{verbatim}
		mkdir -p x86_64-alt-linux-gnu; 
		cmake \
		-DCMAKE_SKIP_INSTALL_RPATH:BOOL=yes \
		-DCMAKE_C_FLAGS:STRING='-O2 -g' \
		-DCMAKE_CXX_FLAGS:STRING='-O2 -g' \
		-DCMAKE_Fortran_FLAGS:STRING='-O2 -g' \
		-DCMAKE_INSTALL_PREFIX=/usr \
		-DINCLUDE_INSTALL_DIR:PATH=/usr/include \
		-DLIB_INSTALL_DIR:PATH=/usr/lib64 \
		-DSYSCONF_INSTALL_DIR:PATH=/etc \
		-DSHARE_INSTALL_PREFIX:PATH=/usr/share \
		-DLIB_DESTINATION=lib64 \
		-DLIB_SUFFIX="64" \
		-S . -B "x86_64-alt-linux-gnu"
	\end{verbatim}
\end{enumerate}

Преимущества использования макросов:
\begin{itemize}
	\item упрощение сборки;
	\item унификация \Sys{SPEC}-файлов;
	\item подбор шаблонов для создания \Sys{SPEC}-файлов;
	\item сокращение размера \Sys{SPEC}-файлов позволяет упростить отладку;
	\item использование макросов обеспечивает гибкость в настройке и конфигурации 
		пакетов, позволяя быстро изменять параметры сборки.
\end{itemize}

Где объявлены \Sys{RPM}-макросы:
\begin{itemize}
	\item Стандартные макросы объявляет установка пакета \Sys{rpmbuild (librpm)}.
	 Информацию о них можно получить из файла \Sys{/usr/lib/rpm/macros} или выполнив 
		команду: \Sys{rpm --showrc};
	 \item Макросы можно объявить самостоятельно, добавив в \Sys{SPEC}-файл;
	 \item Макросы можно объявить в отдельных файлах.
	 Команда \Sys{\%include} позволяет загрузить специальные файлы с объявленными макросами;
	 \item Макросы, объявленные в файлах, поставляемые с другими пакетами.

	 Файлы с объявленными \Sys{RPM}-макросами хранятся в каталоге

		\Sys{/usr/lib/rpm/macros.d}.

	 Например, пакет \Sys{rpm-build-ruby} содержит готовые макросы для сборки пакетов с программами, написанными на языке \Sys{Ruby}.
	 Для того чтобы использовать эти макросы, необходимо этот пакет добавить в зависимости:
	 \Sys{BuildRequires(pre): rpm-build-ruby}.
\end{itemize}

\marginalia{ex_sign_col}{Команда получения значения макроса: \Sys{rpm --eval {<имя\_макроса>}}}

\marginalia{ex_sign_col}{Некоторые макросы могут быть вложенными.}

\section{Вопросы для самопроверки}

\begin{enumerate}
\item Какие пакеты используются для сборки \Sys{rpm}-пакетов?
\item Какая утилита используется для непосредственной сборки бинарного пакета?
\item Верно ли что \Sys{hasher} обеспечивает воспроизводимую сборку в контроллируемой среде?
\item Каково внутреннее устройство \Sys{RPM}-пакета?
\item У вас есть пакет \Sys{admc-0.15.0-alt1.x86\_64.rpm}. 
	\begin{enumerate}
		\item[а)] можно ли его установить на компьтер с процессором Байкал-М? 
		\item[b)] А на компьютер с процессором Эльбрус 16С? 
		\item[c)] Какой пакет нужно взять, чтобы собрать тот же пакет под другую целевую архитектуру?
	\end{enumerate}
\item Какая структура каталогов необходима для сборки \Sys{rpm}-пакета?
\item В чём смысл понятия \Sys{buildroot}?
\item Для чего служит \Sys{SPEC}-файл?
\item Из каких обязательных частей состоит \Sys{SPEC}-файл?
	\begin{enumerate}
                \item[а)] В какой части \Sys{SPEC}-файла указывается имя пакета?
                \item[b)] В какой части \Sys{SPEC}-файла указывается лицензия под которой распространяется пакет?
                \item[c)] В какой части \Sys{SPEC}-файла указывается история изменений пакета?
        \end{enumerate}
\item Что такое \Sys{RPM}-макрос?
	\begin{enumerate}
                \item[а)] Какие преимущества даёт использование макросов?
                \item[b)] Как найти стандартные макросы?
                \item[c)] Где в системе находтся макросы, поставляемые сторонними пакетами?
        \end{enumerate}
\end{enumerate}
