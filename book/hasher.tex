\chapter{Инструмент Hasher}\label{chapter-hasher}
\Emph{Hasher} --- инструмент для сборки пакетов с использованием минимальной контролируемой среды в \Sys{chroot}. \Sys{Hasher} опирается на системный вызов \Sys{chroot} и создает изолированную среду для сборки отдельного пакета утилитой \Sys{rpm-build}.

\Sys{Hasher} облегчает поддержание сборочных зависимостей и позволяет собирать пакеты для разных дистрибутивов. \Sys{Hasher} проверяет пакеты с помощью утилиты \Sys{sisyphus\_check} и создает локальный \Sys{APT}-репозиторий с результатами сборки.

Для подготовки сборочного окружения \Sys{hasher} берет пакеты как из удалённых репозиториев, настроенных в основной системы, так и из локального репозитория (\Sys{~/hasher/repo/}), в который попадают ранее собранные пакеты.

Так же \Sys{hasher} удобен для отладки процесса сборки. Если сборка пакета прервалась, выполните команду \Sys{hsh-shell}, чтобы попасть в терминал \Sys{chroot}, исправьте ошибку и продолжайте сборку с прерванного этапа.


\section{Справочная страница hasher}
Возможности \Sys{Hasher} задокументированы в инструкциях\footnote{\href{http://uneex.ru/static/AltlinuxOrg_Hasher/}{http://uneex.ru/static/AltlinuxOrg\_Hasher/}} к пакетам \Sys{hasher} и \Sys{hasher-priv}. Для вызова справочной информации по \Sys{hasher} наберите в консоли \Sys{man package}:
\begin{verbatim}
	$ man hsh
	
	HASHER(7)                          ALT Linux                         HASHER(7)
	
	NAME
	hasher - modern safe package building technology
	
	SYNOPSIS
	hsh [options] <path-to-workdir> <package>...
	...
\end{verbatim} 

\begin{verbatim}
	$ man hasher-priv
	
	HASHER-PRIV(8)          System Administration Utilities         HASHER-PRIV(8)
	
	NAME
	hasher-priv - privileged helper for the hasher project
	
	SYNOPSIS
	hasher-priv [options] <args>
	...
\end{verbatim}

Это достаточно подробная инструкция по использованию утилиты \Sys{hasher}. Здесь вы найдёте полное описание, опции и флаги, содержимое рабочего каталога. 