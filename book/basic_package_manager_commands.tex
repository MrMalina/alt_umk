\chapter{Основные команды пакетного менеджера RPM}\label{basic-package-manager-commands}
Управлять пакетами можно из командной строки при помощи программы \Sys{rpm}, которая имеет следующий синтаксис:
\begin{verbatim}
    rpm [параметры]
\end{verbatim}

Пакетный менеджер \Sys{RPM} предоставляет базовые возможности для управления пакетами. Основной набор команд%
\footnote{\href{https://wiki.altlinux.ru/\%D0\%9A\%D0\%BE\%D0\%BC\%D0\%B0\%D0\%BD\%D0\%B4\%D1\%8B_RPM}{https://wiki.altlinux.ru/Команды\_RPM}}
позволяет установить, удалить, обновить пакеты, получить разнообразную информацию о самих пакетах и их содержимом:

\begin{itemize}
	\item \textbf{Проверка установки пакета в системе}: \Sys{rpm -q ИМЯ\_ПАКЕТА \dots}\\
	Эта команда проверяет установлен ли пакет в системе. \Sys{rpm -q} означает \Sys{query} (запрос).
	Используется с дополнительными ключами для выполнения запросов различного назначения.

	\item \textbf{Информация о пакете}: \Sys{rpm -qi ИМЯ\_ПАКЕТА \dots}\\
	\Sys{rpm -qi} выводит подробную информацию о конкретном установленном пакете. \Sys{rpm -qi} означает <<query information>>
	(запросить информацию). Но в случае \Sys{rpm -i <имя пакета>}, \Sys{-i} означает инструкцию \Sys{install} (установить).

	\item \textbf{Просмотр установленных пакетов}: \Sys{rpm -qa}\\
	Эта команда выводит список всех установленных пакетов в системе. \Sys{-a: ----all} (весь, все). \Sys{rpm -qa} означает <<query all>> (запрос всего).

	\item \textbf{Проверка зависимостей пакета}: \Sys{rpm -qR ИМЯ\_ПАКЕТА \dots}\\
	 \Sys{rpm -qR} выводит список зависимостей (другие пакеты), необходимых для работы указанного пакета. \Sys{-R: --requires}
	(нуждается). \Sys{rpm -qR} означает <<query requires>> (запрос необходимого, запрос зависимостей).

	\item \textbf{Проверка файла на принадлежность пакету}: \Sys{rpm -qf ФАЙЛ}\\
	Команда \Sys{rpm -qf} определяет к какому пакету принадлежит указанный файл. \Sys{-f: ----file}
	(файл). \Sys{rpm -qf} означает <<query file>> (запрос файла).

	\item \textbf{Просмотр файлов пакета}: \Sys{rpm -ql ИМЯ\_ПАКЕТА \dots}\\
	 \Sys{rpm -ql} выводит список всех файлов, содержащихся в установленном пакете. \Sys{-l: ----list}
	(список). \Sys{rpm -ql} означает <<query list>> (запрос списка).

	\item \textbf{Установка пакета}: \Sys{rpm -i ФАЙЛ\_ПАКЕТА}\\
	Команда \Sys{rpm -i} используется для установки пакета из файла \Sys{.rpm}. \Sys{-i: ----install}
	(установить). Например, \Sys{rpm -i package.rpm} установит содержимое пакета \Sys{package} в систему.

	\item \textbf{Удаление пакета}: \Sys{rpm -e ИМЯ\_ПАКЕТА}\\
	\Sys{rpm -e} удаляет установленный пакет. \Sys{-e: ----erase}
	(стереть). Например, \Sys{rpm -e package} удалит пакет с именем \Sys{package}.

	\item \textbf{Обновление пакета}: \Sys{rpm -U ФАЙЛ\_ПАКЕТА} или \Sys{rpm -F ФАЙЛ\_ПАКЕТА}\\
	Команда \Sys{rpm -U} обновит пакет до новой версии или установит пакет, \Sys{rpm -F}
	только обновит пакет, если он уже установлен. \Sys{-U: ----upgrade} (обновить), \Sys{-F: ----freshen} (освежить).

	\item \textbf{Проверка целостности пакета}: \Sys{rpm -V ИМЯ\_ПАКЕТА | ФАЙЛ\_ПАКЕТА \dots}\\
	\Sys{rpm -V} проверяет целостность файлов в пакете, сравнивая их с информацией в базе данных \Sys{rpm},
	\Sys{-V: ----verify} (проверить).

	Дополнительные ключи:
	\begin{itemize}
		\item \textbf{\Sys{-v: ----verbose}} (подробно). Подробный вывод, если доступно. Например, \Sys{rpm -iv} означает <<install verbose>>
	 (установка подробно) и используется для вывода более подробной информации в процессе установки пакета. Подробный вывод существует не для всех ключей утилиты.
		\item \textbf{\Sys{----quiet}}: (тихо). Вывести минимальный набор сообщений, если возможно. Например, \Sys{rpm -i ----quiet} означает <<install quiet>> (установка тихая).
	\end{itemize}

	\marginalia{ex_sign_col}{Справку по ключам можно получить, набрав в консоли команду \Sys{rpm ----help}.}

\end{itemize}

\section{Установка RPM-пакета}
\marginalia{ex_sign_col}{В команде должен быть указан файл пакета или полный путь к нему.}

Для установки пакета из \Sys{rpm}-файла используйте команду:
\begin{verbatim}
    # rpm -i ФАЙЛ_ПАКЕТА ...
\end{verbatim}

Синтаксис команды:
\begin{verbatim}
    # rpm {-i | ----install} [опции-установки] ФАЙЛ_ПАКЕТА ...
\end{verbatim}

\marginalia{ex_sign_col}{Для работы с командой потребуются права суперпользователя. Их можно получить через команду \Sys{su-} либо команду \Sys{sudo}}

Пример выполнения команды\footnote{\href{https://wiki.altlinux.ru/\%D0\%9A\%D0\%BE\%D0\%BC\%D0\%B0\%D0\%BD\%D0\%B4\%D1\%8B_RPM}{https://wiki.altlinux.ru/Команды\_RPM}}:

\begin{verbatim}
    # rpm -i gpupdate-0.9.12.6-alt1.src.rpm
\end{verbatim}

В конце команды возможно указать дополнительные опции:
\begin{itemize}
	\item \Sys{----nodeps} --- не проверять зависимости пакета;
	\item \Sys{----replacepkgs} или \Sys{----reinstall} --- переустановить пакет.
\end{itemize}

Подробный вывод.
Для отображения прогресса установки используйте дополнительные параметры \Sys{-v} и \Sys{-h}.
\begin{itemize}
	\item \Sys{-v} --- вывести детальные сообщения;
	\item \Sys{-h} --- вывести <<\#>> строку индикатора прогресса по мере установки пакета (используется с \Sys{-v}).
\end{itemize}

Пример выполнения команды:
\begin{verbatim}
    # rpm -ivh gpupdate-0.9.12.6-alt1.src.rpm

        Подготовка...
        ################################[100%]

        Обновление / установка...

        1: gpupdate-0.9.12.6-alt1.src.rpm
        ################################[100%]
\end{verbatim}


\section{Проверка установки пакета в системе}\label{verifying-package}
Чтобы проверить, установлен ли пакет, введите следующую команду:
\begin{verbatim}
    $ rpm -q ИМЯ_ПАКЕТА
\end{verbatim}

Пример:
\begin{verbatim}
    $ rpm -q gpupdate

        gpupdate-0.9.12.2-alt2.noarch

    $ rpm -q mediinfo

        пакет mediinfo не установлен
\end{verbatim}

\section{Просмотр файлов пакета, установленного в системе}
Чтобы получить список файлов пакета, введите следующую команду:
\begin{verbatim}
    $ rpm -ql ИМЯ_ПАКЕТА ...
\end{verbatim}

В этой команде используется ключ <<\Sys{-l}>> (list).

\marginalia{ex_sign_col}{Для развернутой информации укажите ключ <<\Sys{-i}>>.}

Пример использования:
\begin{verbatim}
    $ rpm -ql admc
        /usr/bin/admc
        /usr/lib64/libadldap.so
        /usr/share/applications/admc.desktop
        /usr/share/doc/admc-0.11.2
        /usr/share/doc/admc-0.11.2/CHANGELOG.txt
        /usr/share/doc/admc-0.11.2/CHANGELOG_ru.txt
        /usr/share/doc/admc-0.11.2/README.md
        /usr/share/icons/hicolor/scalable/apps/admc.svg
        /usr/share/man/man1/admc.1.xz
\end{verbatim}

Чтобы узнать содержимое неустановленного \Sys{rpm}-пакета, используйте команду:
\begin{verbatim}
    $ rpm -qlp ИМЯ_ПАКЕТА ...
\end{verbatim}

Пример выполнения команды:
\begin{verbatim}
    $ rpm -qlp udisks2-2.9.4-alt1.1.src.rpm
        udisks-2.9.4.tar.bz2 udisks2.control udisks2.spec
\end{verbatim}

\section{Просмотр недавно установленных пакетов}\label{view-recently-installed-packages}
Чтобы получить список последних установленных пакетов, введите следующую команду:

\begin{verbatim}
    $ rpm -qa ----last | head
\end{verbatim}

Вывод:
\begin{verbatim}
smplayer-23.6.0-alt2.10169.x86_64             Пт 01 дек 2023 12:45:39
qt5-wayland-5.15.10-alt1.x86_64               Пт 01 дек 2023 12:35:44
qt5-tools-5.15.10-alt2.x86_64                 Пт 01 дек 2023 12:35:44
qt5-sql-mysql-5.15.10-alt1.x86_64             Пт 01 дек 2023 12:35:44
qt5-dbus-5.15.10-alt2.x86_64                  Пт 01 дек 2023 12:35:44
libqt5-xmlpatterns-5.15.10-alt1.x86_64        Пт 01 дек 2023 12:35:44
libqt5-x11extras-5.15.10-alt1.x86_64          Пт 01 дек 2023 12:35:44
libqt5-webenginewidgets-5.15.15-alt1.x86_64   Пт 01 дек 2023 12:35:44
libqt5-test-5.15.10-alt1.x86_64               Пт 01 дек 2023 12:35:44
libqt5-quickparticles-5.15.10-alt1.x86_64     Пт 01 дек 2023 12:35:44
\end{verbatim}

Команда \Sys{rpm -qa ----last} используется для вывода списка всех установленных пакетов, отсортированных
по времени их установки. Пакеты будут отсортированы в порядке убывания времени установки --- самые
последние установленные пакеты отобразятся в верхней части списка.

Фильтрация вывода: утилита \Sys{grep} отфильтрует вывод и поможет найти искомый пакет. Например,
следующая команда выведет информацию только о тех пакетах, название которых содержит <<\Sys{kernel}>>:

\begin{verbatim}
    $ rpm -qa ----last | grep kernel
\end{verbatim}

\section{Поиск пакета в системе}
Чтобы найти в системе необходимый пакет среди уже установленных, используйте утилиту \Sys{grep}. Утилита \Sys{grep} находит строки по запросу.
\begin{verbatim}
    $ rpm -qa | grep ИМЯ_ПАКЕТА
\end{verbatim}

Например, запрос:
\begin{verbatim}
    $ rpm -qa | grep smplayer
    kde5-smplayer-common-22.7.0-alt1.noarch
    smplayer-23.6.0-alt2.10169.x86_64
\end{verbatim}

\section{Проверка файла, относящегося к пакету}
Чтобы определить, какому пакету принадлежит указанный файл, используйте команду:
\begin{verbatim}
    $ rpm -qf ФАЙЛ
\end{verbatim}

Например, запрос:
\begin{verbatim}
    $ rpm -qf /usr/share/FBReader/help
\end{verbatim}
Предоставит вывод:
\begin{verbatim}
    fbreader-0.99.5-alt6.x86_64
\end{verbatim}

\section{Вывод информации о пакете}
Чтобы получить вывод подробной информации о конкретном установленном пакете\footnote{\href{https://www.inp.nsk.su/~bolkhov/teach/inpunix/make_rpm.ru.html}{https://www.inp.nsk.su/$\sim$bolkhov/teach/inpunix/make\_rpm.ru.html}} --- название, версию и прочее --- используйте команду:
\begin{verbatim}
    $ rpm -qi ИМЯ_ПАКЕТА ...
\end{verbatim}

\marginalia{ex_sign_col}{В запросе указывается имя пакета из списка уже установленных в системе, либо путь к загруженному пакету.}

\marginalia{ex_sign_col}{Если в системе установлена одна версия пакета, можно указывать только имя. Если
установлено больше одной версии, необходимо указывать конкретную версию и релиз(\Sys{имя-версия-релиз}).}


Пример выполнения команды для установленного пакета в системе:
\begin{verbatim}
    $ rpm -qi gpupdate

        Name        : gpupdate
        Version     : 0.9.12.6
        Release     : alt1
        DistTag     : p10+323062.200.4.1
        Architecture: noarch
        Install Date: Ср 12 июл 2023 13:26:50
        Group       : Other
        Size        : 951193
        License     : GPLv3+
        Signature   : RSA/SHA512, Сб 08 июл 2023 22:59:22, Key ID 42f343a2c7eb80f9
        Source RPM  : gpupdate-0.9.12.6-alt1.src.rpm
        Build Date  : Сб 08 июл 2023 22:59:21
        Build Host  : greh-p10.hasher.altlinux.org
        Relocations : (not relocatable)
        Packager    : Valery Sinelnikov (ALT) <greh@altlinux.org>
        Vendor      : ALT Linux Team
        URL         : https://github.com/altlinux/gpupdate
        Summary     : GPT applier
        Description :
        gpupdate is the facility to apply various GPO/GPT settings retrieved
        from Active Directory domain in UNIX environment.
\end{verbatim}

Пример выполнения команды для \Sys{rpm}-пакета, расположенного на диске, но неустановленного в системе:
\begin{verbatim}
    $ rpm -qip /путь/к/пакету/имя_пакета-версия-архитектура.rpm
\end{verbatim}

Пример выполнения:
\begin{verbatim}
    $ rpm -qip kooha-2.2.4-alt1.x86_64.rpm

        Name        : kooha
        Epoch       : 1
        Version     : 2.2.4
        Release     : alt1
        DistTag     : sisyphus+330279.100.2.2
        Architecture: x86_64
        Install Date: (not installed)
        Group       : Video
        Size        : 2261140
        License     : GPL-3.0+
        Signature   : RSA/SHA512, Пн 25 сен 2023 15:28:07, Key ID ff979dedda2773bb
        Source RPM  : kooha-2.2.4-alt1.src.rpm
        Build Date  : Пн 25 сен 2023 15:28:01
        Build Host  : lvol-sisyphus.hasher.altlinux.org
        Relocations : (not relocatable)
        Packager    : Leontiy Volodin <lvol@altlinux.org>
        Vendor      : ALT Linux Team
        URL         : https://github.com/SeaDve/Kooha
        Summary     : Simple screen recorder with a minimal interface
        Description :
        Simple screen recorder with a minimal interface.
\end{verbatim}

\section{Обновление пакета}
Чтобы обновить пакет до новой версии, используйте команду:
\begin{verbatim}
    # rpm -U ФАЙЛ_ПАКЕТА ...
\end{verbatim}

Для обновления пакетов в пакетном менеджере \Sys{RPM} используют запрос с двумя типами ключей\footnote{\href{https://wiki.altlinux.ru/QuickStart/PkgManagment}{https://wiki.altlinux.ru/QuickStart/PkgManagment}}:
\begin{enumerate}
	\item \Sys{-U} --- может устанавливать и обновлять пакеты;
	\item \Sys{-F} --- только обновляет уже установленные пакеты.
\end{enumerate}

\begin{verbatim}
    # rpm {-U | ----upgrade} [опции-установки] ФАЙЛ_ПАКЕТА ...
\end{verbatim}

Команда \Sys{rpm -U} (Upgrade) обновляет пакеты до новых версий. Если установлена старая версия пакета, \Sys{rpm -U} заменит старую версию на новую, обновив пакет. Если пакет не установлен, команда \Sys{rpm -U} установит его.

\begin{verbatim}
    # rpm {-F | ----freshen} [опции-установки] ФАЙЛ_ПАКЕТА ...
\end{verbatim}

Команда \Sys{rpm -F} (Freshen) обновляет только те пакеты, которые уже установлены в системе. Если установлена старая версия пакета, \Sys{rpm -F} заменит старую версию на новую, обновив пакет. Если пакет не установлен, команда \Sys{rpm -F} ничего не сделает с этим пакетом\footnote{\href{https://access.redhat.com/solutions/1189}{https://access.redhat.com/solutions/1189}}.

Пример установки пакета:
\begin{verbatim}
    # rpm -U foo-1.0-i586.rpm
\end{verbatim}


\section{Вопросы для самопроверки}

\begin{enumerate}
	\item С помощью какого ключа к \Sys{rpm} можно получить информацию об установленном \Sys{rpm}-пакете?
	\item Как получить информацию о \Sys{rpm}-пакете, если он еще не установлен в системе, но у вас есть файл пакета?
	\item Как получить список всех файлов \Sys{rpm}-пакета?
	\item Верно ли что команда \Sys{rpm -Uvh} обновит пакет, а если его нет, то установит его?
	\item Какой ключ отвечает за обновление/установку?
	\item Как проверить установлен ли пакет с именем \Sys{foo} в вашей системе?
	\item Что сделает команда \verb!$ rpm -qi mate-text-editor | grep License!?
\end{enumerate}
