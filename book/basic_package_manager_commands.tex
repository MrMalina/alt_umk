\chapter{Основные команды пакетного менеджера}\label{basic-package-manager-commands}
\section{Просмотр недавно установленных пакетов}\label{view-recently-installed-packages}
Чтобы получить список последних установленных пакетов, введите следующую команду: 

\begin{verbatim}
rpm -qa --last|head	
\end{verbatim}

Вывод:
\begin{verbatim}
smplayer-23.6.0-alt2.10169.x86_64             Пт 01 дек 2023 12:45:39
qt5-wayland-5.15.10-alt1.x86_64               Пт 01 дек 2023 12:35:44
qt5-tools-5.15.10-alt2.x86_64                 Пт 01 дек 2023 12:35:44
qt5-sql-mysql-5.15.10-alt1.x86_64             Пт 01 дек 2023 12:35:44
qt5-dbus-5.15.10-alt2.x86_64                  Пт 01 дек 2023 12:35:44
libqt5-xmlpatterns-5.15.10-alt1.x86_64        Пт 01 дек 2023 12:35:44
libqt5-x11extras-5.15.10-alt1.x86_64          Пт 01 дек 2023 12:35:44
libqt5-webenginewidgets-5.15.15-alt1.x86_64   Пт 01 дек 2023 12:35:44
libqt5-test-5.15.10-alt1.x86_64               Пт 01 дек 2023 12:35:44
libqt5-quickparticles-5.15.10-alt1.x86_64     Пт 01 дек 2023 12:35:44
\end{verbatim}

Команда \Sys{rpm -qa --last} используется для вывода списка всех установленных пакетов, отсортированных 
по времени их установки. Пакеты будут отсортированы в порядке убывания времени установки --- самые 
последние установленные пакеты отобразятся в верхней части списка.

Фильтрация вывода: утилита \Sys{grep} отфильтрует вывод и поможет найти искомый пакет. Например, 
следующая команда выведет информацию только о тех пакетах, название которых содержит <<\Sys{kernel}>>:

\begin{verbatim}
rpm -qa --last | grep kernel
\end{verbatim} 
