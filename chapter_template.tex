\chapter{Заголовок главы (шаблон)}

Далее следует текст главы. Абзацы в тексте разделяются пустой
строкой. Между словами должен быть как минимум один пробел. 

В структуру главы могут входить разделы и
подразделы.

\section{Правила набора}

Следует различать дефис, длинное тире и короткое тире. Вот эта конструкция "--- длинное
тире, оно употребляется в тексте. Обратите внимание, что перед ней не требуется ставить
неразрывный пробел. Короткое тире ставится между числами: 2006--2006 г.

Текстовые кавычки можно ставить <<ёлочками>> или использовать
символы «Unicode».

\subsection{Правила выделения элементов текста}

Выделение текста производится с помощью команд. Выделяемый текст
помещается в фигурные скобки сразу после имени команды. Обычное
текстовое выделение \emph{курсивом} или \textbf{полужирным шрифтом}
рекомендуется употреблять только для логического выделения. Выделение
по формальным признакам осуществляется командами, представленными в
нижеследующем списке, который к тому же представляют собой пример оформления
маркированного списка.

\begin{itemize}
\item адреса URL \url{http://somewhere.org};
\item англоязычные вставки \EN{english phrase};
\item команды и переменные shell или cmd \shell{grep}
\item имя файла/каталога \file{/etc/xorg.conf}
\item пункт меню или вкладка меню \menu{Help}
\item клавиша \key{F3}
\item пара клавиш \dkey{Alt}{F12}
\item тройка клавиш \tkey{Ctrl}{Alt}{Del}
\item панель инструментов и её элементы \panel{}
\item флажок \checkbox{}
\item список выбора \select{}
\item переключатель \radio{}
\item текстовое поле в меню \tarea{}
\item счётчик (в диалоговом окне) \counter{}
\item название окна \window{}
\item кнопка \button{}
\end{itemize}

Допускается создание и использование
любых заранее определённых в шаблоне
или самостоятельно созданных команд,
если они последовательно применяются
для разметки специфических элементов
текста (например, названий функций 
некой программы, описываемой в книге).
В этом случае при передаче текста 
редактору, пожалуйста,
прокомментируйте смысл и случаи 
употребления каждой команды.

\section{Иллюстрации, таблицы, листинги, примеры}
 
Требования к иллюстрациям (пример оформления нумерованного списка):

\begin{enumerate}
\item \emph{Чёрно-белые}\footnote{Обратите внимание, что в наших изданиях принято
  последовательно употреблять букву \emph{ё}.}. Если оригиналы у вас цветные, то стоит
  самостоятельно конвертировать в оттенки серого и посмотреть, что
  стало с контрастностью и насыщенностью, в случае необходимости
  подправить.
\item \emph{Ширина} изображений не должна превышать 110\,мм,
  \emph{высота} "--- не более 160\,мм.
\item \emph{Растровые} -- в формате без потери качества (TIFF), 300 dpi.
\item \emph{Векторные} -- в векторном формате, трансформируемом без потерь в
  eps (лучше всего сразу в eps).
  Необходимо следить за тем, чтобы векторные по происхождению изображения
  (например, графики, построенные различными математическими пакетами), \emph{не 
  были} при сохранении в eps преобразованы в растровые. Такие рисунки необходимо
  представить в векторном формате, лучше всего использовать 
  экспорт из соответствующего пакета непосредственно в eps. Ни в коем случае
  не следует преобразовать такие изображения в eps из растровых (png, jpg...)
  или при помощи редактора растровой графики (GIMP).
\item Если на изображении есть \emph{текст}, стоит проверить, что при
  данном размере (на печати) буквы не слишком маленькие.
\end{enumerate}


% Вставка иллюстраций. Знак процента в начале строки обозначает комментарий.
% \begin{figure}
%   \centering
%   \includegraphics{example.eps}
% \caption{Подпись к иллюстрации (необязательно). Точка в конце не ставится}
% \end{figure}

\section{Библиография и ссылки на источники}

Оформление ссылок\cite{brooks}. Простое оформление библиографии
приведено ниже, однако не возбраняется и даже приветствуется
использование BibTeX.

\begin{thebibliography}{9}
\bibitem{brooks} \textit{Брукс Ф.} Мифический человеко-месяц,
  или как создаются программные системы. \url{http://www.lib.ru/CTOTOR/BRUKS/mithsoftware.txt}
\end{thebibliography}

Желаем приятной работы в издательской системе \LaTeX! :)

%%% Local Variables: 
%%% mode: latex
%%% TeX-master: t
%%% End: 
