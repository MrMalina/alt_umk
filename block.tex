\documentclass[final,10pt,70x100,auto]{altlibrary}
%\documentclass[draft,10pt,70x100,auto]{altlibrary}
%\usepackage[utf8x]{inputenc}
%\usepackage[T2A]{fontenc}
%%\usepackage[english,russian]{babel}
% A4 brochure (for printed docs); class options intended for automatic layout
% \documentclass[final,a4brochure,auto]{altlibrary}
% for ALT docs with m-k sources use additinal package:

%% For m-k modules
\usepackage{m-k} 


%% For DocBook modules
\usepackage[%pdftex,%
bookmarks=true,%
bookmarksnumbered=true,% 
hypertexnames=false,%
final,%
breaklinks=true,%
colorlinks=true,%
linkcolor=blue,%
unicode]{hyperref}
\usepackage{fancyvrb}
\usepackage{fvextra}
\usepackage{pmboxdraw}
\usepackage{memhfixc}
\usepackage{graphicx}
\usepackage{theorem}
% \usepackage{dbsimple}
% \usepackage{tabulary}
\usepackage{longtable}
\usepackage{multirow}
% \usepackage{supertabular}
\usepackage{amsmath}
\usepackage{amssymb}
% \usepackage{array}
\usepackage{color}
\usepackage{enumerate}
%\usepackage{lmodern}
\usepackage[final]{listings}
\usepackage[status=draft,layout=footnote,marginclue,innerlayout={layout=marginnote}]{fixme}
% local setup and definitions
%\input{local}
\fxusetheme{color}
\graphicspath{{img/}}
\DeclareFontShape{OT1}{cmtt}{bx}{n}{<5><6><7><8><9><10><10.95><12><14.4><17.28><20.74><24.88>cmttb10}{}
%\DeclareFontShape{T2A}{cmtt}{m}{n}{<5><6><7><8><9><10><10.95><12><14.4><17.28><20.74><24.88>cmttb10}{}

\lstloadlanguages{XML,HTML}
\lstset{
    language=XML,
%    columns=fixed,
    otherkeywords={},
    basicstyle=\usefont{T2A}{cmtt}{m}{n},
    keywordstyle=\usefont{OT1}{cmtt}{bx}{n},
    escapechar=`,
    basicstyle=\small,
%    commentstyle=\itshape\small
}

{\theorembodyfont{\normalfont\fontshape{ui}\selectfont}
 \newtheorem{tasko}{Пример задачи}}

{\theorembodyfont{\normalfont\selectfont
        \addtocounter{tasko}{-1}}
 \newtheorem{solvo}[tasko]{Решение задачи}}

{\theorembodyfont{\normalfont\fontshape{ui}\selectfont}
    \newtheorem{problemo}{Задание}}

\newcommand{\erf}{\mathrm{erf}}
\newcommand{\erfc}{\mathrm{erfc}}
\newcommand{\len}{\mathrm{len}}


\begin{document}
\mainmatter
\thispagestyle{empty}
%\vspace{\stretch{1}}
\begin{center} 
{\sffamily
{\bfseries\normalsize МИНИСТЕРСТВО ЦИФРОВОГО РАЗВИТИЯ, СВЯЗИ И МАССОВЫХ\\ 
КОММУНИКАЦИЙ РОССИЙСКОЙ ФЕДЕРАЦИИ}
\par\vspace{\baselineskip}
{\bfseries\normalsize Ордена Трудового Красного Знамени федеральное государственное бюджетное образовательное учреждение высшего образования\\
Московский технический университет связи и информатики}

%\par\vspace{4\baselineskip}
\vspace{\stretch{1}}

{\bfseries\huge УЧЕБНО-МЕТОДИЧЕСКОЕ ПОСОБИЕ}
\par\vspace{\baselineskip}
{\bfseries\Large «Инфраструктура разработки программных пакетов\\и сборки программного обеспечения»}
\par\vspace{\baselineskip}

{\normalsize для очного и дистанционного режимов обучения магистратуры и дополнительного профессионального образования

\par\vspace{4\baselineskip}
%___________________________________________________________________
\hrule height 1pt
\smallskip
Базовая кафедра общественно-государственного объединения 
«Ассоциация документальной электросвязи»
«Технологии электронного обмена данными» в Московском техническом университете связи и информатики


\vspace{\stretch{1}}

2023}
}
\end{center}

\tableofcontents*
\clearpage
\chapter*{Введение}\addcontentsline{toc}{chapter}{Введение}
\markboth{Введение}{Введение}

Учебно-методическое пособие <<Инфраструктура разработки программных пакетов и сборки 
программного обеспечения>> предназначено для очного и дистанционного обучения студентов 
в рамках гуманитарно-технологической платформы по программам магистратуры и 
дополнительного профессионального образования <<Информационная культура цифровой 
трансформации>> базовой кафедры общественно-государственного объединения <<Ассоциация 
документальной электросвязи>> (АДЭ) <<Технологии электронного обмена данными>> (ТЭОД) 
в Московском техническом университете связи и информатики (МТУСИ). 

Пособие состоит из введения, заключения, шести глав и содержит практикумы по следующим направлениям:
\begin{itemize}
	\item Пакетный менеджер.
	\item Основные команды пакетного менеджера.
	\item Программное обеспечение, используемое для упаковки пакетов.
	\item Инструмент Gear.
	\item Инструмент Hasher.
	\item Примеры использования инструментов ОС <<Альт>> для сборки пакетов.
\end{itemize}

Пособие включает сведения о программной платформе лабораторного практикума на базе 
отечественных операционных систем семейства <<Альт>>, представленного компанией <<Базальт СПО>>, 
единственного российского разработчика системного программного обеспечения, создавшего 
собственную технологическую среду распределённой коллективной разработки и обеспечения 
жизненного цикла программного обеспечения (<<Альт платформа>>).

Разработки дистрибутивов операционных систем компании <<Базальт СПО>> основаны на 
отечественной инфраструктуре разработки <<Сизиф>> (Sisyphus), которая находится на 
территории РФ, принадлежит и поддерживается компанией <<Базальт СПО>>. В основе 
Sisyphus лежат технологии сборки компонентов системы и учёта зависимостей между ними, а
также отработанные процессы по взаимодействию разработчиков. На базе репозитория
периодически формируется стабильный репозиторий пакетов (программная платформа), которая поддерживается
в течение длительного времени и используется в качестве базы для построения дистрибутивов 
линейки <<Альт>> и обеспечения их жизненного цикла.

ООО <<Базальт СПО>> выпускает линейку дистрибутивов разного назначения для различных аппаратных архитектур.
В репозитории Sisyphus поддерживаются архитектуры: i586, x86\_64, armh (armv7), aarch64 (armv8), 
Эльбрус (с третьего по шестое поколение), riscv64, mipsel, loongarch.

Часть дистрибутивов включена в \href{reestr.digital.gov.ru}{Единый реестр российских программ} для электронных 
вычислительных машин и баз данных --- это <<Альт СП>>, имеющий сертификаты ФСТЭК России, 
Минобороны России и ФСБ России, <<Альт Виртуализация>>, <<Альт Сервер>>, <<Альт Рабочая станция>>, 
<<Альт Образование>>, так и другие, бесплатные и свободные: Simply Linux, различные стартеркиты 
(Starterkits) и регулярные (regular) сборки. Дистрибутив --- это составное произведение, 
в составе которого есть программа для дистрибуции (установки), называемая инсталлятор, и 
набор системного и прикладного ПО. В основе всех дистрибутивов лежат пакеты свободного 
программного обеспечения.

Свободное программное обеспечение (СПО) --- это программное обеспечение, распространяемое 
на условиях простой (неисключительной) лицензии, которая позволяет пользователю:
\begin{enumerate}
	\item использовать программу для ЭВМ в любых, не запрещённых законом, целях;
	\item получать доступ к исходным текстам (кодам) программы как в целях изучения и адаптации, 
	так и в целях переработки программы для ЭВМ; распространять программу (бесплатно или за плату, по своему усмотрению);
	\item вносить изменения в программу для ЭВМ (перерабатывать) и распространять экземпляры изменённой (переработанной) 
	программы с учётом возможных требований наследования лицензии;
	\item в отдельных случаях (copyleft лицензия) распространять модифицированную компьютерную программу пользователем 
	на условиях, идентичных тем, на которых ему предоставлена исходная программа.
\end{enumerate}

Примерами свободных лицензий являются:
\begin{enumerate}
	\item \href{https://www.gnu.org/licenses/gpl-3.0.html}{\Sys{GNU general public license}}. Version 3, 29 June 2007 (Стандартная общественная лицензия GNU. Версия 3, от 29 июня 2007 г.).
	\item \href{https://en.wikipedia.org/wiki/BSD_licenses}{\Sys{BSD license}}, New Berkley Software Distribution license (Модифицированная программная лицензия университета Беркли).
\end{enumerate}

СПО отлично подходит для целей обучения и для разработки собственных решений, потому что весь 
код доступен для изучения и модификации, однако авторы настоятельно советуют всем, кто использует 
СПО для построения своих программных продуктов, учитывать особенности лицензирования не только самих 
пакетов, но и входящих в их состав библиотек. Если вы используете copyleft библиотеку, 
это обязывает вас распространять свою программу под аналогичной лицензией, --- любой человек, который 
получит вашу программу легальным способом, может потребовать предъявить ему исходный код.

В работе над пособием принимал участие коллектив <<Базальт СПО>>:\\
М.\,А.\,Фоканова, М.\,О.\,Алексеева, А.\,А.\,Калинин, И.\,А.\,Мельников, Д.\,Н.\,Воропаев, А.\,А.\,Лимачко, В.\,А.\,Синельников, В.\,А.\,Соколов, В.\,П.\,Кашицин, А.\,В.\,Абрамов. Под редакцией В.\,Л.\,Черного. 


\chapter{Пакетный менеджер}\label{package-manager}
Операционная система состоит из разнообразных компонентов: программ, библиотек, скриптов и приложений.
Число компонентов может достигать тысячи единиц, в каждой из которых могут быть включены десятки файлов.
Для удобства работы пользователя системные компоненты в \Sys{Linux} представлены в виде
пакетов\footnote{\href{https://docs.altlinux.org/books/altlibrary-linuxintro2.pdf}{Курячий Г., Маслинский К. (2010)}.
	Операционная система Linux. Курс лекций. ДМК Пресс.}. Пакет объединяет в общий архив сходные по назначению файлы: исполняемые программы, наборы библиотек, скрипты или конфигурационные файлы и данные.
Пользователь выбирает программы, ориентируясь на общеизвестное имя, устанавливает, обновляет, проверяет,
удаляет их, не вдаваясь в отдельные детали подбора всех необходимых файлов и компонентов.
Работа с пакетами позволяет сохранять целостность программы со всеми её компонентами.

\Emph{Пакет} --- это специально подготовленный архив, содержащий файлы данных, конфигурационные файлы,
управляющую информацию и мета данные. Метаданные пакета содержат полное имя, номер версии, принадлежность
архитектуре, цифровую подпись, описание пакета, информацию о лицензии и некоторую служебную информацию
о сборке. Управляющая информация пакета содержит сценарии установки и удаления пакета, зависимости
устанавливаемого пакета от других пакетов, краткое описание и прочую информацию, которую использует
менеджер пакетов. Пакеты принято хранить в специальном хранилище --- \Emph{репозитории пакетов}.

Для удобства работы команды разработчиков придумали собственные форматы архивов:

\begin{itemize}
	\item \Sys{RPM (.rpm)}. Разработан компанией \Sys{Red Hat}. Применяется в системе \Sys{<<Альт>>}, \Sys{Ред ОС}, \Sys{RHEL} и \Sys{CentOS}.
	\item \Sys{DEB (.deb)}. Формат пакетов дистрибутива \Sys{Debian}, а также \Sys{Ubuntu}.
	\item \Sys{TAR.XZ}. Применяется в дистрибутивах \Sys{ArchLinux} и \Sys{Manjaro}.
	\item \Sys{APK (.apk)}. Применяется в операционной системе \Sys{Android}.
\end{itemize}

Каждый пакет определяется именем, архитектурой системы, под которую он собран,
номером её версии и номером релиза этой программы в дистрибутиве. Если пакет не зависит
от архитектуры процессора, то в качестве архитектуры указывается <<noarch>>.

Например, \Sys{admc-0.15.0-alt1.x86\_64.rpm}:

\noindent
\hspace{0.2cm}
\begin{tabular}{ll}
	Имя: & \Sys{admc} \\
	Номер версии: & \Sys{0.15.0}\\
	Номер релиза: & \Sys{alt1}\\
	Архитектура: &  \Sys{x86\_64}\\
\end{tabular}

\hypertarget{require-def}{\Emph{Зависимость пакета}} --- потребность компонентов в составе пакета в ресурсах или компонентах прочих пакетов.
Может случиться, что для успешного запуска программы из одного пакета необходимы библиотеки или другие ресурсы,
которые находятся в другом пакете. В таком случае говорят о зависимости пакета от одного или нескольких пакетов.
Пакетный менеджер запретит установку пакета в систему без установки всех необходимых пакетов,
удовлетворяющих зависимости.

\Emph{Пакетный менеджер (система управления пакетами)} --- это система управления: установкой, удалением, настройкой
и обновлением пакетов. Пакетные менеджеры средствами входящих в их состав утилит упрощают для пользователя
процесс управления пакетами в операционной системе. Пакетный менеджер ведёт учёт пакетов, установленных в системе.
Существует менеджер зависимостей --- специальная программа, подбирающая пакеты, зависимые друг от друга, и
загружающая эти пакеты из
хранилища\footnote{\href{https://static-sl.insales.ru/files/1/3828/14544628/original/B-BHV-6630_part.pdf}
	{Кетов Д. (2021). Внутреннее устройство Linux. 2-е изд,. перераб. и доп. БХВ-Петербург.}}.
Менеджер зависимостей подбирает правильные версии пакетов и определяет порядок их установки.
При помощи менеджера зависимостей можно узнать с каким пакетом поставляется тот или иной файл.

Задачи пакетного менеджера:

\begin{itemize}
	\item \Emph{установка программ}. Позволяет устанавливать программы из центрального хранилища или из локальных источников;
	\item \Emph{обновление программ}. Позволяет обновлять установленные программы до последних версий, представленных в хранилище;
	\item \Emph{удаление программ}. Позволяет безопасно удалять программы и все связанные с ними файлы;
	\item \Emph{управление зависимостями}. Автоматически устанавливает и управляет зависимостями программ;
	\item \Emph{проверка целостности пакетов}. Предотвращает конфликты при установке новых программ, обеспечивая целостность системы.
\end{itemize}

Утилиты пакетного менеджера позволяют:

\begin{itemize}
	\item узнать информацию о пакете;
	\item определить пакет, которому принадлежит установленная программа;
	\item определить список компонентов, установленных из указанного пакета.
\end{itemize}

Среди утилит пакетного менеджера можно выделить две категории --- низкоуровневые и высокоуровневые.

\begin{itemize}
	\item \Emph{Низкоуровневые утилиты пакетного менеджера}. Используются для установки
	локальных пакетов, загруженных вручную пользователем или высокоуровневым пакетным менеджером.
	\item \Emph{Высокоуровневые утилиты}. Применяются для поиска и скачивания пакетов из репозиториев.
	В процессе работы могут задействовать низкоуровневые менеджеры для установки загруженных программ.
\end{itemize}

В операционной системе \Sys{<<Альт>>} используется формат пакетов \Sys{.rpm}.
Пакеты \Sys{rpm} хранятся в удалённом хранилище.
Для работы с такими пакетами применяется низкоуровневый пакетный менеджер \Sys{RPM}
и консольныe утилиты \Sys{APT} (Advanced Packaging Tool)%
\footnote{\href{https://wiki.altlinux.ru/QuickStart/PkgManagment\#\%D0\%9E\%D1\%81\%D0\%BD\%D0\%BE\%D0\%B2\%D0\%BD\%D1\%8B\%D0\%B5_\%D0\%B8\%D0\%BD\%D1\%81\%D1\%82\%D1\%80\%D1\%83\%D0\%BC\%D0\%B5\%D0\%BD\%D1\%82\%D1\%8B_\%D0\%B4\%D0\%BB\%D1\%8F_\%D1\%83\%D0\%BF\%D1\%80\%D0\%B0\%D0\%B2\%D0\%BB\%D0\%B5\%D0\%BD\%D0\%B8\%D1\%8F_\%D0\%BF\%D0\%B0\%D0\%BA\%D0\%B5\%D1\%82\%D0\%B0\%D0\%BC\%D0\%B8}{https://wiki.altlinux.ru/QuickStart/PkgManagment}}.

\begin{itemize}
	\item \Emph{RPM} используется для просмотра, сборки, установки, инспекции,
		проверки, обновления и удаления отдельных программных пакетов. Каждый такой пакет состоит
		из набора файлов и информации о пакете, включающей название, версию, описание пакета и~т.\,д.
	\item \Emph{APT} умеет автоматически
		разрешать зависимости при установке, обеспечивает установку из нескольких источников и целый
		ряд других уникальных возможностей, включая получение последней версии списка пакетов из
		репозитория и обновление системы.
\end{itemize}

\section{RPM: основной пакетный менеджер в <<Альт Платформа>>}
В дистрибутивах \Sys{<<Альт>>} применяется пакетный менеджер \Sys{RPM}. \Sys{RPM Package Manager} ---
это семейство пакетных менеджеров, применяемых в различных дистрибутивах \Sys{GNU/Linux}.
Практически каждый крупный проект, использующий \Sys{RPM}, имеет свою версию пакетного менеджера,
отличающуюся от остальных.

Различия между представителями семейства \Sys{RPM} выражаются в:

\begin{itemize}
	\item наборе макросов, используемых в \Sys{.spec-файлах};
	\item различии сборки \Sys{rpm}-пакетов <<по умолчанию>> --- при отсутствии каких-либо
	указаний в \Sys{.spec}-файлах, формате строк зависимостей;
	\item отличиях в семантике операций (например, в операциях сравнения версий пакетов);
	\item отличиях в формате файлов.
\end{itemize}

Обратим внимание на то, что в операционных системах ALT ведется самостоятельная
разработка формата \Sys{.rpm} и пакетного менеджера APT. Набор утилит APT в ALT отличается
от аналогичной по названию программы в Debian, также как и RPM отличается от
аналогичного пакетного менеджера в RedHat.

\section{APT: инструменты управления пакетами}
\Sys{APT} --- часть системы управления пакетами в дистибутивах <<Альт>>. Advanced Packaging Tool
(усовершенствованный инструмент работы с пакетами) это набор утилит, позволяющий управлять пакетами.
\Sys{APT}  поддерживает загрузку пакетов из хранилища (репозитория).

\Emph{Хранилище(репозиторий)} ---  в общем виде хранилище данных.

В операционной системе \Sys{<<Альт>>} пакетный менеджер работает с репозиторием rpm-пакетов.

\Emph{Репозиторий пакетов} --- это замкнутая совокупность компонентов системы с
поддерживаемой целостностью и метаинформацией о них, то есть структурированные
компоненты с формализованными инструкциями по установке и разрешенными зависимостями.

Хранилище состоит из двух частей --- индексы (списки пакетов со служебной информацией) и
хранилище (структурированные файлы пакетов).
\Sys{APT} в зависимости от настроек может использовать удалённый репозиторий
с помощью сетевого протокола (например, \Sys{ftp}) или локальный репозиторий (например,
на оптическом диске).
Список источников пакетов хранится в файле
\Sys{/etc/apt/sources.list} и в каталоге \Sys{/etc/apt/sources.list.d/}. В системе \Sys{<<Альт>>}
применяется графическая оболочка для \Sys{apt} --- программа \Sys{Synaptic}\footnote{apt и synaptic
	развиваются ALT Linux Team, не нужно сравнивать реализации с аналогичными утилитами в \Sys{Debian}}.
Утилита \Sys{apt-get} значительно упрощает процесс установки программ в командном режиме.

\marginalia{ex_sign_col}{Для сокращения команд, встречающихся в тексте,  используется нотация:
	\begin{itemize}
		\item[-] команды \textbf{без административных привилегий} начинаются с
		символа <<\$>>;
		\item[-] команды \textbf{с административными привилегиями} начинаются с символа <<\#>>.
\end{itemize}}

Команда \Sys{\$ apt-get} выведет описание и возможности утилиты \Sys{apt-get}:
\begin{verbatim}
    $ apt-get
   apt 0.5.15lorg2 для linux x86_64 собран Jul 26 2023 18:10:41
   Использование: apt-get [параметры] команда
      apt-get [параметры] install|remove пакет1 [пакет2 ...]
      apt-get [параметры] source пакет1 [пакет2 ...]
\end{verbatim}

apt-get предоставляет простой командный интерфейс для получения и
установки пакетов. Чаще других используются команды update (обновить)
и install (установить).
	
Команды:
\begin{itemize}
	\item \Sys{update} --- получить обновлённые списки пакетов;
	\item \Sys{upgrade} --- произвести обновление;
	\item \Sys{install} --- установить новые пакеты;
	\item \Sys{remove} --- удалить пакеты;
	\item \Sys{source} --- скачать архивы исходников;
	\item \Sys{build-dep} --- установить всё необходимое для сборки исходных пакетов;
	\item \Sys{dist-upgrade} --- обновление системы в целом;
	\item \Sys{clean} --- удалить скачанные ранее архивные файлы;
	\item \Sys{autoclean} --- удалить давно скачанные архивные файлы;
	\item \Sys{check} --- удостовериться в отсутствии неудовлетворённых зависимостей;
	\item \Sys{dedup} --- удаление неразрешенных дупликатов пакетов.
\end{itemize}
	
Параметры:
\begin{itemize}
	\item \Sys{-h} --- краткая справка;
	\item \Sys{-q} --- скрыть индикатор процесса;
	\item \Sys{-qq}--- не показывать ничего кроме сообщений об ошибках;
	\item \Sys{-d} --- получить пакеты и выйти БЕЗ их установки или распаковки;
	\item \Sys{-s} --- симулировать упорядочение вместо реального исполнения;
	\item \Sys{-y} --- автоматически отвечать <<ДА>> на все вопросы;
	\item \Sys{-f} --- пытаться исправить положение, если найдены неудовлетворённые зависимости;
	\item \Sys{-m} --- пытаться продолжить, если часть архивов недоступна;
	\item \Sys{-u} --- показать список обновляемых пакетов;
	\item \Sys{-b} --- собрать пакет после получения его исходника;
	\item \Sys{-D} --- при удалении пакета стремиться удалить компоненты, от которых он зависит;
	\item \Sys{-V} --- подробно показывать номера версий;
	\item \Sys{-c=?} --- использовать указанный файл конфигурации;
	\item \Sys{-o=?} --- изменить любой из параметров настройки (например: \Sys{-o dir::cache=/tmp}).
\end{itemize}

Более полное описание доступно на страницах руководства man:
\Sys{apt-get, sources.list} и \Sys{apt.conf}:
\begin{itemize}
	\item\begin{verbatim}$ man apt-get\end{verbatim}
	\item\begin{verbatim}$ man sources.list\end{verbatim}
	\item\begin{verbatim}$ man apt.conf\end{verbatim}
\end{itemize}

В ОС \Sys{<<Альт>>} утилита \Sys{apt-get} использует основной пакетный менеджер \Sys{RPM Package Manager} ---
\Sys{RPM} для установки, обновления, удаления пакетов, управления зависимостями. Обе
утилиты \Sys{rpm} и \Sys{apt-get} позволяют установить, обновить или удалить пакет.

Отличия \Sys{rpm} и \Sys{apt-get}:
\begin{itemize}
	\item \Sys{apt-get} учитывает зависимости устанавливаемого пакета;
	\item \Sys{apt-get} умеет работать с репозиторием в целом:
	\begin{itemize}
		\item искать пакеты;
		\item вычислять список обновлений --- находить разницу версий пакетов,
		установленных локально и хранящихся в репозитории;
	\end{itemize}
	\item \Sys{apt-get} получает информацию из пакетов, используя \Sys{rpm}.
\end{itemize}

Утилита \Sys{rpm} подразумевает работу с конкретными пакетами. Пользователь самостоятельно
принимает решения, связанные с зависимостями пакетов  при работе с \Sys{RPM}. Утилита \Sys{apt-get}
вычисляет и устанавливает необходимые пакеты из репозитория, чтобы удовлетворить зависимости для
каждого \Sys{rpm}-пакета. Утилита \Sys{apt-get} самостоятельно не устанавливает пакеты, а использует для этого \Sys{RPM}.

\marginalia{ex_sign_col}{Установка пакетов в <<Альт Платформа>> осуществляется с помощью утилиты \Sys{apt}}

Целостность компонентов репозитория пакетов обеспечивает инфраструктура разработки операционной системы <<Альт>>.
Результат работы инфраструктуры это репозиторий пакетов. Каждый пакет репозитория формируется на
основе исходных данных пакета. Множество таких исходных данных для каждого пакета составляют git-репозиторий.

\Emph{Git-репозиторий} --- хранилище исходных данных с сохранением истории изменений каждого файла
хранилища. В данном контексте мы подразумеваем множество репозиториев исходных данных компонентов
системы (будь то ядро операционной системы, служебная библиотека, текстовый редактор, сервер для обслуживания
электронных сообщений или набор изображения для офромления графичекской среды), входящих в операционную
систему \Sys{ALT}.

Инфраструктура разработки \Sys{ALT} на основе подготовленных для сборки в пакеты исходных данных компонентов
системы выполняет типовые операции:
\begin{itemize}
	\item собирает компонент в соответствии с подготовленными инструкциями;
	\item проверяет целостность каждого компонента;
	\item проверяет зависимости и целостность связанных компонентов;
	\item добавляет пакет в репозиторий пакетов, если все условия выполнены.
\end{itemize}

Поддерживаемые в \Sys{ALT} репозитории пакеты называются <<Альт Платформа>> и обладают уникальным идентификатором
репозитория. В декабре 2023 года сформирован и поддерживается репозиторий \Sys{p10} под названием <<Альт~Платформа~10>>.

Репозиторий пакетов и утилиты \Sys{APT} вместе автоматизируют процессы управления установкой, обновления и удаления
программного обеспечения, исключают риск случайного повреждения целостности операционной системы и прикладных
программ.

Процесс взаимодействия пользователя с \Sys{APT}:
\begin{itemize}
	\item средствами \Sys{APT} по запросу пользователя загружаются метаданные из репозитория;
	\item \Sys{APT} получает от пользователя информацию о том, какие именно пакеты обновить или установить;
	\item \Sys{APT} проверяет зависимости и возможные конфликты компонентов ;
	\item \Sys{APT} предлагает пути решения, например, загрузку новых пакетов из репозитория, установку дополнительных или обновление имеющихся пакетов.
\end{itemize}

\marginalia{ex_sign_col}{Для обновления практически всего программного обеспечения (за исключением ядра
	операционной системы) на локальном компьютере до новой версии необходимо выполнить команды:\\
	\Sys{\#apt-get update}\\
	\Sys{\#apt-get dist-upgrade}
}

При использовании \Sys{APT} и обновляемого стабильного репозитория операционная система может
служить на компьютере годами, гарантировано обновляясь до новых версий.

\section{Вопросы для самопроверки}

\begin{enumerate}
	\item Что такое пакет?
	\item Какие форматы пакетов вы знаете?
	\item Что такое зависимость пакета?
	\item Что такое репозиторий пакетов?
	\item Какие низкоуровневые пакетные менеджеры вы знаете?
	\item Какой пакетный менеджер используется в <<Альт Платформа>>?
	\item Вы обнаружили в системе пакет \Sys{nagios-domain-discovery-0.1.1-alt1.noarch.rpm}, определите его имя, версию, релиз и архитектуру.
	\item Верно ли утверждение, что утилита \Sys{apt-get} при установке пакетов требует явного указания всех зависимых пакетов?
	\item Для чего используется команда \Sys{apt-get update}?
	\item Выполнится ли успешно команда \verb!$ apt-get dist-upgrade!?
\end{enumerate}

\chapter{Основные команды пакетного менеджера}\label{basic-package-manager-commands}
\section{Просмотр недавно установленных пакетов}\label{view-recently-installed-packages}
Чтобы получить список последних установленных пакетов, введите следующую команду: 

\begin{verbatim}
rpm -qa --last|head	
\end{verbatim}

Вывод:
\begin{verbatim}
smplayer-23.6.0-alt2.10169.x86_64             Пт 01 дек 2023 12:45:39
qt5-wayland-5.15.10-alt1.x86_64               Пт 01 дек 2023 12:35:44
qt5-tools-5.15.10-alt2.x86_64                 Пт 01 дек 2023 12:35:44
qt5-sql-mysql-5.15.10-alt1.x86_64             Пт 01 дек 2023 12:35:44
qt5-dbus-5.15.10-alt2.x86_64                  Пт 01 дек 2023 12:35:44
libqt5-xmlpatterns-5.15.10-alt1.x86_64        Пт 01 дек 2023 12:35:44
libqt5-x11extras-5.15.10-alt1.x86_64          Пт 01 дек 2023 12:35:44
libqt5-webenginewidgets-5.15.15-alt1.x86_64   Пт 01 дек 2023 12:35:44
libqt5-test-5.15.10-alt1.x86_64               Пт 01 дек 2023 12:35:44
libqt5-quickparticles-5.15.10-alt1.x86_64     Пт 01 дек 2023 12:35:44
\end{verbatim}

Команда \Sys{rpm -qa --last} используется для вывода списка всех установленных пакетов, отсортированных 
по времени их установки. Пакеты будут отсортированы в порядке убывания времени установки --- самые 
последние установленные пакеты отобразятся в верхней части списка.

Фильтрация вывода: утилита \Sys{grep} отфильтрует вывод и поможет найти искомый пакет. Например, 
следующая команда выведет информацию только о тех пакетах, название которых содержит <<\Sys{kernel}>>:

\begin{verbatim}
rpm -qa --last | grep kernel
\end{verbatim} 

\chapter{Программное обеспечение используемое для упаковки}\label{software-used-for-packaging}

В разделе \hyperlink{1.3}{1.3 <<Установка необходимых пакетов для процесса сборки>>} 
упоминался перечень пакетов для работы со сборкой:
\begin{verbatim}
	gcc, rpm-build, rpmlint, make, python, git, gear, hasher, patch, rpmdevtools.
\end{verbatim}

Опишем основные инструменты для управления пакетами и их сборки. Технологическую 
базу репозитория Sisyphus составляют адаптированные к нуждам команды разработчиков 
программы и специально разработанные решения%
\footnote{\href{https://wiki.altlinux.ru/QuickStart/PkgManagment\#\%D0\%9E\%D1\%81\%D0\%BD\%D0\%BE\%D0\%B2\%D0\%BD\%D1\%8B\%D0\%B5_\%D0\%B8\%D0\%BD\%D1\%81\%D1\%82\%D1\%80\%D1\%83\%D0\%BC\%D0\%B5\%D0\%BD\%D1\%82\%D1\%8B_\%D0\%B4\%D0\%BB\%D1\%8F_\%D1\%83\%D0\%BF\%D1\%80\%D0\%B0\%D0\%B2\%D0\%BB\%D0\%B5\%D0\%BD\%D0\%B8\%D1\%8F_\%D0\%BF\%D0\%B0\%D0\%BA\%D0\%B5\%D1\%82\%D0\%B0\%D0\%BC\%D0\%B8}{https://wiki.altlinux.ru/QuickStart/PkgManagment}}. 
К ним можно отнести\footnote{\href{https://www.altlinux.org/\%D0\%A0\%D0\%B5\%D0\%BF\%D0\%BE\%D0\%B7\%D0\%B8\%D1\%82\%D0\%BE\%D1\%80\%D0\%B8\%D0\%B9_\%D0\%A1\%D0\%9F\%D0\%9E\#APT_\%D0\%B8_\%D1\%80\%D0\%B5\%D0\%BF\%D0\%BE\%D0\%B7\%D0\%B8\%D1\%82\%D0\%BE\%D1\%80\%D0\%B8\%D0\%B9_\%D0\%BF\%D0\%B0\%D0\%BA\%D0\%B5\%D1\%82\%D0\%BE\%D0\%B2}{https://www.altlinux.org/Репозиторий\_СПО\#APT\_и\_репозиторий\_пакетов}.}:
\begin{itemize}
	\item \Emph{RPM} (менеджер пакетов). Используется для просмотра, сборки, установки, инспекции, 
		проверки, обновления и удаления отдельных программных пакетов. Каждый такой пакет состоит 
		из набора файлов и информации о пакете, включающей название, версию, описание пакета и~т.\,д. 
		Отличительными особенностями \Sys{RPM} в Sisyphus являются: удобное поведение <<по умолчанию>> 
		для уменьшения количества шаблонного кода в \Sys{.spec}-файлах, обширный модульный набор 
		макросов для упаковки различных типов пакетов, развитые механизмы автоматического вычисления 
		межпакетных зависимостей при сборке пакетов, поддержка \Sys{set}-версий в зависимостях на 
		разделяемые библиотеки, автоматическое создание пакетов с отладочной информацией с поддержкой 
		зависимостей между такими пакетами. В контексте данной темы \Sys{RPM} рассматривается не только 
		как менеджер пакетов, но и как набор инструментов для их сборки (\Sys{rpmbuild}). 
	\item \Emph{APT} (усовершенствованная система управления программными пакетами). Умеет автоматически 
		разрешать зависимости при установке, обеспечивает установку из нескольких источников и целый 
		ряд других уникальных возможностей, включая получение последней версии списка пакетов из 
		репозитория и обновление системы.
	\item \Emph{Hasher} --- инструмент для безопасной, воспроизводимой и высокопроизводительной сборки 
		\Sys{RPM}-пакетов в контролируемой среде.
	\item \Emph{Gear} --- набор инструментов для поддержки совместной разработки \Sys{RPM}-пакетов в 
		системе контроля версий \Sys{git}. \Sys{Gear} интегрирован с \Sys{hasher} и \Sys{rpmbuild}. 
		\Sys{Gear} подготавливает \Sys{SRPM} из \Sys{git} репозитория, распаковывает его и запускает 
		\Sys{hasher} или \Sys{rpmbuild}.
\end{itemize}


\section{Описание RPM-пакета}
\Emph{RPM-пакет} --- это специальный архив с файлами. Сам файл пакета состоит из четырёх секций --- 
начального идентификатора, сигнатуры, бинарного заголовка и \Sys{cpio}-архива с файлами проекта и деревом 
каталога\footnote{\href{https://www.opennet.ru/docs/RUS/rpm_guide/13.html}{https://www.opennet.ru/docs/RUS/rpm\_guide/13.html}}.

\Sys{RPM}-пакеты делятся на несколько категорий --- пакеты с исходным кодом, бинарные пакеты и платформо-независимые 
бинарные пакеты.
\begin{itemize}
	\item \Emph{RPM-пакет} (бинарные) --- это архив с расширением \Sys{.rpm}. Такой пакет содержит 
		скомпилированные под определённую процессорную архитектуру исполняемые файлы и библиотеки. 
		На системах с разной процессорной архитектурой не получится использовать один и тот же 
		скомпилированный бинарный \Sys{rpm}-пакет.
	\item \Emph{\Sys{noarch}-пакет} --- платформо-независимый бинарный пакет.
	\item \Emph{SRPM-пакет} (source RPM, пакет с исходным кодом) --- это архив с расширением \Sys{.src.rpm}. 
		\Sys{SRPM} содержит исходный код, патчи, если необходимо, и \Sys{SPEC}-файл. Эти пакеты содержат 
		всю информацию для сборки пакета.
\end{itemize}

По инструкциям из \Sys{SPEC} файла собирается бинарный \Sys{RPM}-пакет%
\footnote{\href{https://uneex.ru/static/RedHatRPMGuideBook/rpm_guide-linux.html\#16_html}{https://uneex.ru/static/RedHatRPMGuideBook/rpm\_guide-linux.html\#16\_html}}. 
На основе бинарных пакетов строится база данных в \Sys{/var/lib/rpm}. Вся информация о пакетах хранится в базе 
данных \Sys{Packages}. Инструкции содержат также информацию о правах доступа и их применении в процессе установки, 
скрипты, запускаемые при установке или удалении пакета.

Принято называть пакеты \Sys{RPM} по типу:
\begin{verbatim}
	имя-версия-релиз.процессорная_архитектура.rpm
\end{verbatim}

При этом в системе имя установленного пакета будет отличаться: в командной строке можно обратиться за информацией 
по пакету, указывая только его имя, если установлена одна версия пакета, и указывая \Sys{имя-версию-релиз}, если 
установлено больше версий\footnote{\href{https://www.opennet.ru/docs/RUS/rpm_guide/13.html}{https://www.opennet.ru/docs/RUS/rpm\_guide/13.html}}.


\section{Инструменты для сборки RPM-пакетов}
\Emph{Инструменты сборки \Sys{rpm}-пакета} --- это пакеты и программы, с помощью которых из набора исходных файлов можно получить специальный архив в формате \Sys{rpm}-пакета. Приведём список этих инструментов\footnote{\href{https://www.altlinux.org/\%D0\%A1\%D0\%B1\%D0\%BE\%D1\%80\%D0\%BA\%D0\%B0_\%D0\%BF\%D0\%B0\%D0\%BA\%D0\%B5\%D1\%82\%D0\%B0_\%D1\%81_\%D0\%BD\%D1\%83\%D0\%BB\%D1\%8F}{https://www.altlinux.org/Сборка\_пакета\_с\_нуля}}: \Sys{rpmdevtools}, \Sys{rpmdev-setuptree}, \Sys{rpmbuild}, \Sys{rpmspec}, \Sys{rpmlint}. 

\begin{itemize}
	\item \textbf{rpmdevtools} --- пакет с набором программ для сборки пакетов; 
	
	\begin{itemize}
		\item \textbf{rpmdev-setuptree} --- утилита для создания структуры рабочих каталогов;
		\item \textbf{rpmdev-newspec} --- утилита для создания \Sys{spec}-файла;
	\end{itemize}
	
	\item \textbf{rpmspec} --- утилита работы с файлами спецификации --- текстовыми файлами с расширением \Sys{.spec}. Служит для проверки подготовленного \Sys{spec}-файла;
	
	\item \textbf{rpmbuild} --- утилита сборки \Sys{rpm}-пакета из набора подготовленных файлов;
	
	\item \textbf{rpmlint} --- утилита для тестирования собранного \Sys{rpm}-пакета;
	
	\item \textbf{rpm-utils} --- пакет с набором программ для работы с \Sys{rpm}-пакетами.
\end{itemize}


\section{Описание SPEC-файла}
\Emph{SPEC-файл} --- RPM Specification File --- это текстовый файл, который описывает процесс сборки и конфигурацию 
пакета, служит инструкцией для утилиты \Sys{rpmbuild}. Он содержит метаданные, такие как имя пакета, версию, 
лицензию, а также разделы с инструкциями для сборки, установки и упаковки программного обеспечения, журнал 
изменений пакета%
\footnote{\href{https://wiki.mageia.org.ru/index.php?title=\%D0\%A1\%D1\%82\%D1\%80\%D1\%83\%D0\%BA\%D1\%82\%D1\%83\%D1\%80\%D0\%B0_RPM_SPEC-\%D1\%84\%D0\%B0\%D0\%B9\%D0\%BB\%D0\%B0}%
{https://wiki.mageia.org.ru/index.php?title=Структура\_RPM\_SPEC-файла}}. %
\Sys{SPEC}-файл можно рассматривать как <<инструкцию>>, которую утилита \Sys{rpmbuild} использует для сборки \Sys{RPM}-пакета.

\Sys{Spec}-файл состоит из трёх разделов: Header (Заголовок/Преамбула), Body (Тело) и Сhangelog(Журнал изменений).
\begin{enumerate}
	\item \Emph{Header} (Заголовок) --- этот раздел содержит метаданные о пакете, такие как его имя (Name), 
		версия (Version), релиз (Release), краткое описание (Summary), лицензия (License) и другие 
		параметры, которые идентифицируют и характеризуют пакет.
	\item \Emph{Body} (Тело) --- этот раздел содержит инструкции для процесса сборки пакета. В нём 
		определяются различные секции, такие как \Sys{BuildRequires} (зависимости для сборки), 
		\Sys{\%build} (инструкции для сборки), \Sys{\%install} (инструкции для установки), 
		\Sys{\%files} (список файлов, включённых в пакет), и другие.	
	\item \Emph{Changelog} (Журнал изменений) --- этот раздел содержит историю изменений пакета. 
		Он включает записи о внесённых изменениях, включая дату изменения, автора и краткое 
		описание того, что было изменено.
\end{enumerate}

\paragraph{Преамбула (Заголовок)}
Заголовок \Sys{SPEC}-файла содержит информацию о пакете: версию, исходный код, патчи, зависимости.

Рекомендуемый порядок заголовочных тэгов:
\begin{itemize}
	\item Name, Version, Release, Serial
	\item далее Summary, License, Group, Url, Packager, BuildArch
	\item потом Source*, Patch*
	\item далее PreReqs, Requires, Provides, Conflicts
	\item и, наконец, Prefix, BuildPreReqs, BuildRequires.
\end{itemize}

Ниже приведён пример части \Sys{SPEC}-файла \Sys{notepadqq}:
\begin{verbatim}
	Summary:	A Linux clone of Notepad++
	Name:		notepadqq
	Version:	1.4.8
	Release:	alt2
	License:	GPLv3
	Group:		Editors
	URL:		http://notepadqq.altervista.org/wp/
	Source0:	%name-%version.tar
	Source1:	codemirror.tar
\end{verbatim}

\hypertarget{4}{\chapter{Инструмент Gear}}\label{chapter-gear}
\Emph{GEAR (Get Every Archive from git package Repository)} --- 
инструмент для поддержки и сборки пакетов из \Sys{git}-репозиториев%
\footnote{\href{https://www.altlinux.org/Gear}{https://www.altlinux.org/Gear}}%
\footnote{\href{https://www.altlinux.org/Gear/\%D0\%A1\%D0\%BF\%D1\%80\%D0\%B0\%D0\%B2\%D0\%BE\%D1\%87\%D0\%BD\%D0\%B8\%D0\%BA}{https://www.altlinux.org/Gear/Справочник}}.

\marginalia{ex_sign_col}{Для работы с текущим разделом необходимо изучить 
документацию по работе с распределённой системой управления версиями \Sys{Git}.}

\Emph{Git} --- это технология контроля версий, которая следит за изменениями в файлах. 
\Sys{Git}-репозиторий хранит исходный код и данные для сборки пакетов, включая историю 
изменений, сведения о версиях, авторах изменений и прочую информацию. 

При сборке пакетов \Sys{gear} предлагает работать в том же \Sys{git}-репозитории, 
в котором хранятся исходные тексты пакета. \Sys{Gear} позволяет целиком импортировать 
историю разработки, предоставляет различные средства импорта исходных текстов и 
различные варианты организации репозитория. 

Идея \Sys{gear} в доступности всего необходимого для сборки пакета в \Sys{git}-репозитории, 
либо в репозитории пакетов, на основе которого ведётся сборка. Система контроля версий 
\Sys{Git} предоставляет встроенные механизмы обеспечения целостности (контрольные суммы, 
криптографически подписанные теги) для задач управления пакетами. Появляется возможность 
воспроизводимой сборки --- собрать <<такой же>> пакет ещё раз, опираясь на логически 
законченную версию изменений зафиксированную на момент времени --- коммит (commit) в 
терминологии git.

Сборка пакета ведётся из конкретного коммита, который мы в дальнейшем будем называть главным. 
Именно в нём должна находится нужная версия \Sys{SPEC}-файла и правил экспорта. \Sys{GEAR} 
позволяет экспортировать из \Sys{git}-репозитория каталоги и подкаталоги в виде архивов; 
экспортировать отдельные файлы; вычислять разницу (\Sys{diff}) и сохранять её в виде патча. 
При этом может использоваться как состояние в главном коммите, так и в любом из его предков, 
прямых и не прямых. Это позволяет гибко и удобно организовать работу по поддержке пакета: 
все нужные исходные тексты можно хранить в \Sys{git} так, чтобы с ними было удобно работать, 
а затем экспортировать так, чтобы ими было удобно воспользоваться из \Sys{SPEC}-файла.

Правила экспорта для \Sys{gear} описываются в текстовом файле, который также хранится в 
репозитории. По умолчанию \Sys{gear} ищет этот файл по пути \Sys{.gear/rules} или \Sys{.gear-rules}. 
Подробнее про синтаксис этого файла можно прочитать ниже, в разделе <<Правила экспорта>>.

Для удобства сборки пакетов \Sys{gear} интегрирован с инструментами \Sys{rpm-build} и 
\Sys{hasher}: из \Sys{gear}-репозитория можно одной командой собрать пакет при помощи 
\Sys{rpmbuild} или \Sys{hsh}. Подробнее об этом можно прочитать далее в разделе <<Быстрый старт>>. 


\section{Структура репозитория}
Обобщим опыт, упрощающий работу с \Sys{git}-репозиториями, предназначенными для хранения исходного кода пакетов. 

\begin{itemize}
	\item \Emph{Одна кодовая база --- один репозиторий.}
	
	Не имеет смысла хранить в одном репозитории исходные тексты, не связанных между собой пакетов. 
		В этом правиле бывают исключения в случае, если культура разработки предполагает хранение 
		разных составных частей проекта в одном репозитории.
	\item  \Emph{Храните в \Sys{git}-репозитории распакованный исходный код.}
	
	Исходные тексты, поставляемые в виде архивов (\Sys{tar}, \Sys{zip}) и сжатые различными 
		компрессорами (\Sys{gz}, \Sys{bzip2}, \Sys{xz}) удобнее распаковать. Это упрощает 
		использование средств \Sys{git} для работы с этими данными: так легче отслеживать изменения, 
		ниже трафик при обновлениях и~т.\,д.
	\item \Emph{Отделяйте свои изменения от изменений исходной кодовой базы.}
	
	Если вы не являетесь разработчиком пакета, который собираете, лучше хранить свои изменения отдельно 
		от кода разработчиков, например в отдельной ветке (и формировать \Sys{diff} средствами \Sys{GEAR}) 
		или в виде патчей. Это заметно упрощает совместную работу над пакетом, обновления и аудит.
\end{itemize}


\section{Правила экспорта}
Правила экспорта для \Sys{GEAR} описываются в текстовом файле, который также хранится в репозитории. 
По умолчанию \Sys{GEAR} ищет этот файл по пути \Sys{.gear/rules} или \Sys{.gear-rules}, от корня репозитория, 
в главном коммите. 

Этот файл состоит из одной или нескольких строк, каждая из которых имеет следующий формат: \verb!<директива>: <параметры>!

Параметры разделяются пробельными символами. Пробелы между директивой и <<:>> и <<:>> и параметрами 
не обязательны. 

Пустые строки и строки, начинающиеся с <<\#>>, игнорируются. 

В значениях многих параметров и опций директив могут применяться ключевые слова: 
\begin{itemize}
	\item \Sys{@name@} --- будет заменено на имя пакета (извлекается из \Sys{SPEC}-файла);
	\item \Sys{@version@} --- будет заменено на версию пакета (извлекается из \Sys{SPEC}-файла);
	\item \Sys{@release@} --- будет заменено на релиз пакета (извлекается из \Sys{SPEC}-файла).
\end{itemize}

\Emph{Теги и пути}

По умолчанию все пути в аргументах директив считаются от корня репозитория главного коммита. 
Однако большинство директив позволяет указать другой коммит в качестве основы. Для этого путь 
должен быть передан в формате: 
\begin{verbatim}
	base_tree:path_to_file.
\end{verbatim}

В качестве \Sys{base\_tree} может выступать: 
\begin{itemize}
	\item полный идентификатор коммита (\Sys{SHA-1}, 40 шестнадцатиричных цифр);
	\item имя тега \Sys{GEAR};
	\item символ <<.>>, обозначающий главный коммит.
\end{itemize}

В любом случае, коммит, на который так ссылаются, должен быть предком главного коммита. 

\Emph{Основные директивы}

Ниже приведены основные директивы \Sys{GEAR}, и их аргументы. Подробнее с ними можно 
ознакомиться в \Sys{man gear-rules(5)}.

\begin{itemize}
	\item \Emph{\Sys{spec: <путь> }}
	
	Задаёт путь к \Sys{SPEC}-файлу. По умолчанию, \Sys{GEAR} использует файл с 
		расширением \Sys{.spec} из корня репозитория в главном коммите, 
		если такой файл там только один. Единственный аргумент --- путь к \Sys{SPEC}-файлу.
	
	\item \Emph{\Sys{copy: <glob>...}}
	Скопировать файл, соответствующий указанному шаблону поиска (glob pattern). 
		Может принимать несколько аргументов, для каждого из которых должны 
		быть найдены соответствующие файлы.
	
	Также существуют директивы \Sys{gzip}, \Sys{bzip2}, \Sys{lzma}, \Sys{lzma}, \Sys{zstd}, 
		аналогичные copy, но сжимающие экспортированный файл подходящим алгоритмом сжатия.
	\item  \Emph{\Sys{tar: <tree\_path>}}

	Экспортировать каталог из репозитория в виде \Sys{tar}-архива. Допустимые опции: 
	\begin{itemize}
		\item \Sys{name=<archive\_name>} имя архива (без суффикса \Sys{.tar});
		\item \Sys{base=<base\_name>} внутри архива будет создан каталог с указанным именем, 
			и все файлы будут помещены в него;
		\item \Sys{suffix=<suffix>} расширение создаваемого архива (по умолчанию --- \Sys{.tar});
		\item \Sys{exclude=<glob\_patter>} не включать в архив файлы, соответствующие указанному шаблону поиска.
	\end{itemize}
	
	Помимо стандартных ключевых слов, в опциях \Sys{name} и \Sys{base} может применяться ключевое 
		слово \Sys{@dir@}, которое будет заменено на имя каталога из параметра \Sys{tree\_path}. 
	\item \Emph{\Sys{zip: <tree\_path>}}
	
	Экспортировать каталог из репозитория в виде \Sys{zip}-архива. Принимает те же аргументы, 
		что и директива \Sys{tar}, использует \Sys{.zip} в качестве расширения по умолчанию. 
	
	Также существуют директивы \Sys{tar.gz, tar.bz2, tar.lzma, tar.xz, tar.zst}, аналогичные 
		\Sys{tar}, но сжимающие созданный архив подходящим алгоритмом сжатия. Чаще всего используются 
		несжатые архивы, так как сжатия, используемого при сборке \Sys{SRPM}, обычно достаточно.
	\item  \Emph{\Sys{diff: <old\_tree\_path> <new\_tree\_path>}}
	
	Создать \Sys{unified diff} между указанными каталогами и сохранить его в виде патча. Допустимые опции:
	\begin{itemize}
		\item \Sys{name=<diff\_name>} имя создаваемого файла;
		\item \Sys{exclude=<glob\_patter>} игнорировать файлы, соответствующие указанному шаблону поиска.
	\end{itemize}
	
	Помимо стандартных ключевых слов, в опции \Sys{name} могут применяться ключевые слова \Sys{@old\_dir@} 
		и \Sys{@new\_dir@}, которые будут заменены на имя каталога из параметра \Sys{old\_tree\_path} 
		и \Sys{new\_tree\_path} соответственно. 
\end{itemize}

Все приведённые директивы требуют, чтобы все указанные в них файлы и каталоги существовали; если какого-то 
файла или каталога не будет существовать, экспорт завершится ошибкой. Однако существуют аналогичные им директивы, 
заканчивающиеся знаком вопроса (например, \Sys{tar?:} или \Sys{copy?:}), которые игнорируют отсутствующие файлы. 
Это может быть удобно, чтобы не приходилось менять правила экспорта при добавлении и удалении патчей.


\section{Основные типы устройства \Sys{gear}-репозитория}
Гибкость \Sys{GEAR}\footnote{\href{https://www.altlinux.org/\%D0\%A0\%D1\%83\%D0\%BA\%D0\%BE\%D0\%B2\%D0\%BE\%D0\%B4\%D1\%81\%D1\%82\%D0\%B2\%D0\%BE_\%D0\%BF\%D0\%BE_gear}{https://www.altlinux.org/Руководство\_по\_gear}} 
означает, что каждый пользователь может настроить его по своему усмотрению. Существует несколько 
распространённых способов организации \Sys{gear}-репозитория в качестве основы для более сложных конфигураций. 
Знакомство с ними поможет понять организацию репозитория при совместной работе над пакетами. 

Базовые виды ведения \Sys{GEAR} репозиториев:
\begin{itemize}
	\item \Emph{Архив с исходными текстами и патчи.}
	
	Исходные тексты хранятся в каталоге \Sys{package\_name}; дополнительные изменения хранятся в виде патчей. 
		Подобные репозитории создаёт команда \Sys{gear-srpmimport}, и также выглядят импортированные 
		пакеты в \Sys{git.altlinux.org/srpms}.
	
	В каталоге \Sys{package\_name} принято хранить не изменённые исходники; для их обновления удобно 
		использовать команду \Sys{gear-update}.
	
	Такой формат будет больше всего знаком пользователям без опыта поддержки пакетов \Sys{source RPM}, 
		и при работе с проектами, не имеющими публичного \Sys{git}-репозитория или не использующими \Sys{git}.
	
	Пример \Sys{.gear/rules}:
	\begin{verbatim}
		tar: package_name
		copy?: *.patch
	\end{verbatim} 
	
	\item \Emph{Репозиторий с историей исходного репозитория и модифицированными исходными текстами.}
	
	Создаётся копия исходного \Sys{git}-репозитория. Находится коммит, из которого нужно взять исходные 
		тексты для сборки. На основе этого коммита добавляется каталог \Sys{.gear} и \Sys{SPEC}-файл. 
		При обновлении новые исходные тексты сливаются (\Sys{merge}) в эту ветку. Создаётся \Sys{gear}-тег, 
		соответствующий собираемой версии. Изменения могут храниться в виде патчей. Но чаще вносятся в 
		текущую ветку со \Sys{spec}-файлом, затем изменения экспортируются в виде одного большого патча. 
	
	Пример \Sys{.gear/rules} с генерацией патча:
	\begin{verbatim}
		tar: v@version@:.
		diff: v@version@:. . exclude=.gear/** exclude=*.sp
	\end{verbatim}
	
	\item \Emph{Пустая ветка со \Sys{SPEC}-файлом и отдельная история разработки.}
	
	Изначально \Sys{SPEC}-файл и \Sys{.gear} ведутся в отдельной ветке, содержащей главный коммит. 
		Исходные тексты, необходимые для сборки пакета, сливаются (\Sys{merge}) при необходимости 
		с \Sys{git} стратегией \Sys{ours} --- таким образом, нужные коммиты оказываются в истории 
		главного, но сами исходные тексты в ветку не попадают. Если нужно внести какие-то специфичные 
		изменения, они вносятся в отдельную ветку, например \Sys{alt-fixes}. При каждом обновлении 
		кода ветка \Sys{alt-fixes} и соответствующий ей \Sys{GEAR}-тег должны обновляться. В нашем 
		примере \Sys{GEAR}-тег совпадает с именем ветки. 
	
	Пример \Sys{.gear/rules}:
	\begin{verbatim}
		tar: v@version@:.
		diff: v@version@:. alt-fixes:.
	\end{verbatim} 
\end{itemize}


\section{Быстрый старт Gear}
\Emph{Установка gear}

В разделе \hyperlink{1.3}{1.3 <<Установка необходимых пакетов для процесса сборки>>} 
также упоминается инструмент \Sys{gear}. Для отдельной установки используйте команду: 
\begin{verbatim}
	#apt-get install gear
\end{verbatim}

\Emph{Импорт .src.rpm }

Пакет в формате \Sys{source RPM} можно импортировать в \Sys{gear}-репозиторий командой \Sys{gear-srpmimport}:
\begin{verbatim}
	mkdir package_name
	cd package_name
	git init -b sisyphus
	gear-srpmimport /путь/к/package_name.src.rpm
\end{verbatim}

\Emph{Получение готового репозитория с внешнего git-сервера}

Достаточно клонировать репозиторий: 
\begin{verbatim}
	git clone <repository url> package_name
	cd package_name
\end{verbatim}

Никаких дополнительных настроек не требуется.

\Emph{Сборка пакета}

Основным инструментом экспорта и сборки пакетов является команда \Sys{gear}. На практике удобно 
использовать предоставляемые команды-обёртки, а к самой команде \Sys{gear} прибегать только в 
самых сложных случаях. Командной обёртке можно передавать опции как утилиты \Sys{gear}, так и 
вложенной утилиты.

Командная обертка \Sys{gear} для \Sys{rpmbuild} --- \Sys{gear-rpm}. Для сборки пакета используйте команду: 
\begin{verbatim}
	gear-rpm --verbose -ba
\end{verbatim}


Собрать пакет при помощи \Sys{hsh} (установка и настройка \Sys{hasher} описана в следующих главах): 
\begin{verbatim}
	gear-hsh --verbose
\end{verbatim}

Команде \Sys{gear-hsh} можно передать как аргументы \Sys{gear}, так и аргументы \Sys{hsh}.

Если не указана опция \Sys{--tree-ish}, \Sys{gear} в качестве главного коммита использует текущий 
коммит (HEAD). Если хочется проверить свежие изменения без создания коммита, можно использовать опцию 
\Sys{--commit}, доступную и для \Sys{gear}, и для \Sys{gear-rpm}, и для \Sys{gear-hsh}. В таком случае 
будет создан временный коммит (аналогично \Sys{git commit -a}), и пакет будет собран уже из него. 
Стоит отметить, что, аналогично \Sys{git commit -a}, в таком коммите не будет новых файлов, если они 
ещё не были добавлены в \Sys{git} командой \Sys{git add}; если они нужны для сборки, то их стоит добавить. 

\Emph{Сборка пакета из репозитория разработчиков}

Для примера возьмём конкретный пакет, который уже есть в репозитории Сизиф, например \Sys{pixz}, и попробуем
собрать его с нуля.

Для начала клонируем репозиторий. Сразу зададим имя удалённого репозитория: 
\begin{verbatim}
	git clone https://github.com/vasi/pixz -o upstream
	cd pixz
\end{verbatim}

Перейдём в ветку, из которой будем собирать пакет: 
\begin{verbatim}
	git checkout -b sisyphus upstream/master
\end{verbatim}

Переместимся на тег (тег --- это ссылка, указывающие на определённый \Sys{git}-коммит в репозитории), 
соответствующий версии, которую мы будем собирать. На момент написания пособия последний тег \Sys{v1.0.7}: 
\begin{verbatim}
	git reset --hard v1.0.7
\end{verbatim}

Определим правила экспорта: 
\begin{verbatim}
	mkdir .gear
	vim .gear/rules
\end{verbatim}

Правила зададим следующие: 
\begin{itemize}
	\item \Sys{SPEC}-файл перенесём в каталог \Sys{.gear}, чтобы он не путался с основными исходными текстами;
	\item исходники будем забирать из тега \Sys{GEAR}, соответствующего апстримному;
	\item сразу создадим патч, включающий все наши изменения, каталог \Sys{.gear} в этот патч мы включать не будем.
\end{itemize}

Получаем следующий файл \Sys{.gear/rules}:
\begin{verbatim}
	spec: .gear/pixz.spec
	tar: v@version@:.
	diff: v@version@:. . exclude=.gear/**
\end{verbatim}

Создадим соответствующий версии тег \Sys{GEAR}:
\begin{verbatim}
	gear-store-tags v1.0.7
\end{verbatim} 

Напишем \Sys{SPEC}-файл:
\begin{verbatim}
	vim .gear/pixz.spec
\end{verbatim} 

Важно, что версия в спеке --- \Sys{1.0.7}, поэтому конструкция \Sys{v@version@}, 
которую мы использовали в \Sys{.gear/rules}, раскроется в строку \Sys{v1.0.7} --- 
именно такой тег \Sys{GEAR} мы создали. 

Добавим каталог \Sys{.gear} в \Sys{git}:
\begin{verbatim}
	git add .gear
\end{verbatim} 

Можно попробовать собрать пакет: 
\begin{verbatim}
	gear-rpm --verbose --commit -ba
\end{verbatim}

При необходимости, можно поправить \Sys{spec}, добавить нужные зависимости. 
Когда пакет собирается и всем устраивает, можно зафиксировать все изменения: 
\begin{verbatim}
	gear-commit -a
\end{verbatim}

\section{Вопросы для самопроверки}

\begin{itemize}
	\item Что такое инструмент \Sys{git}?
	\item Что такое инструмент \Sys{gear}?
	\item Где хранятся правила экспорта для \Sys{gear}?
	\item Перечислете правила, упрощающие работу с \Sys{git}-репозиториями?
	\item Какой формат у файла с правилами экспорта для \Sys{gear}?
	\item Что означает параметр \Sys{@name@} в файле \Sys{.gear-rules}?
	\item Как формируется путь в аргументах директив в файле \Sys{.gear-rules}?
        \item Перечислите основные дерективы в файле \Sys{.gear-rules}?
	\item Какой командой собирается пакет с помощью \Sys{gear}?
        \item Как создать тег соответствующий версии пакета с помощью \Sys{gear}?
\end{itemize}

\hypertarget{5}{\chapter{Инструмент Hasher}}
\Emph{Hasher} --- инструмент для сборки пакетов с использованием изолированной среды (\Sys{chroot}).
Представляет собой сложный усовершенствованный инструмент, направленный на устранение недостатков,
связанных с процессом сборки пакетов в локальной среде средствами \Sys{rpm-build}.
\begin{itemize}
	\item Благодаря входящим в комплект утилитам упрощается процесс поддержания сборочных зависимостей и сохранение
	целостности пакета.
	\item  Изолированная среда обновляется для каждой сборки, инициированной средствами \Sys{Hasher},
	что гарантирует независимый от конфигурации операционной системы процесс сборки пакета.
	\item  Изменяя конфигурацию источников для \Sys{Hasher}, указывая различные
	репозитории, возможно собирать пакеты для пакетных баз отличных от локальной
	операционной системы.
\end{itemize}

Для начала работы необходимо установить пакет \Sys{hasher} и настроить работу с утилитой:
\begin{verbatim}
	# apt-get install hasher
\end{verbatim}

Возможности \Sys{Hasher} задокументированы%
\footnote{\href{http://uneex.ru/static/AltlinuxOrg_Hasher/}{http://uneex.ru/static/AltlinuxOrg\_Hasher/}}.
\begin{verbatim}
	$ man hsh
\end{verbatim}

\begin{verbatim}
	$ man hasher-priv
\end{verbatim}

\hypertarget{5.1}{\section{Настройка Hasher}}
Для начала работы необходимо настроить утилиту:
\begin{itemize}
	\item Запустить демон \Sys{hasher-priv} (необходимо начиная с версии hasher 2.0):
	\begin{verbatim}
		# systemctl enable ----now hasher-privd.service
	\end{verbatim}
	
	\item Подготовить учётную запись для работы с \Sys{hasher} можно средствами встроенной утилиты.
	Пользователь должен отличаться от корневого супер пользователя (\Sys{root}) и должен быть
	лишен повышенных привилегий для обеспечения безопасности системы от процесса сборки.
	Утилита создаст дополнительных пользователей добавит в необходимы группы и настроит
	\Sys{hasher} для работы от указанного пользователя:
	\begin{verbatim}
		# hasher-useradd <имя учетной записи>
	\end{verbatim}
	\marginalia{ex_sign_col}{Если утилитой \Sys{hasher-useradd} настраивается активный пользователь в системе, то
		поскольку утилита производит манипуляции с добавлением пользователя группы, пользователю
		необходимо создать новую или перезапустить рабочую сессию, чтобы изменения вступили в силу.}
	
	\item Создать каталоги расположения сборочной среды и каталоги настроек (следующие указанные
	пути используются утилитой  по умолчанию):
	\begin{verbatim}
		$ mkdir ~/hasher
		$ mkdir ~/.hasher
	\end{verbatim}
	
	Необходимо добавить имя в указанном формате в конфигурационный файл:
	\begin{verbatim}
		$ echo 'packager="Ваше_имя  Ваша_фамилия <my_mail@altlinux.org>"' >> ~/.hasher/config
	\end{verbatim}
	Этот параметр важен для встроенных проверок \Sys{hasher}.
	
	При создании каталога \Sys{hasher} следует учитывать два правила:
	\begin{enumerate}
		\item права доступа соответствуют \Sys{drwxr-xr-x}, то есть каталог доступен на запись;
		\item на файловой системе, смонтированной с \Sys{noexec} или \Sys{nodev}, каталог располагать нельзя%
		\footnote{\href{https://serverfault.com/questions/547237/explanation-of-nodev-and-nosuid-in-fstab}{https://serverfault.com/questions/547237/explanation-of-nodev-and-nosuid-in-fstab}};
		\begin{itemize}
			\item \Sys{noexec} устанавливается, если в системе есть файлы с правами \Sys{rwsr-xr-x}
			(запустить исполняемые файлы с правами владельца или группы) и владельцем \Sys{root}.
			Запустить файл \Sys{rwxr-xr-x} на такой файловой системе невозможно. а следовательно,
			и создать корректное сборочное окружение. \Sys{Hasher} не зависит от пользовательского окружения.
			\item \Sys{nodev} говорит о том, что на файловой системе не будут созданы файлы устройств.
			Это не соответствует функциональности \Sys{hasher} (см.~раздел 5.7
			\hyperlink{mount_fs_hasher}{<<Монтирование файловых систем внутри \Sys{Hasher}>>}).
		\end{itemize}
	\end{enumerate}
\end{itemize}
Если всё сделано верно, мы сможем создать минимальное сборочное окружение, исполнив команду:
\begin{verbatim}
	$ hsh ----initroot-only
\end{verbatim}
По умолчанию сборочное окружение создается в каталоге \Sys{$\sim$/hasher}. Для инициализации окружения
этот каталог должен существовать. Можно переобозначить путь для сборочного окружения по умолчанию.
Для этого нужно создать каталог, в котором планируем расположить новое сборочное окружение и выполнить
команду с переобозначенным путем:
\begin{verbatim}
	$ mkdir ~/some-new-hasher
	$ hsh ----initroot-only ~/some-new-hasher
\end{verbatim}
Окружение сбрасывается для каждого нового запущенного процесса сборки, и создается заново, если
не существовало. При необходимости содержимое каталога может быть очищено с правами
супер пользователя.\\
\\
Часто используемые параметры конфигурации $\sim$/.hasher/config:
\begin{itemize}
	\item \Sys{no\_sisyphus\_check="packager,buildhost,gpg"} ---  параметр необходимо  добавить в конфигурационный
	файл для работы с примерами обучающего пособия. Рекомендуемый параметр для работы в локальном
	окружении. Отключает проверки излишних значений в условиях локальной сборки;
	\item \Sys{packager='' ` rpm ----eval \%packager ` ''} --- альтернативный вариант указать параметр \Sys{packager} для \Sys{rpm}.
	(Будет работать если заполнен макрос  \Sys{\%packager} в \Sys{$\sim$/.rpmmacros});
	\item \Sys{allowed\_mountpoints=/dev/pts,/proc,/dev/shm} --- параметр, позволяющий подключать в корень изолированного сборочного
	окружения список локальных файловых систем;
	\item \Sys{lazy\_cleanup=1} --- параметр очищает среду сборки перед каждой новой сборкой;
	\item \Sys{apt\_config="\$HOME/.hasher/apt.conf"} --- параметр позволяет переобозначить конфигурацию источников
	репозиториев для сборочного окружения.
\end{itemize}

\section{Описание системы Hasher}
Переместиться в корень сборочного окружения можно выполнив команду:
\begin{verbatim}
	$ hsh-shell
\end{verbatim}
Приведенная команда перемещает командную оболочку в изолированную среду \Sys{hasher}.
Команда позволяет перемещаться по структуре изолированного окружения и продолжить работу
внутри.\\
\\
Опишем структуру каталогов \Sys{hasher}.

\begin{itemize}
	\item \Sys{$\sim$/hasher}
	\begin{itemize}[$\circ$]
		\item \Sys{chroot} --- сборочное окружение. В этом каталоге находится
		корневое дерево содержащее минимальный набор пакетов, необходимых
		для сборки.
		\item \Sys{aptbox} --- набор утилит для установки, обновления и удаления
		пакетов \Sys{chroot}. Например, тут лежит модифицированный \Sys{apt-get},
		с помощью которого происходит установка пакетов в \Sys{chroot}.
		\item \Sys{cache} --- в этом каталоге хранятся временные файлы, необходимые для
		создания \Sys{chroot}.
	\end{itemize}
	\item \Sys{repo}, который содержит подкаталоги:
	\begin{itemize}[$\circ$]
		\item \Sys{SRPMS.hasher} --- пакеты с исходными данными (sources).
		\item \Sys{<архитектура>/RPMS.hasher/} --- каталог с пакетами, собранными
		под конкретную архитектуру.
		Содержимое каталога дополняет источники пакетов для сборочного окружения.
		Помещенные в него пакеты можно установить в изолированном окружении \Sys{hasher}.
		Такой механизм позволяет добавлять пакеты в сборочные зависимости и использовать пакеты в
		изолированной среде, которые ещё не существуют в подключенных репозиториях.
	\end{itemize}
	\item \Sys{$\sim$/.hasher} --- каталог конфигурационных файлов. Каталог может отсутствовать,
	в этом случае \Sys{hasher} использует конфигурацию по умолчанию.
	\begin{itemize}[$\circ$]
		\item \Sys{apt.config} --- конфигурация для \Sys{apt-get} из \Sys{$\sim$/hasher/aptbox/}.
		\item \Sys{config} --- конфигурация самого \Sys{hasher}.
	\end{itemize}
	\item \Sys{/etc/hasher-priv/} каталог с конфигурацией для вспомогательной утилиты \Sys{hasher-priv}.
	\begin{itemize}[$\circ$]
		\item \Sys{./user.d} --- каталог содержит файлы настроенных пользователей для работы с \Sys{hasher}.
		\item \Sys{fstab} --- информация о точках монтирования вспомогательной программы \Sys{hasher-priv}
		\Sys{system} --- конфигурация вспомогательной программы \Sys{hasher-priv}.
	\end{itemize}
\end{itemize}

В структуре каталогов \Sys{hasher} стоит обратить внимание на служебные подкаталоги,
позволяющие взаимодействовать с локальной системой:
\begin{itemize}
	\item \Sys{$\sim$/hasher/chroot/.in} --- предполагает добавление файлов из локальной системы.
	\item \Sys{$\sim$/hasher/chroot/.out} --- предполагает получение файлов из \Sys{chroot} системы.
\end{itemize}

Непосредственно структура каталогов \Sys{rpmbuild} сборки находится:
\begin{verbatim}
	~/hasher/chroot/usr/src/RPM
\end{verbatim}

\Sys{Hasher} умеет монтировать внутрь изолированной среды виртуальные файловые системы
из локальной машины%
\footnote{\href{https://www.altlinux.org/Hasher/\%D0\%A0\%D1\%83\%D0\%BA\%D0\%BE\%D0\%B2\%D0\%BE\%D0\%B4\%D1\%81\%D1\%82\%D0\%B2\%D0\%BE\#\%D0\%9C\%D0\%BE\%D0\%BD\%D1\%82\%D0\%B8\%D1\%80\%D0\%BE\%D0\%B2\%D0\%B0\%D0\%BD\%D0\%B8\%D0\%B5_\%D1\%84\%D0\%B0\%D0\%B9\%D0\%BB\%D0\%BE\%D0\%B2\%D1\%8B\%D1\%85_\%D1\%81\%D0\%B8\%D1\%81\%D1\%82\%D0\%B5\%D0\%BC_\%D0\%B2\%D0\%BD\%D1\%83\%D1\%82\%D1\%80\%D0\%B8_hasher}{https://www.altlinux.org/Hasher/Руководство}}.
Этот механизм применяется в тех случаях, когда собираемому приложению для сборки требуется доступ к ресурсам
основной машины, которые \Sys{Hasher} не предоставляет по умолчанию. Например, виртуальная файловая система
\Sys{/proc} или \Sys{/dev/pts}, которых по умолчанию нет в \Sys{hasher}-контейнере. Файловая система \Sys{/proc}
получает информацию о состоянии и конфигурации ядра и системы.

Для монтирования файловой системы следует:
\begin{enumerate}
	\item В файле \Sys{/etc/hasher-priv/fstab} описать файловую систему.
	\item В файле \Sys{/etc/hasher-priv/system} указать файловую систему с помощью опции \Sys{allowed\_mountpoints}.
	\item Указать файловую систему либо при запуске \Sys{Hasher} в опции \Sys{----mountpoints}, либо в
	конфигурационном файле \Sys{$\sim$/.hasher/config} в ключе \Sys{known\_mount\-points}.
	\item Прописать необходимую файловую систему в \Sys{spec}-файле в теге \Sys{BuildReq}, либо в списке зависимостей.
\end{enumerate}


\section{Сборка в Hasher}
Сборка выполняется от предварительно настроенного пользователя
(см.~раздел~5.1 \hyperlink{5.1}{<<Базовая настройка \Sys{Hasher}>>}).

Изначально сборка в \Sys{hasher} предполагает наличие подготовленного \Sys{.src.rpm}-пакета
(см.~раздел 3.3 \hyperlink{rpm-pack-desc}{\mbox{<<Описание \Sys{RPM}-пакета>>}}),
или специально подготовленного \Sys{tar}-архива.

Для запуска сборки необходим подготовленный \Sys{.src.rpm} (см.~раздел~4
\hyperlink{rpmbuild-exampl-src}{<<Пример оборки пакета с исходными данными \Sys{.src.rpm}>>}),
после чего можно запустить сборку, выполнив команду:

\begin{verbatim}
	$ hsh ./<имя пакета>.src.rpm
\end{verbatim}

Для сборки можно передать специально подготовленный архив \Sys{.tar}. Приведем упрощенный пример
использования архива для сборки.

Пример структуры каталога для архива:
\begin{verbatim}
	./
	├── <имя пакета>.spec
	├── <имя патча>.patch
	└── <имя пакета>-<версия из spec>.tar
	  └── <имя пакета>-<версия из spec> --- Имя каталога с исходными данными для пакета.
\end{verbatim}

Упрощенный пример команды для создания \Sys{tar}-архива:
\begin{verbatim}
	$ tar ----create ----file=pkg.tar ----label=<имя пакета>.spec \
	<имя пакета>-<версия из spec>.tar <имя патча>.patch <имя пакета>.spec
\end{verbatim}

Подготовленный архив можно запустить на сборку, выполнив команду:
\begin{verbatim}
	$ hsh ./pkg.tar
\end{verbatim}

Собранный бинарный пакет появляется в директории:\\ \Sys{$\sim$/.hasher/repo/<архитектура>/RPMS.hasher/}.\\

Собрать пакет в изолированной среде можно вручную из исходных данных, инструмент \Sys{rpmbuild} (пример работы
в главе~4  \hyperlink{rpmbuild}{<<Инструмент rpmbuild>>}). Для этого подготовленный файл \Sys{.spec} и архив
с исходными данными нужно поместить в сборочное окружение, воспользовавшись служебным каталогом
\Sys{$\sim$/hasher/chroot/.in}, и провести сборку \Sys{rpmbuild} внутри подготовленного сборочного окружения.

Из-за своей сложности инструмент не используется отдельно и приведен для ознакомления и
подготовки к работе в дальнейшем. Полноценно функционал инструмента раскрывается в связке
с \Sys{GEAR} (инструмент GEAR будет описан в следующей главе).

\section{Вопросы для самопроверки}

\begin{enumerate}
	\item Что такое \Sys{Hasher} и для чего он предназначен?
	\item Какова структура каталогов \Sys{hasher}?
	\item Почему ранее собранные в \Sys{hasher} пакеты не оказывают влияния на новую сборку?
	\item Можно ли указать из какого репозитория будет происходить сборка в \Sys{hasher}?
	\item Какие шаги, помимо установки, необходимо сделать для настройки \Sys{hasher}?
	\item Какие сценарии сборки возможны при использовании \Sys{hasher}?
	\item Как посмотреть справочную информацию по пакету \Sys{hasher} и \Sys{hasher-priv}?
	\item Зачем монтировать внутрь \Sys{hasher} файловые системы?
\end{enumerate}
\chapter{Примеры использования инструментов ОС Альт для сборки пакетов}

В текущих разделах представлен материал с практическими примерами по сборке пакетов. Простыми шагами разберём подготовку системы к работе с инструментами, описанными в предыдущих главах пособия. Ниже следуют примеры подготовки и сборки \Sys{rpm}-пакета инструментами \Sys{rpmbild}, \Sys{gear} и \Sys{hasher}.

\begin{itemize}
	\item \textbf{Пример №1. Сборка пакета с помощью \Sys{rpmbuild}}.
	
	Подготовьте дерево каталогов для сборки пакета с помощью \Sys{rpmbuild}. Сформируйте \Sys{.tar} архив с исходным кодом программы и загрузите в каталог \Sys{SOURCES}. Назовите архив в соответствии с именем \Sys{SPEC}-файла. В каталог \Sys{SPECS} загрузите \Sys{SPEC}-файл. 
	
	\begin{verbatim}
		$ cd ~/RPM
		$ tree
		.
		|-- BUILD
		|   `-- HelloUniverse-1.0
		|-- RPMS
		|   |-- noarch
		|   `-- x86_64
		|-- SOURCES
		|   `-- HelloUniverse-1.0.tar
		|-- SPECS
		|   `-- HelloUniverse.spec
		`-- SRPMS
	\end{verbatim}
	
	Для сборки \Sys{RPM}-пакета выполните команду: 
	\begin{verbatim}
		$ rpmbuild -bb ./SPECS/HelloUniverse.spec 
		...
		$ ls ./RPMS/x86_64/
		HelloUniverse-1.0-alt1.x86_64.rpm
		HelloUniverse-debuginfo-1.0-alt1.x86_64.rpm
	\end{verbatim}
	
	\item \textbf{Пример №2. Сборка из \Sys{.src.rpm} с помощью \Sys{hasher}.}
	
	Возьмите готовый файл \Sys{.src.rpm} и выполните команду:
	\begin{verbatim}
		$ hsh -v HelloUniverse-1.0-alt1.src.rpm  --no-sisyphus-check
	\end{verbatim} 
	
	Результат успешного завершения сборки: 
	\begin{verbatim}
		$ ls ~/hasher/repo/x86_64/RPMS.hasher/
		HelloUniverse-1.0-alt1.x86_64.rpm
		HelloUniverse-debuginfo-1.0-alt1.x86_64.rpm
	\end{verbatim}
	
	\item \textbf{Пример №3. Сборка с помощью \Sys{gear-hsh}.}
	
	Выполните команду: 
	\begin{verbatim}
		$ gear-hsh -v --no-sisyphus-check 
	\end{verbatim}
	
	Результат успешного завершения сборки --- собранный \Sys{RPM}-пакет и \Sys{.src.rpm}:
	\begin{verbatim}
		$ ls ~/hasher/repo/x86_64/RPMS.hasher/
		HelloUniverse-1.0-alt1.x86_64.rpm
		HelloUniverse-debuginfo-1.0-alt1.x86_64.rpm
		
		$ ls ~/hasher/repo/SRPMS.hasher/
		HelloUniverse-1.0-alt1.src.rpm
	\end{verbatim} 
	
	Установите собранный пакет командой: 
	\begin{verbatim}
		$ sudo apt-get install ~/hasher/repo/x86_64/RPMS.hasher/\
		HelloUniverse-1.0-alt1.x86_64.rpm
	\end{verbatim}
\end{itemize} 


\section{Подготовка пространства}
Соберем в \Sys{rpm}-пакет собственную программу. В каталоге сохраним файлы с исходным кодом --- \Sys{HelloUniverse.cpp} и \Sys{Makefile}.
\begin{verbatim}
	$ ls -la
	HelloUniverse.cpp
	Makefile
\end{verbatim}

\Emph{Подготовка окружения}

Инструменты, необходимые для сборки: 
\begin{itemize}
	\item система контроля версий Git;
	\item текстовый редактор (Vim, Emacs, MC);
	\item \Sys{rpmbuild};
	\item \Sys{Hasher};
	\item \Sys{Gear}.
\end{itemize}

Для работы с \Sys{rpmbuild} установите также сборочные зависимости:
\begin{itemize}
	\item make;
	\item gcc-c++.
\end{itemize} 

После установки программ: 
\begin{enumerate}
	\item Настройте Git;
	\item Создайте \Sys{gpg}-ключ подписи для работы с \Sys{gear};
	\item Настройте окружение \Sys{RPM};
	\item Настройте окружение \Sys{Hasher}.
\end{enumerate}


\section{Написание \Sys{SPEC}-файла и правил \Sys{Gear}}
Подготовка \Sys{gear}-репозитория.

Перейдите в каталог с исходным кодом и проинициализируйте репозиторий: 
\begin{verbatim}
	$ cd HelloUniverse/
	$ git init	
\end{verbatim}

Создайте каталог \Sys{.gear}, перейдите в него и создайте файлы \Sys{HelloUniverse.spec} и \Sys{rules}:
\begin{verbatim}
	$ mkdir .gear
	$ cd .gear
	$ touch HelloUniverse.spec
	$ touch rules
\end{verbatim} 

Заполните \Sys{SPEC}-файл \Sys{HelloUniverse.spec}:
\begin{verbatim}
	%define _unpackaged_files_terminate_build 1
	
	Name:       HelloUniverse
	Version:    1.0
	Release:    alt1
	Summary:    Most simple RPM package
	License:    no
	Group:	Development/Other
	Source: %name-%version.tar
	BuildRequires: gcc-c++
	%description
	This is my first RPM
	
	%prep
	%setup
	
	%build
	%make_build HelloUniverse
	
	%install
	mkdir -p %{buildroot}%_sbindir
	install -m 755 HelloUniverse %{buildroot}%_sbindir
	
	%files
	%_sbindir/%name
	
	%changelog
\end{verbatim} 

В файл \Sys{rules} пропишите:
\begin{verbatim}
	tar: . name=@name@-@version@ base=@name@-@version@
	spec: .gear/HelloUniverse.spec
\end{verbatim} 

Перейдите на уровень выше, в корневой каталог проекта, и сделайте первый коммит: 
\begin{verbatim}
	$ cd ../ 
	$ git add HelloUniverse.cpp
	$ git add Makefile
	$ git add .gear/HelloUniverse.spec
	$ git add .gear/rules
	$ git commit -m "first commit"
\end{verbatim}

Теперь у нас есть готовая к сборке в \Sys{RPM}-пакет программа. 

После того, как мы убедимся, что программа собирается в пакет и работает, рекомендуется написать \Sys{changelog}, сделать новый коммит с релизом и поставить тег. 
\chapter*{Заключение}\addcontentsline{toc}{chapter}{Заключение}
Материал пособия должен помочь освоить настройку инфраструктуры для разработки и сборки программных пакетов. Разделы глав сформированы в соответствии с рекомендуемым порядком освоения инструментов --- от простого к сложному. 

Вначале описываются основные термины и понятия --- системы управления пакетами низкого и высокого уровня. Дается структура \Sys{rpm}-пакета и объясняются базовые методы работы с пакетами в системах Альт. Инфраструктура разработки программных пакетов и сборки программного обеспечения описана в \hyperlink{3}{3}, \hyperlink{4}{4}, \hyperlink{5}{5}, \hyperlink{6}{6} главах и включает:
\begin{itemize}
	\item пакетный менеджер \Sys{RPM};
	\item систему управления пакетами \Sys{APT};
	\item контейнер для сборки пакетов \Sys{Hasher};
	\item набор инструментов \Sys{Gear}.
\end{itemize}

В заключительном разделе приводится практический пример подготовки программы на языке \Sys{С++} и упаковки ее в \Sys{rpm}-пакет различными способами. 
\end{document}
