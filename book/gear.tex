\chapter{Инструмент Gear}\label{chapter-gear}
\Emph{GEAR (Get Every Archive from git package Repository)} --- 
инструмент для поддержки и сборки пакетов из \Sys{git}-репозиториев%
\footnote{\href{https://www.altlinux.org/Gear}{https://www.altlinux.org/Gear}}%
\footnote{\href{https://www.altlinux.org/Gear/\%D0\%A1\%D0\%BF\%D1\%80\%D0\%B0\%D0\%B2\%D0\%BE\%D1\%87\%D0\%BD\%D0\%B8\%D0\%BA}{https://www.altlinux.org/Gear/Справочник}}.

\marginalia{ex_sign_col}{Для работы с текущим разделом необходимо изучить 
документацию по работе с распределённой системой управления версиями \Sys{Git}.}

\Emph{Git} --- это технология контроля версий, которая следит за изменениями в файлах. 
\Sys{Git}-репозиторий хранит исходный код и данные для сборки пакетов, включая историю 
изменений, сведения о версиях, авторах изменений и прочую информацию. 

При сборке пакетов \Sys{gear} предлагает работать в том же \Sys{git}-репозитории, 
в котором хранятся исходные тексты пакета. \Sys{Gear} позволяет целиком импортировать 
историю разработки, предоставляет различные средства импорта исходных текстов и 
различные варианты организации репозитория. 

Идея \Sys{gear} в доступности всего необходимого для сборки пакета в \Sys{git}-репозитории, 
либо в репозитории пакетов, на основе которого ведётся сборка. Система контроля версий 
\Sys{Git} предоставляет встроенные механизмы обеспечения целостности (контрольные суммы, 
криптографически подписанные теги) для задач управления пакетами. Появляется возможность 
воспроизводимой сборки --- собрать <<такой же>> пакет ещё раз, опираясь на логически 
законченную версию изменений зафиксированную на момент времени --- коммит (commit) в 
терминологии git.

Сборка пакета ведётся из конкретного коммита, который мы в дальнейшем будем называть главным. 
Именно в нём должна находится нужная версия \Sys{SPEC}-файла и правил экспорта. \Sys{GEAR} 
позволяет экспортировать из \Sys{git}-репозитория каталоги и подкаталоги в виде архивов; 
экспортировать отдельные файлы; вычислять разницу (\Sys{diff}) и сохранять её в виде патча. 
При этом может использоваться как состояние в главном коммите, так и в любом из его предков, 
прямых и не прямых. Это позволяет гибко и удобно организовать работу по поддержке пакета: 
все нужные исходные тексты можно хранить в \Sys{git} так, чтобы с ними было удобно работать, 
а затем экспортировать так, чтобы ими было удобно воспользоваться из \Sys{SPEC}-файла.

Правила экспорта для \Sys{gear} описываются в текстовом файле, который также хранится в 
репозитории. По умолчанию \Sys{gear} ищет этот файл по пути \Sys{.gear/rules} или \Sys{.gear-rules}. 
Подробнее про синтаксис этого файла можно прочитать ниже, в разделе <<Правила экспорта>>.

Для удобства сборки пакетов \Sys{gear} интегрирован с инструментами \Sys{rpm-build} и 
\Sys{hasher}: из \Sys{gear}-репозитория можно одной командой собрать пакет при помощи 
\Sys{rpmbuild} или \Sys{hsh}. Подробнее об этом можно прочитать далее в разделе <<Быстрый старт>>. 


\section{Быстрый старт Gear}
\Emph{Установка gear}

В разделе \hyperlink{1.3}{1.3 <<Установка необходимых пакетов для процесса сборки>>} 
также упоминается инструмент \Sys{gear}. Для отдельной установки используйте команду: 
\begin{verbatim}
	#apt-get install gear
\end{verbatim}

\Emph{Импорт .src.rpm }

Пакет в формате \Sys{source RPM} можно импортировать в \Sys{gear}-репозиторий командой \Sys{gear-srpmimport}:
\begin{verbatim}
	mkdir package_name
	cd package_name
	git init -b sisyphus
	gear-srpmimport /путь/к/package_name.src.rpm
\end{verbatim}

\Emph{Получение готового репозитория с внешнего git-сервера}

Достаточно клонировать репозиторий: 
\begin{verbatim}
	git clone <repository url> package_name
	cd package_name
\end{verbatim}

Никаких дополнительных настроек не требуется.

\Emph{Сборка пакета}

Основным инструментом экспорта и сборки пакетов является команда \Sys{gear}. На практике удобно 
использовать предоставляемые команды-обёртки, а к самой команде \Sys{gear} прибегать только в 
самых сложных случаях. Командной обёртке можно передавать опции как утилиты \Sys{gear}, так и 
вложенной утилиты.

Командная оберка \Sys{gear} для \Sys{rpmbuild} --- \Sys{gear-rpm}. Для сборки пакета используйте команду: 
\begin{verbatim}
	gear-rpm --verbose -ba
\end{verbatim}


Собрать пакет при помощи \Sys{hsh} (установка и настройка \Sys{hasher} описана в следующих главах): 
\begin{verbatim}
	gear-hsh --verbose
\end{verbatim}

Команде \Sys{gear-hsh} можно передать как аргументы \Sys{gear}, так и аргументы \Sys{hsh}.

Если не указана опция \Sys{--tree-ish}, \Sys{gear} в качестве главного коммита использует текущий 
коммит (HEAD). Если хочется проверить свежие изменения без создания коммита, можно использовать опцию 
\Sys{--commit}, доступную и для \Sys{gear}, и для \Sys{gear-rpm}, и для \Sys{gear-hsh}. В таком случае 
будет создан временный коммит (аналогично \Sys{git commit -a}), и пакет будет собран уже из него. 
Стоит отметить, что, аналогично \Sys{git commit -a}, в таком коммите не будет новых файлов, если они 
ещё не были добавлены в \Sys{git} командой \Sys{git add}; если они нужны для сборки, то их стоит добавить. 

\paragraph{Сборка пакета из репозитория разработчиков}
Для примера возьмём конкретный пакет, который уже есть в репозитории Сизиф -- \Sys{pixz} -- и представим 
себе, что хотим собрать его с нуля.

Для начала клонируем репозиторий. Сразу зададим имя удалённого репозитория: 
\begin{verbatim}
	git clone https://github.com/vasi/pixz -o upstream
	cd pixz
\end{verbatim}

Перейдём в ветку, из которой будем собирать пакет: 
\begin{verbatim}
	git checkout -b sisyphus upstream/master
\end{verbatim}

Переместимся на тег(тег --- это ссылка, указывающие на определённый \Sys{git}-коммит в репозитории), 
соответствующий версии, которую мы будем собирать. На данный момент это последний тег: 
\begin{verbatim}
	git reset --hard v1.0.7
\end{verbatim}

Определим правила экспорта: 
\begin{verbatim}
	mkdir .gear
	vim .gear/rules
\end{verbatim}

Правила зададим следующие: 
\begin{itemize}
	\item \Sys{SPEC}-файл перенесём в каталог \Sys{.gear}, чтобы он не путался с основными исходными текстами;
	\item исходники будем забирать из тега \Sys{GEAR}, соответствующего апстримному;
	\item сразу создадим патч, включающий все наши изменения, каталог \Sys{.gear} в этот патч мы включать не будем.
\end{itemize}

Получаем следующий файл \Sys{.gear/rules}:
\begin{verbatim}
	spec: .gear/pixz.spec
	tar: v@version@:.
	diff: v@version@:. . exclude=.gear/**
\end{verbatim}

Создадим соответствующий версии тег \Sys{GEAR}:
\begin{verbatim}
	gear-store-tags v1.0.7
\end{verbatim} 

Напишем \Sys{SPEC}-файл:
\begin{verbatim}
	vim .gear/pixz.spec
\end{verbatim} 

Важно, что версия в спеке --- \Sys{1.0.7}, поэтому конструкция \Sys{v@version@}, 
которую мы использовали в \Sys{.gear/rules}, раскроется в строку \Sys{v1.0.7} --- 
именно такой тег \Sys{GEAR} мы создали. 

Добавим каталог \Sys{.gear} в \Sys{git}:
\begin{verbatim}
	git add .gear
\end{verbatim} 

Можно попробовать собрать пакет: 
\begin{verbatim}
	gear-rpm --verbose --commit -ba
\end{verbatim}

При необходимости, можно поправить \Sys{spec}, добавить нужные зависимости. 
Когда пакет собирается и всем устраивает, можно зафиксировать все изменения: 
\begin{verbatim}
	gear-commit -a
\end{verbatim}
