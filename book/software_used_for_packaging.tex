\hypertarget{3}{\chapter{Общая информация о сборке RPM-пакета}}\label{software-used-for-packaging}

\Emph{Сборка} --- формирование пакета на основе специальных сборочных (spec) инструкций.

Сборка также устоявшееся определение компиляции исходного кода (формирование машинных
инструкций), в случае \Sys{RPM} термин Сборка более общий, и в некоторых случаях включает
компиляцию.

Использование специализированных пакетов, в отличие от обычного файлового архива, имеет ряд
преимуществ\footnote{\href{https://rpm-packaging-guide-ru.github.io/\#Why-Package-Software-with-RPM}
	{https://rpm-packaging-guide-ru.github.io/\#Why-Package-Software-with-RPM}}:

\begin{itemize}
	\item Пользователи могут использовать средства управления пакетами --- пакетные менеджеры
		(например, \Sys{Yum} или 	\Sys{PackageKit}) для установки, переустановки, удаления,
		обновления и проверки \Sys{rpm}-пакетов.
	\item Пакетный менеджер \Sys{RPM} предполагает наличие базы данных, которая с помощью специализированных
		утилит позволяет получать информацию о пакетах в системе.
	\item Каждый пакет \Sys{rpm} содержит метаданные, описывающие компоненты пакета: версию, выпуск,
		размер, URL проекта, установочные инструкции и~т.\,д.
	\item \Sys{RPM} позволяет брать оригинальные источники исходных данных и упаковывать их в
		пакеты с исходными данными (\Sys{.src.rpm}) и в бинарные пакеты (\Sys{.rpm}). В пакетах \Sys{.src.rpm}
		хранятся оригинальные исходные данные вместе со всеми изменениями (\Sys{*.patch}), а также сборочные
		инструкции (\Sys{.spec}) и дополнительная информация. В бинарных пакетах вместо исходного кода
		упакованы подготовленные файлы и скрипты установки, но нет сборочных инструкций. Ещё существуют
		случаи распространения пакетов без исходного кода и бинарных данных. В таких пакетах присутствуют
		скрипты для скачивания и модификации файлов, необходимые для работы приложения, распространяемого таким способом.
	\item Для обеспечения верификации подлинности \Sys{rpm}-пакетов используется механизм электронных цифровых
		подписей \Sys{GPG}. Он позволяет подписать \Sys{rpm}-пакет или обновить цифровую подпись:
	\Sys{rpm -addsign package.rpm} и \Sys{rpm -resign package.rpm}.
	\item Вы можете добавить свой пакет в \Sys{rpm}-репозиторий, что позволит клиентам легко находить и
		устанавливать ваше программное обеспечение.
\end{itemize}

\marginalia{ex_sign_col}{Надо понимать, что процесс сборки пакета поддерживает предварительную подготовку исходных
данных к использованию в операционной системе(например ,такую как компиляция), но это не обязательно. Случай сборки уже
готовых данных(например, графические изображения) иногда называют --- упаковка в rpm-пакет. }

Задача сборки пакета начинается со сбора всех необходимых компонентов и завершается этапами сборки и тестирования.

Классическая сборка пакетов \Sys{rpm} состоит из следующих этапов:%\footnote{\href{https://alt-packaging-guide.github.io/}{https://alt-packaging-guide.github.io/}}:

\begin{itemize}
	\item поиск исходных данных;
	\item написание инструкций сборки;
	\item сборка пакета.
\end{itemize}

\section{Набор инструментов, необходимый для сборки}

Опишем основные инструменты для сборки пакетов. Технологическую
базу репозитория Sisyphus составляют адаптированные к нуждам команды разработчиков
программы и специально разработанные решения%
\footnote{\href{https://www.altlinux.org/\%D0\%A0\%D0\%B5\%D0\%BF\%D0\%BE\%D0\%B7\%D0\%B8\%D1\%82\%D0\%BE\%D1\%80\%D0\%B8\%D0\%B9_\%D0\%A1\%D0\%9F\%D0\%9E\#APT_\%D0\%B8_\%D1\%80\%D0\%B5\%D0\%BF\%D0\%BE\%D0\%B7\%D0\%B8\%D1\%82\%D0\%BE\%D1\%80\%D0\%B8\%D0\%B9_\%D0\%BF\%D0\%B0\%D0\%BA\%D0\%B5\%D1\%82\%D0\%BE\%D0\%B2}{https://www.altlinux.org/Репозиторий\_СПО\#APT\_и\_репозиторий\_пакетов}.}:
\begin{itemize}
	\item \Emph{RPM}  В контексте данной темы \Sys{RPM} рассматривается не только как менеджер пакетов,
		но и как набор инструментов для их сборки (\Sys{rpmbuild}). Отличительные особенности \Sys{RPM}
		в Sisyphus --- удобное поведение <<по умолчанию>> для уменьшения количества шаблонного кода
		в \Sys{.spec}-файлах, обширный модульный набор
		макросов для упаковки различных типов пакетов, развитые механизмы автоматического вычисления
		межпакетных зависимостей при сборке пакетов, поддержка \Sys{set}-версий в зависимостях на
		разделяемые библиотеки, автоматическое создание пакетов с отладочной информацией с поддержкой
		зависимостей между такими пакетами.
	\item \Emph{Hasher} --- инструмент, который позволяет производить сборку \Sys{rpm}-пакетов в изолированной среде,
		что обеспечивает повышенную безопасность и идентичный результат, не зависимый от стороннего
		программного обеспечения, установленного в системе.
	\item \Emph{GEAR} --- инструмент направленный на упрощение работы с источниками исходных данных для пакетов
		и облегчение процесса сборки.
\end{itemize}

\section{Программное обеспечение для сборки RPM-пакетов}

Перечислим набор базовых пакетов и входящих в их состав программ, необходимых для
сборки rpm-пакета%
\footnote{\href{https://www.altlinux.org/\%D0\%A1\%D0\%B1\%D0\%BE\%D1\%80\%D0\%BA\%D0\%B0_\%D0\%BF\%D0\%B0\%D0\%BA\%D0\%B5\%D1\%82\%D0\%B0_\%D1\%81_\%D0\%BD\%D1\%83\%D0\%BB\%D1\%8F}%
{https://www.altlinux.org/Сборка\_пакета\_с\_нуля}}.
Необходимо отметить, что перечень может отличаться для различных исходных данных
и выбранных инструментов сборки пакетов, которые подробнее будут разобраны позже.

\begin{itemize}
	\item \Emph{rpmdevtools} --- пакет с набором программ для сборки пакетов:
	\begin{itemize}
		\item \Emph{rpmdev-setuptree} --- утилита для создания структуры рабочих каталогов;
		\item \Emph{rpmdev-newspec} --- утилита для создания \Sys{spec}-файла.
	\end{itemize}
	\item \Emph{rpmspec} --- утилита работы с файлами спецификации --- текстовыми файлами с расширением
		\Sys{.spec}. Утилита служит для проверки подготовленного \Sys{spec}-файла;
	\item \Emph{rpm-build} --- пакет содержит сценарии и исполняемые программы, которые используются для сборки пакетов с помощью \Sys{RPM}:
	\begin{itemize}
		\item \Emph{rpmbuild} --- утилита сборки \Sys{rpm}-пакета из набора подготовленных файлов.
	\end{itemize}
	\item \Emph{rpmlint} --- утилита для тестирования собранного \Sys{rpm}-пакета;
	\item \Emph{rpm-utils} --- пакет с набором программ для работы с \Sys{rpm}-пакетами;
	\item \Emph{gcc} --- набор компиляторов для различных языков программирования, разработанный в рамках проекта GNU;
	\item \Emph{make} --- инструмент GNU, упрощающий процесс сборки для пользователей;
	\item \Emph{python} --- интерпретируемый интерактивный объектно-ориентированный язык программирования;
	\item \Emph{patch} --- программа исправлений применяет патчи к оригиналам;
	\item \Emph{gear} --- пакет, содержащий утилиты для сборки пакетов \Sys{rpm} из \Sys{gear}-репозитория и управления \Sys{gear}-репозиториями;
	\item \Emph{hasher} --- современная технология создания независимых от сборочной системы пакетов.
\end{itemize}

\hypertarget{rpm-pack-desc}{\section{Описание RPM-пакета}}

\Emph{rpm-пакет} --- это специальный архив с файлами. Сам файл пакета состоит из четырёх секций:
начального идентификатора, сигнатуры, бинарного заголовка и \Sys{cpio}-архива с файлами проекта в структуре
каталогов\footnote{\href{https://www.opennet.ru/docs/RUS/rpm_guide/13.html}{https://www.opennet.ru/docs/RUS/rpm\_guide/13.html}}.

\Sys{rpm}-пакеты делятся на несколько категорий: пакеты с исходным кодом и бинарные пакеты.
\begin{itemize}
	\item \Emph{rpm-пакет} (бинарный) --- это архив с расширением \Sys{.rpm}. Такой пакет готов к установке в
		операционной системе средствами \sys{RPM}.
	\item \Emph{srpm-пакет} (source RPM, пакет с исходным кодом) --- это архив с расширением \Sys{.src.rpm}.
		\Sys{SRPM} содержит исходный код, патчи, если необходимо, и \Sys{spec}-файл. Эти пакеты содержат
		всю информацию для сборки пакета.
\end{itemize}

По инструкциям из \Sys{spec}-файла собирается бинарный \Sys{rpm}-пакет%
\footnote{\href{https://uneex.ru/static/RedHatRPMGuideBook/rpm_guide-linux.html\#16_html}{https://uneex.ru/static/RedHatRPMGuideBook/rpm\_guide-linux.html\#16\_html}}. Инструкции содержат также информацию о правах доступа и их применении в
процессе установки, скрипты, запускаемые при установке или удалении пакета.

Файлы бинарного пакета могут быть скомпилированы под определённую процессорную архитектуру (аппаратную платформу).
На системах с разной процессорной архитектурой не получится использовать один и тот же
собранный бинарный \Sys{rpm}-пакет без дополнительных настроек и манипуляций.
Имя файла таких пакетов обычно содержит общепринятую маркировку архитектуры и имеет вид:
\Sys{*.\{архитектура\}.rpm}

Существуют бинарные пакеты, которые не зависят от процессорной архитектуры. Такие пакеты содержат
в своем имени маркировку архитектуры \Emph{noarch}. Имя платформо-независимого бинарного пакета
заканчивается на: \Sys{.noarch.rpm}. Например, упакованные картинки рабочего стола или наборы иконок
обычно имеют такое окончание.

На основе установленных в систему бинарных пакетах строится база данных в \Sys{/var/lib/rpm}. Вся
информация о пакетах хранится в базе данных \Sys{Packages}.

Принято называть пакеты \Sys{RPM}:
\begin{verbatim}
        {имя-пакета}.{имя-версия-релиз}.{архитектура}.rpm
\end{verbatim}

\marginalia{ex_sign_col}{Например, для процессоров ARM имя пакета будет выглядеть так: \Sys{admc-0.16.0-alt1.armh.rpm},
а для архитектурно-независимого пакета имя может выглядеть так: \Sys{icon-theme-simple-sl-2.7-alt4.noarch.rpm}.}

Имя установленного пакета в системе может отличаться от имени файла пакета\footnote{\href{https://www.opennet.ru/docs/RUS/rpm_guide/13.html}{https://www.opennet.ru/docs/RUS/rpm\_guide/13.html}}.

\hypertarget{3.4}{\section{Рабочее пространство для сборки RPM-пакетов}}
\Emph{Рабочее пространство сборки \Sys{rpm}-пакета} это структура файлов и каталогов. Эту структуру
можно создать двумя способами --- вручную или через утилиту \Sys{rpmdev-setuptree}.

Для подготовки ручным способом структуры каталогов выполните команду:
\begin{verbatim}
        $ mkdir -p ~/RPM/{BUILD,SRPMS,RPMS,SOURCES,SPECS}
\end{verbatim}

Альтернативный способ подготовки рабочей среды --- утилита \Sys{rpmdev-setuptree}. Утилита входит в
состав пакета \Sys{rpmdevtools} (см.\,\hyperlink{1.3}{раздел 1.3}). Для подготовки структуры каталогов
через утилиту \Sys{rpmdev-setuptree} выполните команду:
\begin{verbatim}
        $ rpmdev-setuptree
\end{verbatim}

Утилита создаст базовую структуру каталогов и файл \Sys{$\sim$/.rpmmacros}, если его не существовало.

Для системы ALT расположение структуры каталогов%
\footnote{\href{https://rpm-packaging-guide-ru.github.io/\#rpm-packaging-workspace}{https://rpm-packaging-guide-ru.github.io/\#rpm-packaging-workspace}}
 по умолчанию определяется в файле \Sys{$\sim$/.rpmmacros} и находится в каталоге \Sys{$\sim$/RPM}:
\begin{verbatim}
        $ tree ~/RPM
        /home/user/RPM
        ├── BUILD
        ├── RPMS
        ├── SOURCES
        ├── SPECS
        └── SRPMS

        5 directories, 0 files
\end{verbatim}

\marginalia{ex_sign_col}{Методическое пособие расcчитано на работу с файлом  \Sys{$\sim$/.rpmmacros}
вида:
\\\\
\Sys{\%\_topdir          \%homedir/RPM}
\\
\Sys{\%packager        Ваше\_имя Ваша\_фамилия <my\_mail@altlinux.org>}
\\\\
где имя, фамилия и почтовый адрес соответствуют используемым данным пользователя.
Остальные строки в файле рекомендуется удалить.
}

\begin{enumerate}
	\item \Emph{BUILD} --- в каталог попадают распакованные исходные файлы из
		\Sys{SOURCES} с уже применёнными патчами --- стадия \Sys{\%prep}.
		В каталоге \Sys{BUILD} происходит сборка программного обеспечения.
	\item \Emph{RPMS} --- в каталог \Sys{RPMS} попадают бинарные \Sys{rpm}-пакеты после
		сборки, в соответствии с подкаталогами для поддерживаемых архитектур.
	\item \Emph{SOURCES} --- в каталоге размещают архивы исходных данных и патчи.
	\item \Emph{SPECS} --- в каталоге размещают \Sys{spec}-файлы пакетов для сборки.
	\item \Emph{SRPMS} --- в каталог \Sys{SRPMS} попадают результаты сборки \Sys{srpm}-пакетов.
\end{enumerate}

Созданная структура каталогов становится рабочей областью упаковки \Sys{rpm}-пакета.

\marginalia{ex_sign_col}{В структуре сборочного окружения \Sys{RPM} существует понятие
\Sys{buildroot}. Это каталог \Sys{/TMP}, в который попадают служебные файлы в ходе сборки
пакета и уже подготовленные бинарные данные для упаковки в соответствии со структурой
каталогов. По умолчанию \Sys{buildroot} создаётся во временном системном каталоге,
но может быть переназначен.}


\section{Описание SPEC-файла}
\Emph{Spec}-файл --- RPM Specification File --- это текстовый файл, который описывает процесс сборки и конфигурацию
пакета, служит инструкцией для утилиты \Sys{rpmbuild}. Он содержит метаданные, такие как имя пакета, версию,
лицензию, а также разделы с инструкциями для сборки, установки и упаковки программного обеспечения, журнал
изменений пакета%
\footnote{\href{https://www.altlinux.org/Spec}%
{https://www.altlinux.org/Spec}}. %

\Sys{Spec}-файл можно рассматривать как <<инструкцию>>, которую утилита \Sys{rpmbuild} использует для сборки \Sys{rpm}-пакета.
\Sys{Spec}-файл состоит из трёх разделов: Header (Заголовок/Преамбула), Body (Тело) и Сhangelog (Журнал изменений).
\begin{enumerate}
	\item \Emph{Header} (Заголовок) --- этот раздел содержит метаданные о пакете, такие как его имя (Name),
		версия (Version), релиз (Release), краткое описание (Summary), лицензия (License) и другие
		параметры, которые идентифицируют и характеризуют пакет.
	\item \Emph{Body} (Тело) --- этот раздел содержит инструкции для процесса сборки пакета. В нём
		определяются различные секции, такие как \Sys{BuildRequires} (зависимости для сборки),
		\Sys{\%build} (инструкции для сборки), \Sys{\%install} (инструкции для установки),
		\Sys{\%files} (список файлов, включённых в пакет) и другие.
	\item \Emph{Changelog} (Журнал изменений) --- этот раздел содержит историю изменений пакета.
		Он содержит записи о внесённых изменениях, включая дату изменения, автора и краткое
		описание того, что было изменено.
\end{enumerate}

\Emph{Сборочная зависимость пакета (зависимость для сборки)} --- вид зависимостей пакета
необходимых в процессе его сборки. Компоненты такого пакета не участвуют в работе пакета после его установки в
систему и используются только на этапе сборки.

\Emph{Заголовок (Преамбула)}

Заголовок \Sys{spec}-файла содержит информацию о пакете: версию, исходный код, патчи, зависимости.

Рекомендуемый порядок заголовочных тегов:
\begin{itemize}
	\item Name, Version, Release, Serial;
	\item далее Summary, License, Group, Url, Packager, BuildArch;
	\item потом Source[номер архива с исходными данными], Patch[номер патча];
	\item далее PreReqs, Requires, Provides, Conflicts;
	\item и, наконец, Prefix, BuildPreReqs, BuildRequires.
\end{itemize}

Ниже приведён пример части \Sys{spec}-файла \Sys{notepadqq}:
\begin{verbatim}
        Summary:   A Linux clone of Notepad++
        Name:      notepadqq
        Version:   1.4.8
        Release:   alt2
        License:   GPLv3
        Group:     Editors
        URL:       http://notepadqq.altervista.org/wp/
        Source0:   %name-%version.tar
        Source1:   codemirror.tar
\end{verbatim}

\Emph{Тело}

Тело \Sys{spec}-файла отвечает за выполнение сборки, установки или очистки пакета.

Описание структуры%\footnote{\href{https://alt-packaging-guide.github.io/\#body-items}{https://alt-packaging-guide.github.io/\#body-items}}:
\begin{itemize}
	\item В секции \Sys{\%prep} производится распаковка архивов с исходными кодами и
		формируется директория с исходниками\footnote{\href{https://www.opennet.ru/docs/RUS/rpm_guide/48.html}{https://www.opennet.ru/docs/RUS/rpm\_guide/48.html}}.
	\begin{itemize}
		\item Макрос \Sys{\%setup} перемещает файлы и распаковывает архивы с исходными данными в каталог для сборки,
		выполняет переход в распакованный каталог (источники описаны в преамбуле в тегах Source[номер архива с исходными данными]).
		\item Макрос \Sys{\%patch[номер патча]} инструкции для индивидуального применение патча с номером (патчи описаны в преамбуле в тегах Patch[номер патча]).
	\end{itemize}
	\item В секции \Sys{\%build} внутри ранее подготовленной директории производится
		сборка программы. Если это компилируемый язык, то исходные данные компилируются в бинарные
		файлы. Если это интерпретируемый язык, то процесс может не подразумевать компиляцию.
		Обычно за процесс сборки отвечают системы сборки, отличающиеся для разных языков
		программирования. Для \Sys{C/C++} обычно используется \Sys{automake/autoconf} и макросы
		\Sys{\%configure} и \Sys{\%make\_build}. Есть и другие системы сборки с другими
		макросами --- \Sys{CMake}, \Sys{meson}, \Sys{pyproject} и~т.\,д.
	\item В секции \Sys{\%install} подготавливается новая директория с теми файлами, которые будут
		помещены в \Sys{rpm}-пакет в конце процесса сборки. Эта директория обозначается макросом
		\Sys{\%buildroot}. Из текущей директории подготовленные на предыдущем этапе файлы (бинарные
		файлы, файлы документации, конфигурационные файлы и~т.\,д.) нужно перенести в \Sys{\%buildroot}.
		Например, файл \Sys{build/application.bin} нужно перенести в \Sys{\%buildroot/usr/bin/application.bin}.
		За это в некоторых случаях может также отвечать система сборки, например, \Sys{automake/autoconf}
		так умеет, запускается через макрос \Sys{\%makeinstall\_std}. Для других систем сборки есть другие макросы.
	\item Возможно добавление секции \Sys{\%clean}. Её задача очистить дерево сборки и каталог установки.
	\item Если разработчик добавляет в \Sys{spec}-файл собственные скрипты, их следует распределять в секции:
	\begin{itemize}
		\item \Sys{\%pre} (выполнение перед установкой);
		\item \Sys{\%post} (выполнение после установки);
		\item \Sys{\%preun} (перед удалением пакета);
		\item \Sys{\%postun} (после удаления пакета).
	\end{itemize}
	\item Секция \Sys{\%files} содержит список путей и файлов, которые будут упакованы в \Sys{rpm}-пакет
		и в дальнейшем установлены в систему.
	\begin{itemize}
		\item В этой секции можно создать каталог (\Sys{\%dir}), отметить, что файл является
			документацией (\Sys{\%doc}) или файлом конфигурации (\Sys{\%config}), или файл не
			относится к пакету, но необходим в начале работы приложения (\Sys{\%gost}).
	\end{itemize}
\end{itemize}

\marginalia{ex_sign_col}{Сборка \Sys{rpm}-пакета выполняет все инструкции, указанные в \Sys{spec}-файле.
В процессе сборки утилита \Sys{rpmbuild} выводит на экран информацию о процессе сборки.
С помощью этой информации можно отследить возможные ошибки описания \Sys{spec}-файла.}

\Emph{Журнал изменений}

Секция журнал изменений (\Sys{\%changelog}) содержит оформленный в принятом сообществом формате список
с описанием изменений, проводимых с исходными данными и сборочными инструкциями пакета.

\marginalia{ex_sign_col}{В пакете rpm-utils присутствует утилита \Sys{add\_changelog}. Утилита добавит
заголовок для журнала изменений на основе данных в преамбуле \Sys{spec}-файла.}

Пример использования утилиты \Sys{add\_changelog}:
\begin{verbatim}
        $ add_changelog имя_спецификации.spec
\end{verbatim}

\section{Пример \Sys{.spec}-файла}

Представим образцы \Sys{spec}-файлов%
\footnote{\href{https://www.altlinux.org/SampleSpecs}{https://www.altlinux.org/SampleSpecs}}: шаблонный образец,
образец для программы со сборочной системой \Sys{autotools}, образец для программы со сборочной системой
\Sys{cmake} и образец модуля для \Sys{Python 3}.

\Emph{Пример 1.} Пустой \Sys{SPEC}
\begin{Verbatim}[breaklines=true,breakanywhere=true,fontsize=\scriptsize]
Name:    <имя-пакета>
Version: <версия-пакета>
Release: alt<релиз-пакета>
	
Summary: <однострочное описание>
License: <лицензия>
Group:   <группа>

Url:     <URL>
Source:  %name-%version.tar
Patch1:

PreReq:
Requires:
Provides:
Conflicts:

BuildPreReq:
BuildRequires:
%{?!_without_test:%{?!_disable_test:%{?!_without_check:%{?!_disable_check:BuildRequires: }}}}
BuildArch:

%description
<многострочное
описание>

%prep
%setup
%patch1 -p1

%build
%configure
%make_build

%install
%makeinstall_std

%check
%make_build check

%files
%_bindir/*
%_man1dir/*
%doc AUTHORS NEWS README

%changelog
* <дата> <ваше имя> <$login@altlinux.org> <версия_пакета>-<релиз_пакета>
- initial build for ALT Linux Sisyphus
\end{Verbatim}

%\newpage

\Emph{Пример 2.} Для программы на \Sys{autotools}
	
Название \Sys{GNU Autotools} обычно относится к программным пакетам \Sys{Autoconf},
\Sys{Automake}, \Sys{Libtool} и \Sys{Gnulib}. Вместе они составляют систему сборки
\Sys{GNU}. Этот \Sys{spec}-файл является примером пакета с программой.
\begin{Verbatim}[breaklines=true,breakanywhere=true,fontsize=\scriptsize]
Name:    sampleprog
Version: 1.0
Release: alt1
	
Summary: Sample program specfile
Summary(ru_RU.UTF-8): Пример spec-файла для программы
	
License: GPLv2+
Group: Development/Other
Url: http://www.altlinux.org/SampleSpecs/program
		
Source: %name-%version.tar
Patch0: %name-1.0-alt-makefile-fixes.patch
		
%description
This specfile is provided as sample specfile for packages with programs.
It contains most of usual tags and constructions used in such specfiles.
		
%description -l ru_RU.UTF-8
		
%prep
%setup
%patch0 -p1
	
%build
%configure
%make_build
		
%install
%makeinstall_std
%find_lang %name
		
%files -f %name.lang
%doc AUTHORS ChangeLog NEWS README THANKS TODO contrib/ manual/
%_bindir/*
%_man1dir/*
		
%changelog
* Sat Sep 33 3001 Sample Packager <sample@altlinux.org> 1.0-alt1
- initial build
\end{Verbatim}
	
\Emph{Пример 3.} Для программы на \Sys{cmake}
	
	\Emph{CMake} --- это кроссплатформенный инструмент с открытым исходным кодом, который использует
	независимые от компилятора и платформы файлы конфигурации. \Sys{CMake} позволяет создавать
	собственные файлы инструментов сборки, специфичных для вашего компилятора и платформы. \Sys{CMake}
	считается стандартом де-факто для сборки кода на \Sys{C++}.
\begin{Verbatim}[breaklines=true,breakanywhere=true,fontsize=\scriptsize]
Name:    sampleprog
Version: 1.0
Release: alt1
		
Summary: Sample program specfile
License: GPLv2+
Group:   Development/Other
		
Url: http://www.altlinux.org/SampleSpecs/cmakeprogram
Source: %name-%version.tar.bz2
		
BuildRequires(pre): cmake rpm-macros-cmake
		
%description
This specfile is provided as a sample specfile
for a package built with cmake.
		
%prep
%setup
		
%build
%cmake_insource
%make_build # VERBOSE=1
		
%install
%makeinstall_std
%find_lang %name
		
%files -f %name.lang
%doc AUTHORS ChangeLog NEWS README THANKS TODO contrib/ manual/
%_bindir/*
%_man1dir/*
		
%changelog
* Sat Jan 33 3001 Example Packager <example@altlinux.org> 1.0-alt1
- initial build
\end{Verbatim}
	
\Emph{Пример 4.} Модуль для \Sys{Python 3}
	
	Макросы для сборки модулей \Sys{python3} содержатся в пакете \Sys{rpm-build-python3} и
	аналогичны тем, что используются в ALT для \Sys{python}\footnote{\href{https://www.altlinux.org/Python3}{https://www.altlinux.org/Python3}}.
\begin{Verbatim}[breaklines=true,breakanywhere=true,fontsize=\scriptsize]
%define pypi_name @NAME@
%define mod_name %pypi_name
		
%def_with check
		
Name:    python3-module-%pypi_name
Version: 0.0.0
Release: alt1
Summary: @TEMPLATE@
License: MIT
Group: Development/Python3
Url: https://pypi.org/project/@NAME@
Vcs: @SOURCE_GIT@
BuildArch: noarch
Source: %name-%version.tar
Source1: %pyproject_deps_config_name
Patch: %name-%version-alt.patch
		
# mapping of PyPI name to distro name
Provides: python3-module-%{pep503_name %pypi_name} = %EVR
		
%pyproject_runtimedeps_metadata
BuildRequires(pre): rpm-build-pyproject
%pyproject_builddeps_build
%if_with check
%pyproject_builddeps_metadata
%endif
		
%description
@DESCR@
	
%prep
%setup
%autopatch -p1
%pyproject_deps_resync_build
%pyproject_deps_resync_metadata
		
%build
%pyproject_build
		
%install
%pyproject_install
		
%check
%pyproject_run_pytest
		
%files
%doc README.*
%python3_sitelibdir/%mod_name/
%python3_sitelibdir/%{pyproject_distinfo %pypi_name}
		
%changelog
\end{Verbatim}

\section{RPM макросы}
\Emph{Макрос RPM} --- это именованная переменная, которая напрямую подставляет текст в \Sys{spec}-файл
во время сборки \Sys{rpm}-пакета. Имена макросов начинаются с символа \%. Имена макросов --- это
сокращённые псевдонимы для часто используемых фрагментов текста.
	
Ниже приведены примеры макросов\footnote{\href{https://www.altlinux.org/Spec/\%D0\%9F\%D1\%80\%D0\%B5\%D0\%B4\%D0\%BE\%D0\%BF\%D1\%80\%D0\%B5\%D0\%B4\%D0\%B5\%D0\%BB\%D0\%B5\%D0\%BD\%D0\%BD\%D1\%8B\%D0\%B5_\%D0\%BC\%D0\%B0\%D0\%BA\%D1\%80\%D0\%BE\%D1\%81\%D1\%8B}{https://www.altlinux.org/Spec/Предопределённые\_макросы}}:
\begin{enumerate}
	\item \Emph{Пример макроса, содержащего значение.}
		Если во время сборки некоторым командам необходимо передать имя собираемого пакета,
		то можно передавать им макрос \Sys{\%name}. Во время сборки этот макрос подставляет
		имя пакета, объявленное в начале \Sys{spec}-файла:
\begin{verbatim}
        %define some_macro foo
        ....
        Name: bar-%some_macro
        .....
        %build
        %dune_build -p %some_macro
\end{verbatim}
	
	\item \Emph{Пример макроса с набором команд.}
	\Sys{\%cmake\_build} --- макрос, необходимый для сборки пакетов с помощью \Sys{cmake}.
		Он подставляет следующую последовательность команд:
\begin{verbatim}
        mkdir -p x86_64-alt-linux-gnu;
        cmake \
        -DCMAKE_SKIP_INSTALL_RPATH:BOOL=yes \
        -DCMAKE_C_FLAGS:STRING='-O2 -g' \
        -DCMAKE_CXX_FLAGS:STRING='-O2 -g' \
        -DCMAKE_Fortran_FLAGS:STRING='-O2 -g' \
        -DCMAKE_INSTALL_PREFIX=/usr \
        -DINCLUDE_INSTALL_DIR:PATH=/usr/include \
        -DLIB_INSTALL_DIR:PATH=/usr/lib64 \
        -DSYSCONF_INSTALL_DIR:PATH=/etc \
        -DSHARE_INSTALL_PREFIX:PATH=/usr/share \
        -DLIB_DESTINATION=lib64 \
        -DLIB_SUFFIX="64" \
        -S . -B "x86_64-alt-linux-gnu"
\end{verbatim}
\end{enumerate}

Преимущества использования макросов:
\begin{itemize}
	\item упрощение сборки;
	\item унификация \Sys{spec}-файлов;
	\item подбор шаблонов для создания \Sys{spec}-файлов;
	\item сокращение размера \Sys{spec}-файлов позволяет упростить отладку;
	\item использование макросов обеспечивает гибкость в настройке и конфигурации
		пакетов, позволяя быстро изменять параметры сборки.
\end{itemize}

Где объявлены \Sys{rpm}-макросы:
\begin{itemize}
	\item cтандартные макросы предопределены в пакете \Sys{rpmbuild (librpm)}.
		Информацию о них можно получить из файла \Sys{/usr/lib/rpm/macros} или выполнив
		команду: \Sys{rpm ----showrc};
	 \item макросы можно объявить самостоятельно, добавив в \Sys{spec}-файл;
	 \item макросы можно объявить в отдельных файлах.
	 	Команда \Sys{\%include} позволяет загрузить специальные файлы с объявленными макросами;
	 \item макросы, объявленные в файлах, поставляемые с другими пакетами.

	 Файлы с объявленными \Sys{rpm}-макросами хранятся в каталоге

	\Sys{/usr/lib/rpm/macros.d}.

	 Например, пакет \Sys{rpm-build-ruby} содержит готовые макросы для сборки пакетов с программами, написанными на языке \Sys{Ruby}.
	 Для того, чтобы использовать эти макросы, необходимо этот пакет добавить в зависимости:
	 \Sys{BuildRequires(pre): rpm-build-ruby}.
\end{itemize}

\marginalia{ex_sign_col}{Команда получения значения макроса: \Sys{rpm ----eval {<имя\_макроса>}}}

\marginalia{ex_sign_col}{Некоторые макросы могут быть вложенными.}

\section{Вопросы для самопроверки}

\begin{enumerate}
\item Какие инструменты (программы) нужны для сборки пакетов в <<Альт Платформа>>?
\item Какая утилита используется для непосредственной сборки бинарного пакета?
\item Верно ли что \Sys{hasher} обеспечивает воспроизводимую сборку в изолированной среде?
\item Каково внутреннее устройство \Sys{rpm}-пакета?
\item У вас есть пакет \Sys{admc-0.15.0-alt1.x86\_64.rpm}:
	\begin{enumerate}
		\item[а)] можно ли его установить на компьютер с процессором Байкал-М?
		\item[b)] а на компьютер с процессором Эльбрус 16С?
		\item[c)] какой пакет нужно взять, чтобы собрать тот же пакет под другую целевую архитектуру?
	\end{enumerate}
\item Какая структура каталогов необходима для сборки \Sys{rpm}-пакета?
\item Понятие \Sys{buildroot} в пространстве сборки \Sys{rpm}-пакетов, какое назначение?
\item Для чего служит \Sys{spec}-файл?
\item Из каких обязательных частей состоит \Sys{spec}-файл?
\item Какие зависимости можно называть <<для сборки>>, а какие <<времени исполнения>>?
	\begin{enumerate}
		\item[а)] В какой части \Sys{spec}-файла указывается имя пакета?
		\item[b)] В какой части \Sys{spec}-файла указывается лицензия под которой распространяется пакет?
		\item[c)] В какой части \Sys{spec}-файла указывается история изменений пакета?
	\end{enumerate}
\item Что такое \Sys{rpm}-макрос?
	\begin{enumerate}
		\item[а)] Какие преимущества даёт использование макросов?
		\item[b)] Как найти стандартные макросы?
		\item[c)] Где в системе находятся макросы, поставляемые сторонними пакетами?
	\end{enumerate}
\end{enumerate}
