\chapter{Основные команды пакетного менеджера}\label{basic-package-manager-commands}
Управлять пакетами можно из командной строки при помощи программы \Sys{rpm}, которая имеет следующий синтаксис:
\begin{verbatim}
	rpm [options]
\end{verbatim}

Пакетный менеджер \Sys{RPM} предоставляет базовые возможности для управления пакетами. Основной набор команд\footnote{\href{https://wiki.altlinux.ru/Команды_RPM}{https://wiki.altlinux.ru/Команды\_RPM}} позволяет установить, удалить, обновить пакеты, получить разнообразную информацию о самих пакетах и их содержимом:

\begin{itemize}
	\item \textbf{Информация о пакете}: \Sys{rpm -qi ИМЯ\_ПАКЕТА \dots}\\
	\Sys{rpm -qi} выводит подробную информацию о конкретном установленном пакете.
	
	\marginalia{ex_sign_col}{\Sys{rpm -qi} означает <<query information>> (запросить информацию). Но в случае \Sys{rpm -i <имя пакета>}, \Sys{-i} означает инструкцию \Sys{install} (установи).}
	
	\item \textbf{Просмотр установленных пакетов}: \Sys{rpm -qa}\\
	Эта команда выводит список всех установленных пакетов в системе.
	
	\marginalia{ex_sign_col}{\Sys{-a: All} (весь, все). \Sys{rpm -qa} означает <<query all>>.}
	
	\item \textbf{Проверка установки пакета в системе}: \Sys{rpm -q ИМЯ\_ПАКЕТА \dots}\\
	Эта команда проверяет установлен ли пакет в системе.
	
	\item \textbf{Проверка зависимостей пакета}: \Sys{rpm -qR ИМЯ\_ПАКЕТА \dots}\\
	 \Sys{rpm -qR} выводит список зависимостей (другие пакеты), необходимых для работы указанного пакета.
	 
	 \marginalia{ex_sign_col}{\Sys{-R: Requires} (нуждается). Например, \Sys{rpm -qR} означает <<query requires>> (запрос нуждается).}
	 
	\item \textbf{Проверка файла на принадлежность пакету}: \Sys{rpm -qf ФАЙЛ}\\
	Команда \Sys{rpm -qf} определяет, к какому пакету принадлежит указанный файл.
	
	\marginalia{ex_sign_col}{\Sys{-f: File} (файл). Например, \Sys{rpm -qf} означает <<query file>> (файл запроса).}
	
	\item \textbf{Просмотр файлов пакета}: \Sys{rpm -ql ИМЯ\_ПАКЕТА \dots}\\
	 \Sys{rpm -ql} выводит список всех файлов, содержащихся в установленном пакете.
	 
	\item \textbf{Установка пакета}: \Sys{rpm -i ФАЙЛ\_ПАКЕТА}\\
	Команда \Sys{rpm -i} используется для установки пакета из файла \Sys{.rpm}. Например, \Sys{rpm -i package.rpm} установит содержимое пакета в системе.
	
	\item \textbf{Удаление пакета}: \Sys{rpm -e ФАЙЛ\_ПАКЕТА}\\
	\Sys{rpm -e} удаляет установленный пакет. Например, \Sys{rpm -e package} удалит пакет с именем \Sys{package}.
	
	\item \textbf{Обновление пакета}: \Sys{rpm -U ФАЙЛ\_ПАКЕТА}\\
	Команда \Sys{rpm -U} обновляет пакет до новой версии, если он уже установлен.
	
	\item \textbf{Проверка целостности пакета}: \Sys{rpm -V ИМЯ\_ПАКЕТА|ФАЙЛ\_ПАКЕТА \dots}\\
	\Sys{rpm -V} проверяет целостность файлов в пакете, сравнивая их с информацией в базе данных \Sys{rpm}.
	Дополнительные ключи:
	\begin{itemize}
		\item \textbf{\Sys{-v}}: Verbose (подробно). Например, \Sys{rpm -qv} означает <<query verbose>> (подробный запрос) и используется для вывода более подробной информации о пакете. Подробный вывод существует не для всех ключей утилиты.
	\end{itemize}
	
	\marginalia{ex_sign_col}{Справку по ключам можно получить, набрав в консоли команду \Sys{rpm --help}}
	
\end{itemize}



\section{Просмотр недавно установленных пакетов}\label{view-recently-installed-packages}
Чтобы получить список последних установленных пакетов, введите следующую команду: 

\begin{verbatim}
rpm -qa --last|head	
\end{verbatim}

Вывод:
\begin{verbatim}
smplayer-23.6.0-alt2.10169.x86_64             Пт 01 дек 2023 12:45:39
qt5-wayland-5.15.10-alt1.x86_64               Пт 01 дек 2023 12:35:44
qt5-tools-5.15.10-alt2.x86_64                 Пт 01 дек 2023 12:35:44
qt5-sql-mysql-5.15.10-alt1.x86_64             Пт 01 дек 2023 12:35:44
qt5-dbus-5.15.10-alt2.x86_64                  Пт 01 дек 2023 12:35:44
libqt5-xmlpatterns-5.15.10-alt1.x86_64        Пт 01 дек 2023 12:35:44
libqt5-x11extras-5.15.10-alt1.x86_64          Пт 01 дек 2023 12:35:44
libqt5-webenginewidgets-5.15.15-alt1.x86_64   Пт 01 дек 2023 12:35:44
libqt5-test-5.15.10-alt1.x86_64               Пт 01 дек 2023 12:35:44
libqt5-quickparticles-5.15.10-alt1.x86_64     Пт 01 дек 2023 12:35:44
\end{verbatim}

Команда \Sys{rpm -qa --last} используется для вывода списка всех установленных пакетов, отсортированных 
по времени их установки. Пакеты будут отсортированы в порядке убывания времени установки --- самые 
последние установленные пакеты отобразятся в верхней части списка.

Фильтрация вывода: утилита \Sys{grep} отфильтрует вывод и поможет найти искомый пакет. Например, 
следующая команда выведет информацию только о тех пакетах, название которых содержит <<\Sys{kernel}>>:

\begin{verbatim}
rpm -qa --last | grep kernel
\end{verbatim} 
