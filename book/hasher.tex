\hypertarget{5}{\chapter{Инструмент Hasher}}
\Emph{Hasher} --- инструмент для сборки пакетов с использованием минимальной контролируемой 
среды в \Sys{chroot}. \Sys{Hasher} опирается на системный вызов \Sys{chroot} и создаёт 
изолированную среду для сборки отдельного пакета утилитой \Sys{rpm-build}.

\Sys{Hasher} облегчает поддержание сборочных зависимостей и позволяет собирать пакеты для 
разных дистрибутивов. \Sys{Hasher} проверяет пакеты с помощью утилиты \Sys{sisyphus\_check} 
и создаёт локальный \Sys{APT}-репозиторий с результатами сборки.

Для подготовки сборочного окружения \Sys{hasher} берёт пакеты, как из удалённых репозиториев, 
настроенных в основной системе, так и из локального репозитория (\Sys{$\sim$/hasher/repo/}), 
в который попадают ранее собранные пакеты.

Так же \Sys{hasher} удобен для отладки процесса сборки. Если сборка пакета прервалась, 
выполните команду \Sys{hsh-shell}, чтобы попасть в терминал \Sys{chroot}, исправьте ошибку 
и продолжайте сборку с прерванного этапа.


\section{Описание системы Hasher}
Опишем структуру каталогов в \Sys{hasher}.

\begin{itemize}
	\item \Sys{hasher}
	\begin{itemize}
		\item \Sys{chroot} --- само минимальное окружение. В этом каталоге находится 
			корневое дерево содержащее минимальный набор пакетов, \Sys{rpmbuild} 
			и сборочные зависимости.
		\item \Sys{aptbox} --- набор утилит для установки, обновления и удаления 
			пакетов \Sys{chroot}. Например, тут лежит модифицированный \Sys{apt-get}, 
			с помощью которого происходит установка пакетов в \Sys{chroot}.
		\item \Sys{cache} --- в этом каталоге кэшируются файлы, необходимые для 
			создания \Sys{chroot}.
		\begin{itemize}
			\item каталог \Sys{$\sim$/hasher/chroot/usr/src/RPM}, который содержит 
				подкаталоги для \Sys{rpmbuild}:
			\begin{itemize}
				\item \Sys{BUILD}
				\item \Sys{RPMS}
				\item \Sys{SOURCES}
				\item \Sys{SPECS}
				\item \Sys{SRPMS}
			\end{itemize} 
		\end{itemize}
		\item \Sys{repo}, который содержит подкаталоги:
		\begin{itemize}
			\item \Sys{SRPMS.hasher} --- исходники (sources) пакета.
			\item \Sys{<платформа>/RPMS.hasher/} --- каталог с пакетами, собранными 
				под конкретную архитектуру.
			В этот каталог можно добавить необходимые сборочные зависимости, когда 
				пакета ещё нет в репозитории и оценить сборку.
		\end{itemize}
	\end{itemize}
	\item \Sys{$\sim$/.hasher}
	\begin{itemize}
		\item \Sys{apt.config} --- конфигурация для \Sys{apt-get} из \Sys{$\sim$/hasher/aptbox/}
		\item \Sys{config} --- конфигурация самого \Sys{hasher}
	\end{itemize}
	\item \Sys{/etc/hasher-priv/} каталог с конфигурацией для вспомогательной утилиты \Sys{hasher-priv}
	\begin{itemize}
		\item \Sys{./user.d}
		\item \Sys{fstab} --- информация о точках монтирования вспомогательной программы \Sys{hasher-priv}
		\Sys{system} --- конфигурация вспомогательной программы \Sys{hasher-priv}
	\end{itemize} 
\end{itemize}


\section{Принцип действия}
В каталоге \Sys{$\sim$/hasher/chroot} создаётся сборочная среда. В неё устанавливается минимальный набор 
пакетов, \Sys{rpmbuild} и сборочные зависимости. Вся работа, связанная непосредственно со сборкой пакетов, 
происходит в этом \Sys{chroot}. \Sys{Hasher} не использует одну и ту же созданную среду. У каждого пакета 
свои зависимости и поэтому для каждого пакета его изолированное сборочное окружение будет уникальным. Для 
этого в \Sys{Hasher} заложено создание новой сборочной среды под каждый собираемый пакет. При успешном 
завершении сборки каталог \Sys{chroot} очищается%
%\footnote{\href{https://alt-packaging-guide.github.io/\#_hasher_start}{https://alt-packaging-guide.github.io/\#\_hasher\_start}}.

Сборочное окружение можно сохранить. Для этого используют флаг:\\ \Sys{--lazy-cleanup}.

Сборочное окружение каждый раз удаляется при инициализации нового \Sys{chroot}.

Из принципа действия \Sys{Hasher} следует:
\begin{itemize}
	\item cборка не пройдёт корректно, если в пакете не указаны все зависимости;
	\item сборку в \Sys{hasher} можно воспроизвести на другом компьютере, поскольку \Sys{hasher} 
		не использует индивидуальные пользовательские настройки;
	\item в настройках можно определить, для какого репозитория будет проходить сборка. Поэтому не 
		меняя пользовательское окружение, можно собрать пакет под разные дистрибутивы.
\end{itemize}


\hypertarget{5.3}{\section{Настройка Hasher}}
В разделе \hyperlink{1.3}{<<Установка необходимых пакетов для процесса сборки>>}  упоминалась 
установка пакета \Sys{hasher}. Для отдельной установки используйте команду: 
\begin{verbatim}
	# apt-get install hasher
\end{verbatim}

Добавьте пользователя в группу \Sys{rpm}:
\begin{itemize}
	\item для локального пользователя:
	\begin{verbatim}
		# usermod -aG rpm <user>
	\end{verbatim}
	
	\item для доменного пользователя: 
	\begin{verbatim}
		# gpasswd -a <username> rpm
	\end{verbatim}
\end{itemize} 

Запустите демон \Sys{hasher-priv}:
\begin{verbatim}
	# systemctl enable --now hasher-privd.service
\end{verbatim} 

Укажите учётную запись для работы с \Sys{hasher}. Добавьте учётные записи для 
работы с \Sys{hasher} в группу \Sys{hashman}:
\begin{verbatim}
	# hasher-useradd <user>
\end{verbatim} 

Заново авторизуйтесь, чтобы изменения вступили в силу. 

Укажите расположение сборочной среды (каталоги настроек и их инициализации):
\begin{verbatim}
	$ mkdir ~/hasher
	$ mkdir ~/.hasher
\end{verbatim}}

При создании каталога \Sys{.hasher} следует учитывать два правила: 
\begin{enumerate}
	\item права доступа соответствуют \Sys{drwxr-xr-x}, то есть каталог доступен на запись;
	\item на файловой системе, смонтированной с \Sys{noexec} или \Sys{nodev}, каталог располагать нельзя%
		\footnote{\href{https://serverfault.com/questions/547237/explanation-of-nodev-and-nosuid-in-fstab}{
			https://serverfault.com/questions/547237/explanation-of-nodev-and-nosuid-in-fstab}};
	\begin{itemize}
		\item \Sys{noexec} устанавливается, если в системе есть файлы с правами \Sys{rwsr-xr-x} 
			(запустить исполняемые файлы с правами владельца или группы) и владельцем \Sys{root}. 
			Запустить файл \Sys{rwxr-xr-x} на такой файловой системе невозможно. а следовательно, 
			и создать корректное сборочное окружение. \Sys{Hasher} не зависит от пользовательского окружения.
		\item \Sys{nodev} говорит о том, что на файловой системе не будут созданы файлы устройств. 
			Это не соответствует функциональности \Sys{hasher} (см.~раздел %6.7 
			\hyperlink{mount_fs_hasher}{<<Монтирование файловых систем внутри \Sys{Hasher}>>}). 
	\end{itemize} 
\end{enumerate}

Создайте конфигурационный файл: 
\begin{verbatim}
	packager="`rpm --eval %packager`"
	no_sisyphus_check="packager,buildhost,gpg"
	allowed_mountpoints=/dev/pts,/proc,/dev/shm
	lazy_cleanup=1
	# share_ipc=1
	# Необходимо в случае, когда требуется разрешить доступ в сеть 
	# из Hasher chroot.
	# Данная опция не должна применяться без особой необходимости.
	# share_network=1
	# нужен для сборки с репозиториями отличающимися от системного:
	# apt_config="$HOME/.hasher/apt.conf"
\end{verbatim}}

Создадим сборочное окружение: 
\begin{verbatim}
	$ hsh --initroot-only 
	# путь ~/hasher является параметром по умолчанию. 
	# Полностью команда выглядит так:
	$ hsh --initroot-only ~/hasher
	$ ls -al ~/hasher
\end{verbatim}

Если сборочное окружение не создано, оно построится при первой сборке пакета. 

Чем больше пакет, тем больше времени понадобится на его сборку. 
В конфигурационном файле \Sys{hasher-priv} прописываются ограничения на ресурсы, 
выделяемые на сборку (CPU, память, общее время исполнения и другие). Чтобы 
увеличить время бездействия утилиты \Sys{hasher-priv}, требуется отредактировать 
файл \Sys{/etc/hasher-priv/system/}:
\begin{verbatim}
	wlimit_time_elapsed=360000
	wlimit_time_idle=360000
\end{verbatim} 

Сборочная среда строится по умолчанию на источниках для основной системы 
(пользователя/хост-системы).


\section{Сборка в Hasher}
Сборка в \Sys{hasher} предполагает два возможных сценария%
\footnote{\href{https://www.basealt.ru/fileadmin/user_upload/manual/Alt_Server_V_life_cicle_p10.pdf}{https://www.basealt.ru/fileadmin/user\_upload/manual/Alt\_Server\_V\_life\_cicle\_p10.pdf}}: 
\begin{enumerate}
	\item \Emph{Сборка из \Sys{.src.rpm}}.
	
		Этот сценарий подходит, если у нас уже есть \Sys{.src.rpm} (см.~раздел %3.1 
		\hyperlink{3.1}{\mbox{<<Описание \Sys{RPM}-пакета>>}}).
	
	Команда сборки выполняется от обычного пользователя, так как он добавлен в группу \Sys{hashman} 
		(см.~раздел %~5.3 
		\hyperlink{5.3}{<<Настройка \Sys{Hasher}>>}): 
	\begin{verbatim}
		hsh /home/work/rpm/package.src.rpm
	\end{verbatim}
	Собранный пакет появляется в директории:\\ \Sys{$\sim$/.hasher/repo/<платформа>/RPMS.hasher/}. 
	
	Если сборка завершилась неудачей, информация об ошибках будет выведена в терминале. Следует ознакомиться 
		с выводом в терминале для понимания, на каком этапе сборки возникают ошибки, и на основе этого 
		вносить необходимые исправления в \Sys{SPEC}-файл.
	
	\item  \Emph{Сборка в связке с \Sys{GEAR}}.
	
	Этот сценарий подходит, если у нас есть готовый \Sys{gear}-репозиторий (см. глава~5 
		\hyperlink{4}{<<Инструмент Gear>>}).
	
	 Команда выполняется внутри \Sys{gear}-репозитория и выглядит следующим образом:
	 \begin{verbatim}
	 	gear-hsh
	 \end{verbatim}
	 
	 Эта команда часто используется с параметрами: 
	 \begin{itemize}
	 	\item \Sys{-v} --- сделать подробный вывод;
	 	\item \Sys{--no-sisyphus-check} отключить проверки.
	 \end{itemize}
	 Так же в строке запуска \Sys{gear-hsh} можно передавать параметры для самого \Sys{hasher}. 
\end{enumerate}

В обоих сценариях при успешном выполнении сборки новые пакеты будут находиться в каталоге 
\Sys{$\sim$/hasher/repo/<архитектура>}. 


\section{Сборочные зависимости}
\Emph{Сборочные зависимости (build dependencies)} --- необходимые для сборки пакета другие пакеты. 
Также выделяют зависимости исполнения (runtime dependencies). При этом некоторые сборочные зависимости 
могут быть необходимы и во время выполнения программы. 

\Emph{Зависимость} --- это ресурс, необходимый программному обеспечению, но не поставляемый вместе с ним. 
Например, программы, библиотеки, утилиты, необходимые во время работы программы, скрипты и 
программа-интерпретатор для их запуска, файлы, необходимые для корректной работы программы. 

При сборке \Sys{rpm}-пакета в пакетной базе \Sys{Sisyphus} могут не оказаться зависимые для него пакеты. 
Такие пакеты придётся собрать и только после этого заняться сборкой интересующего пакета. 

\Emph{Как искать зависимости}

Чаще всего зависимости уже прописаны в спецификации к пакету.

Если пакет новый, попробуйте составить начальный \Sys{SPEC}-файл и собрать пакет. Проанализируйте 
вывод процесса сборки и отметьте, чего не хватает. Часто сборочные зависимости указываются в 
документации к собираемому приложению --- добавляйте их по необходимости.

Виды зависимостей можно обозначить по названию тегов \Sys{SPEC}-файла для пакета и порядку их употребления: 
\begin{itemize}
	\item \Sys{BuildRequires(pre)} --- пакеты, которые содержат определения макросов, указанных 
		в \Sys{SPEC}-файле. Они помогают создать исходный \Sys{src.rpm} из \Sys{SPEC}-файла; 
		тег не содержит макросы;
	\item \Sys{BuildRequires} --- пакеты, необходимые для сборки.
\end{itemize}

Установите в системе пакеты-зависимости первого вида (\Sys{BuildRequires (pre)}), так как \Sys{hasher} 
собирает пакеты из \Sys{src.rpm}. Название большинства из таких пакетов начинается с \Sys{rpm-build-*}.

\Emph{Установка зависимостей}
\begin{itemize}
	\item локально (\Sys{gear-rpm -ba}) --- необходимо вручную установить пакеты;
	\item в \Sys{hasher'е} (\Sys{gear-hsh}) --- \Sys{hasher} сам установит в контейнер все 
		необходимое на основе \Sys{SPEC}-файла из тегов \Sys{BuildRequires (pre)} и 
		\Sys{BuildRequires}.
\end{itemize} 

Когда в базе данных менеджера пакетов отсутствуют необходимые зависимости, протестировать сборку пакета 
можно с помощью флага \Sys{--nodeps}. Флаг \Sys{--nodeps} даёт команду пропустить проверку зависимостей. 
Использовать эту опцию нужно крайне аккуратно%
\footnote{\href{https://www.endpointdev.com/blog/2008/08/rpm-nodeps-really-disables-all/}{https://www.endpointdev.com/blog/2008/08/rpm-nodeps-really-disables-all/}}. 
Флаг \Sys{--nodeps} отключает всю систему отслеживания \Sys{RPM} --- нарушается порядок установки зависимостей 
и назначение владельца и группы.

\begin{verbatim}
	$ rpm -bs --nodeps package.spec
\end{verbatim} 

\section{Справочная страница Hasher}
Возможности \Sys{Hasher} задокументированы в инструкциях%
\footnote{\href{http://uneex.ru/static/AltlinuxOrg_Hasher/}{http://uneex.ru/static/AltlinuxOrg\_Hasher/}} 
к пакетам \Sys{hasher} и \Sys{hasher-priv}. Для вызова справочной информации по \Sys{hasher} наберите 
в консоли \Sys{man <package>}:
\begin{verbatim}
	$ man hsh
	
	HASHER(7)                          ALT Linux                         HASHER(7)
	
	NAME
	hasher - modern safe package building technology
	
	SYNOPSIS
	hsh [options] <path-to-workdir> <package>...
	...
\end{verbatim} 

\begin{verbatim}
	$ man hasher-priv
	
	HASHER-PRIV(8)          System Administration Utilities         HASHER-PRIV(8)
	
	NAME
	hasher-priv - privileged helper for the hasher project
	
	SYNOPSIS
	hasher-priv [options] <args>
	...
\end{verbatim}

Это достаточно подробная инструкция по использованию утилиты \Sys{hasher}. Здесь вы найдёте 
полное описание, опции и флаги, содержимое рабочего каталога. 


\hypertarget{mount_fs_hasher}{\section{Монтирование файловых систем внутри Hasher}}
\Sys{Hasher} умеет монтировать внутрь \Sys{hasher}-контейнера точки монтирования и виртуальные файловые системы 
из основной машины%
\footnote{\href{https://www.altlinux.org/Hasher/\%D0\%A0\%D1\%83\%D0\%BA\%D0\%BE\%D0\%B2\%D0\%BE\%D0\%B4\%D1\%81\%D1\%82\%D0\%B2\%D0\%BE\#\%D0\%9C\%D0\%BE\%D0\%BD\%D1\%82\%D0\%B8\%D1\%80\%D0\%BE\%D0\%B2\%D0\%B0\%D0\%BD\%D0\%B8\%D0\%B5_\%D1\%84\%D0\%B0\%D0\%B9\%D0\%BB\%D0\%BE\%D0\%B2\%D1\%8B\%D1\%85_\%D1\%81\%D0\%B8\%D1\%81\%D1\%82\%D0\%B5\%D0\%BC_\%D0\%B2\%D0\%BD\%D1\%83\%D1\%82\%D1\%80\%D0\%B8_hasher}{https://www.altlinux.org/Hasher/Руководство}}. 
Этот механизм применяется в тех случаях, когда собираемому приложению для сборки требуется доступ к ресурсам 
основной машины, которые \Sys{Hasher} не предоставляет по умолчанию. Например, виртуальная файловая система 
\Sys{/proc} или \Sys{/dev/pts}, которых по умолчанию нет в \Sys{hasher}-контейнере. Файловая система \Sys{/proc} 
получает информацию о состоянии и конфигурации ядра и системы.

Для монтирования файловой системы следует:
\begin{enumerate}
	\item В файле \Sys{/etc/hasher-priv/fstab} описать файловую систему.
	\item В файле \Sys{/etc/hasher-priv/system} указать файловую систему с помощью опции \Sys{allowed\_mointpoints}.
	\item Указать файловую систему либо при запуске \Sys{Hasher} в опции \Sys{--mountpoints}, либо в 
		конфигурационном файле \Sys{$\sim$/.hasher/config} в ключе \Sys{known\_mount\-points}.
	\item Прописать необходимую файловую систему в \Sys{SPEC}-файле в теге \Sys{BuildReq}, либо в списке зависимостей.
\end{enumerate}

\section{Вопросы для самопроверки}

\begin{enumerate}
	\item Что такое \Sys{hasher} и для чего он предназначен? 
	\item Какоыа структура каталогов \Sys{hasher}?
        \item Почему ранее собранные в \Sys{hasher} пакеты не оказывают влияния на новую сборку?
	\item Можно ли указать из какого репозитория будет происходить сборка в \Sys{hasher}?
	\item Какие шаги, помимо установки, необходимо сделать для настройки \Sys{hasher}?
	\item Какие два сценария сборки возможны при использовании \Sys{hasher}?
        \item Что такое зависимость программного пакета?
	\item Какие зависимости называют <<сборочными>>, а какие <<времени исполнения>>?
        \item Как посмотреть справочную информацию по пакету \Sys{hasher} и \Sys{hasher-priv}?
	\item Зачем монтировать внутрь \Sys{hasher} файловые системы? 
\end{enumerate}

