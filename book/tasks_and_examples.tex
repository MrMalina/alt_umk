\chapter{Примеры и упражнения}

В главе разобраны подготовленные примеры для иллюстрирования работы инструментов.
Примеры подготовлены для популярной архитектуры x86\_64 и подходят для других
процессорных архитектур. Различия будут содержаться в путях где упоминается имя
используемой архитектуры.

\section{Базовые инструменты \Sys{rpm} \textbf{Пример №1}}

Рассмотрим пример сборки пакета средствами \Sys{rpmbuild} и использование
некоторых команд утилиты \Sys{rpm}. В сети доступен простой пример \Sys{C++}
программы с инструкциями для сборки.
Если окружение сборки \Sys{RPM} не настроено, выполните следующие команды:
\begin{verbatim}
# apt-get update
# apt-get install rpmdevtools rpm-build gcc-c++ git-core
$ rpmdev-setuptree
\end{verbatim}

Скопируем необходимую версию примера проекта и перейдем
в рабочий каталог. Для однозначности точно укажем расположение рабочего
каталога \Sys{~/ExampleFirstProject/} в команде \Sys{git}:
\begin{verbatim}
$ git clone ----branch=rpmbuild-v1 \
 https://gitlab.basealt.space/alt/ExampleFirstProject.git ~/ExampleFirstProject/
$ cd ~/ExampleFirstProject/
\end{verbatim}

Сформируем \Sys{.tar} архив с исходными данными проекта. Имя архива должно
соответствовать инструкциям в \Sys{.spec} файле:
\begin{verbatim}
$ tar cvf HelloUniverse-1.0.tar HelloUniverse
\end{verbatim}

Расположим архив и \Sys{.spec}-файл в соответствии с правилами структуры каталогов
\Sys{RPM}:
\begin{verbatim}
$ cp HelloUniverse-1.0.tar ~/RPM/SOURCES/
$ cp HelloUniverse.spec ~/RPM/SPECS/
\end{verbatim}

Запустим сборку пакета с ключом \Sys{-ba} и передадим инструкции для сборки
в структуре каталогов \Sys{RPM}.
\begin{verbatim}
$ rpmbuild -ba ~/RPM/SPECS/HelloUniverse.spec
\end{verbatim}

В ходе выполнения команда скомпилирует исходные данные, соберет
в бинарный \Sys{.rpm}-пакет и создаст \Sys{src.rpm}-пакет с исходными данными.

Проверим наличие и внутреннюю информацию собранного бинарного пакета:
\begin{verbatim}
$ rpm -qpi ~/RPM/RPMS/x86_64/HelloUniverse-1.0-alt1.x86_64.rpm
\end{verbatim}

Убедившись в наличии пакета, установим его в систему:
\begin{verbatim}
$ sudo rpm -i ~/RPM/RPMS/x86_64/HelloUniverse-1.0-alt1.x86_64.rpm
\end{verbatim}

Теперь можно запустить исполняемый файл пакета и убедиться в его
работоспособности:
\begin{verbatim}
$ HelloUniverse
\end{verbatim}

К установленному в систему пакету можно обращаться по имени, заданному в
метаданных. Посмотрим информацию о пакете, список файлов пакета и удалим
пакет из системы:
\begin{verbatim}
$ sudo rpm -qi HelloUniverse
$ rpm -ql HelloUniverse
$ sudo rpm -e HelloUniverse
\end{verbatim}

\section{Hasher \textbf{Пример №2}}

Переходя к примеру №2, предполагается, что инструмент \Sys{Hasher} настроен,
как описано в главе \hyperlink{5}{Инструмент \Sys{Hasher}}.

\Sys{Hasher} позволяет собирать пакеты в изолированном окружении по
переданным специально подготовленным архивам \Sys{pkg.tar} и пакетам
исходных данных \Sys{.src.rpm}. Он так же позволяет работать в ручном
режиме. Продемонстрируем все перечисленные варианты работы с
инструментом \Sys{Hasher}.\\

\textbf{Пример №2.1 <<Ручная сборка в изолированном окружении>>}:\\

Используем \textbf{Пример №1} и соберем пакет  в ручном режиме
для знакомства с базовыми командами \Sys{Hasher}. Работу начнем в
каталоге \Sys{~/ExampleFirstProject/}.

Создадим изолированное сборочное окружение \Sys{Hasher}:

\begin{verbatim}
$ hsh ----initroot-only ~/hasher/
\end{verbatim}

В сборочное окружение необходимые зависимости нужно установить в ручную:
\begin{verbatim}
$ hsh-install gcc-c++
\end{verbatim}

Уже имеющийся архив и \Sys{.spec}-файл поместим в изолированное сборочное
окружение через служебные каталоги \Sys{~/hasher/.in}:

\begin{verbatim}
$ cp HelloUniverse-1.0.tar ~/hasher/chroot/.in/
$ cp HelloUniverse.spec  ~/hasher/chroot/.in/
\end{verbatim}

Войдем в сборочное окружение \Sys{Hasher}:
\begin{verbatim}
$ hsh-shell
\end{verbatim}

В окружение \Sys{Hasher} изначально мы окажемся в каталоге \Sys{/.in}.
Скопируем архив и \Sys{.spec}-файл в уже подготовленные каталоги \Sys{RPM/}:
\begin{verbatim}
$  cp HelloUniverse-1.0.tar /usr/src/RPM/SOURCES/
$  cp HelloUniverse.spec /usr/src/RPM/SPECS/
\end{verbatim}

Запустим сборку:
\begin{verbatim}
$  rpmbuild -bb /usr/src/RPM/SPECS/HelloUniverse.spec
\end{verbatim}

Полученный результат будет размещен в сборочной структуре \Sys{RPM}
в каталоге \Sys{ /usr/src/RPM/RPMS/}. Выйти из сборочного окружения можно
нажав комбинацию клавиш \Sys{Ctrl+d} или исполнив команду:
\begin{verbatim}
$ logout
\end{verbatim}

\textbf{Пример №2.2 <<Сборка с использованием \Sys{.src.rpm}-файла>>}:\\

Пример предполагает наличие пакета с исходными данными \Sys{.src.rpm} и
позволяет в автоматическом режиме запустить сборку в изолированном окружении
\Sys{Hasher}.

Используем полученный в примере №1 \Sys{.src.rpm}-файл и передадим на сборку:
\begin{verbatim}
$ hsh ~/RPM/SRPMS/HelloUniverse-1.0-alt1.src.rpm
\end{verbatim}

\Sys{Hasher} распакует пакет, установит все необходимые зависимости, исполнит
сборочные инструкции, проведет проверки и положит результат в каталог
\Sys{~/hasher/repo/}. Собранный в примере пакет можно найти в каталоге:
\begin{verbatim}
$ ls ~/hasher/repo/x86_64/RPMS.hasher/HelloUniverse-1.0-alt1.x86_64.rpm
\end{verbatim}

\textbf{Пример №2.3 <<Сборка с использованием \Sys{pkg.tar}-архива>>}:

В примере создадим архив \Sys{pkg.tar} в упрощенном виде . Такой архив с более строгими
параметрами создает инструмент \Sys{GEAR}. Этот архив можно передать на сборку в
\Sys{Hasher}.

Используем сформированный архив с исходными данными и \Sys{.spec}-файл, созданные
в примере №1, и  упакуем в архив с указанными параметрами.

Перейдем в рабочий каталог и запустим команду архивирования:
\begin{verbatim}
$ cd ~/ExampleFirstProject/
$ tar -c ----file=pkg.tar ----label=HelloUniverse.spec HelloUniverse-1.0.tar HelloUniverse.spec
\end{verbatim}
Использованные ключи команды \Sys{tar}:
\begin{itemize}
	\item \Sys{-c} --- создать архив;
	\item \Sys{--file=pkg.tar} --- задать имя архива(шаблонное имя архива);
	\item \Sys{--label=HelloUniverse.spec} --- метка архива(должна совпадать с именем \Sys{.spec}-файла).
\end{itemize}

Созданный архив можно передать на сборку в изолированное сборочное
окружение \Sys{Hasher}:
\begin{verbatim}
$ hsh ./pkg.tar
\end{verbatim}

Проверим наличее собранного пакета:
\begin{verbatim}
$ ls ~/hasher/repo/x86_64/RPMS.hasher/HelloUniverse-1.0-alt1.x86_64.rpm
\end{verbatim}

\section{GEAR \textbf{Пример №3}}

Примеры №2 предполагают большое количество шагов подготовки исходных данных.
Все шаги подготовки выполняются <<вручную>>. Продемонстрируем пример работы
инструмента \Sys{GEAR} с набором формализованных инструкций.

\begin{verbatim}
$ git clone ----branch=gear-v2 \
 https://gitlab.basealt.space/alt/ExampleFirstProject.git ~/ExampleFirstProject-v2/
$ cd ~/ExampleFirstProject-v2/
\end{verbatim}

Загруженный проект дополнен  инструкциями \Sys{.gear/rules}. По данным инструкциям,
\Sys{GEAR} сам подготовит исходные данные для сборки и сформирует архив.
\begin{verbatim}
$ gear pkg.tar
\end{verbatim}

Продемонстрируем как \Sys{GEAR} работает в связке с инструментами \Sys{rpmbuild}
и \Sys{Hasher}.

\textbf{Пример №3.1 <<Сборка с \Sys{rpmbuild}>>}:
Выполним сборку с использованием \Sys{gear-rpm} и ключем \Sys{-bb}
для сборки только бинарного {.rpm}-пакета:
\begin{verbatim}
$ gear-rpm -bb
\end{verbatim}

\Sys{GEAR} сформирует архив с исходными данными, архив и \Sys{.spec}-файл разложит в
соответствии с определенной структурой и выполнит команду сборки в необходимом формате.
Собранный пакет попадет в каталог \Sys{~/RPM/RPMS/}. Отобразим информацию
о только что собранном пакете:
\begin{verbatim}
$ rpm -qi ~/RPM/RPMS/x86_64/HelloUniverse-2.0-alt1.x86_64.rpm
\end{verbatim}

\textbf{Пример №3.2 <<Сборка с \Sys{Hasher}>>}:
Запустим сборку пакета с использованием \Sys{gear-hsh}. Передадим ключи вида
\Sys{----verbose ---- ----verbose} для подробного вывода информации о процессе выполнения
как для \Sys{gear}, так и для вложенного \Sys{Hasher} инструментов:
\begin{verbatim}
gear-hsh ----verbose ---- ----verbose
\end{verbatim}

\Sys{GEAR} подготовит исходные данные и передаст в необходимом формате на сборку в
изолированное сборочное окружение \Sys{Hasher}.
Собранный в примере пакет можно найти в каталоге:
\begin{verbatim}
$ ls ~/hasher/repo/x86_64/RPMS.hasher/HelloUniverse-2.0-alt1.x86_64.rpm
\end{verbatim}

\textbf{Пример №3.3 <<Сборка по тэгу релиза с использованием автогенерируемого патча>>}:

Продемонстрируем сборку пакета по метке, связанной с ключевыми изменениями исходных
данных. В нашем примере возьмем метку \Sys{v2.0} и дополнительно внесем собственные
изменения с применением механизма \Sys{GEAR} для авто генерации патча.

Получим подготовленные примеры инструкций \Sys{.spec}-файла и \Sys{.gear/rules}:

\begin{verbatim}
cd ~/ExampleFirstProject-v2/
git pull origin 2.0-alt2
\end{verbatim}

Изменения в инструкциях позволяют собрать пакет с содержимым, связанным с меткой
\Sys{v2.0} в истории \Sys{git}-репозитория. Но мы дополнительно внесем изменения
в исходные данные нашего проекта и применим их средствами автогенерируемго патча.

Удобным для вас текстовым редактором измените в файле:\\
\Sys{HelloUniverse/HelloUniverse.cpp}\\
слово \Sys{New} на \Sys{Alt} или выполните команду:
\begin{verbatim}
$ sed -i '4 s/New/Alt/g' HelloUniverse/HelloUniverse.cpp
\end{verbatim}

Нужные изменения подготовлены, но не зафиксированы в истории \Sys{git}. Несмотря на это,
мы можем применить изменения для локальной сборки, передав \Sys{GEAR} ключ \Sys{--commit}.
Соберем пакет в изолированном окружении с  дополнительным ключом \Sys{--lazy-cleanup} для
сохранения файлов после окончания сборки.
Запустим сборку с перечисленными ключами:
\begin{verbatim}
$ gear-hsh ----commit ----lazy-cleanup
\end{verbatim}

После завешения сборки ознакомимся с автоматически созданным и примененным патчем:
\begin{verbatim}
$ cat ~/hasher/chroot/usr/src/RPM/SOURCES/HelloUniverse-2.0-alt2.patch
\end{verbatim}

Установим  собранный пакет пакетным менеджером \Sys{apt}:
\begin{verbatim}
$ sudo apt-get install ~/hasher/repo/x86_64/RPMS.hasher/HelloUniverse-2.0-alt2.x86_64.rpm
\end{verbatim}

Запустим исполняемый файл пакета:
\begin{verbatim}
$ HelloUniverse
\end{verbatim}

Можем удалить пакет, используя пакетный менеджер \Sys{apt}:
\begin{verbatim}
$ sudo apt-get remove HelloUniverse
\end{verbatim}

\section{GEAR обновление пакета \textbf{Пример №4}}

В примере будет рассмотрена упрощенная процедура обновления пакета с
использованием инструментов \Sys{Alt}.

Чтобы продолжить работу, необходимо добавить набор утилит \Sys{GEAR} для
настройки и взаимодействия с удаленными репозиториями:
\begin{verbatim}
# apt-get install gear-remotes-utils
\end{verbatim}

Для удобства установим набор утилит \Sys{rpm}:
\begin{verbatim}
# apt-get install rpm-utils
\end{verbatim}

Пример упрощен и объединяет в себе внешний(upstream) источник и
рабочий источник примера. Скопируем версию примера, подготовленную для
обновления:
\begin{verbatim}
$ git clone ----branch=2.0-alt2 \
https://gitlab.basealt.space/alt/ExampleFirstProject.git ~/ExampleFirstProject-v3/
$ cd ~/ExampleFirstProject-v3/
\end{verbatim}

Назначим наш базовый-рабочий источник(origin) внешним(upstream). Такой вариант
используется для примера и в обычной рабочей ситуации подразумевает различие источников.
\begin{verbatim}
gear-remotes-save origin
\end{verbatim}

Теперь возможно обращаться к нашему источнику примера как к \Sys{upstream}.
Притянем новые изменения и проверим новые метки изменений:
\begin{verbatim}
git fetch upstream
gear-remotes-watch
\end{verbatim}

Нас интересует метка \Sys{v3.0}. Теперь притянем изменения в рабочее файловое дерево:
\begin{verbatim}
git merge v3.0
\end{verbatim}

Специализированная утилита поможет легко исправить инструкции
\Sys{.gear/tags/list}.
Обновим инструкции для сборки исходных данных с меткой \Sys{v3.0}:
\begin{verbatim}
gear-update-tag v3.0
\end{verbatim}

Удобным для вас текстовым редактором необходимо внести изменения в версию и релиз
в \Sys{.spec}-файле. Пример изменений:
\begin{verbatim}
Name:    HelloUniverse
Version: 3.0
Release: alt1
Summary: Most simple RPM package
\end{verbatim}
Все остальные поля остаются неизменными.

Утилита \Sys{add\_changelog} дополняет секцию журнала \Sys{.spec}-файла
списком изменений в утвержденном формате(каждую строку списка принято начинать
с символа ''\Sys{-}''). Создадим запись в журнале для \Sys{HelloUniverse.spec}:
\begin{verbatim}
add_changelog -e "- Updated to new version v3.0." HelloUniverse.spec
\end{verbatim}

Добавим все изменения в историю \Sys{git} и сформируем комментарий к изменениям
одной командой:
\begin{verbatim}
gear-commit -a
\end{verbatim}

Соберем пакет и проверим результат:
\begin{verbatim}
gear-hsh
\end{verbatim}

После проверки и окончания работы нужно поставить метку на созданные изменения.
В случае, если конфигурация \Sys{git} настраивалась для работы с \Sys{gpg}-ключом,
выполните команду:
\begin{verbatim}
gear-create-tag
\end{verbatim}
Утилита \Sys{gear-create-tag} позволяет автоматически сформировать имя метки, комментарий
и подключить \Sys{gpg}-подпись для достоверности и дальнейшего использования в
сборочной инфраструктуре Альт.

Все изменения можно сохранить удаленно. Например, на личный репозиторий:
\begin{verbatim}
git remote add myrepo http://some.place/myrepo
git push v3.0
\end{verbatim}

\section{Сборка на различных процессорных архитектурах на примере Эльбрус \textbf{Пример №5}}

Исполнение примера предполагает наличие компьютера с процессором семейства
Эльбрус и настроенной операционной системой Альт.

Благодаря специализированно сформированным репозиториям пакетов, сборка
под различные архитектуры, не требует внесения изменений в инструкции.
Что позволяет проводить абсолютно идентичный процесс сборки пакета.
Но такое доступно не всегда и засвистит от собираемых в пакет исходных данных.
В таких случаях нужно вникать в тонкости компиляции под выбранную архитектуру
и проводить оптимизацию исходных данных перед сборкой в пакет.
Со спецификой сборки для процессов Эльбрус можно подробнее познакомиться
на ALT Linux Wiki%
\footnote{\href{https://www.altlinux.org/\%D0\%AD\%D0\%BB\%D1\%8C\%D0\%B1\%D1\%80\%D1\%83\%D1\%81/lcc}{https://www.altlinux.org/Эльбрус/lcc}}%
\footnote{\href{https://www.altlinux.org/\%D0\%AD\%D0\%BB\%D1\%8C\%D0\%B1\%D1\%80\%D1\%83\%D1\%81/\%D0\%BF\%D0\%BE\%D1\%80\%D1\%82\%D0\%B8\%D1\%80\%D0\%BE\%D0\%B2\%D0\%B0\%D0\%BD\%D0\%B8\%D0\%B5}{https://www.altlinux.org/Эльбрус/портирование}}

Возмем специально подготовленные исходные данные адаптированные для
формирования специфичного вывода доступного только процессорам Эльбрус.

\begin{verbatim}
$ git clone ----branch=4.0-alt1 \
https://gitlab.basealt.space/alt/ExampleFirstProject.git ~/ExampleFirstProject-v4/
$ cd ~/ExampleFirstProject-v4/
\end{verbatim}

Пример подготовлен для сборки средствами GEAR и не несет в себе существенных
изменений в сборочных инструкциях. Все изменения касаются версии и релиза.
Убедиться в этом можно запустив команду:
\begin{verbatim}
$ git show
\end{verbatim}

Запустим сборку пакета:
\begin{verbatim}
$ gear-hsh
\end{verbatim}

Установите пакет(Обратите внимание что версия архитектуры может отличаться и быть равной одному из значений e2kv6, e2kv5, e2kv4, e2k):
\begin{verbatim}
$ sudo apt-get install ~/hasher/repo/e2kv5/RPMS.hasher/HelloUniverse-4.0-alt1.e2kv5.rpm
\end{verbatim}

Если всё сделано верно то запустив исполняемый файл пакета будет получен вывод:
\begin{verbatim}
$ HelloUniverse
Hello Elbrus!

\end{verbatim}

Предлагается самостоятельно собрать предложенную в примере версию исходных
данных для других, отличных от архитектуры процессоров семейства Эльбрус
операционной системе. Это может продемонстрировать  идентичность сборочных
инструкций  для различных процессорных архитектур.


\section{Приложение}
В случае отсутствия доступа к внешним сетевым ресурсам, приведена структура
и содержимое файлов примеров. В некоторых примерах предполагается работа с
\Sys{git}-репозиторием. Формирование репозитория в таком случае остается
самостоятельной работой.

\textbf{Исходные данные:} Базовые инструменты \Sys{rpm} \textbf{Пример №1}

\noindent Структура:
\begin{verbatim}
.
├── HelloUniverse
│   ├── HelloUniverse.cpp
│   └── Makefile
└── HelloUniverse.spec

\end{verbatim}

\noindent\underline{\Sys{HelloUniverse/HelloUniverse.cpp}:}
\begin{verbatim}
#include <iostream>
int main()
{
    std::cout << "Hello Universe\n";
    return 0;
}

\end{verbatim}

\noindent\underline{\Sys{HelloUniverse/Makefile}:}
\begin{verbatim}
    g++ ./HelloUniverse.cpp -o HelloUniverse
clear:
    rm ./HelloUniverse

\end{verbatim}

\noindent\underline{\Sys{HelloUniverse.spec}:}
\begin{verbatim}
%define _unpackaged_files_terminate_build 1

Name:    HelloUniverse
Version: 1.0
Release: alt1
Summary: Most simple RPM package
License: no
Group:   Development/Other
Source:  %name-%version.tar
BuildRequires: gcc-c++

%description
This is my first RPM

%prep
%setup -n %name

%build
%make_build HelloUniverse

%install
mkdir -p %{buildroot}%_bindir
install -m 755 HelloUniverse %{buildroot}%_bindir

%files
%_bindir/%name

%changelog
* Mon Apr 01 2024 Some One <someone@altlinux.org> 1.0-alt1
- Init Build

\end{verbatim}

\textbf{Исходные данные:} Hasher \textbf{Пример №2}

\noindent Структура:
\begin{verbatim}
.
├── .gear
│   ├── HelloUniverse.spec
│   ├── rules
│   └── tags
│       └── list
└── HelloUniverse
    ├── HelloUniverse.cpp
    └── Makefile

\end{verbatim}

\noindent\underline{\Sys{.gear/HelloUniverse.spec}:}
\begin{verbatim}
%define _unpackaged_files_terminate_build 1

Name:    HelloUniverse
Version: 2.0
Release: alt2
Summary: Most simple RPM package
License: GPL-1.0
Group:   Development/Other
Source:  %name-%version.tar
Patch:   %name-%version-%release.patch
BuildRequires: gcc-c++

%description
This is my first RPM

%prep
%setup
%autopatch -p1

%build
%make_build HelloUniverse

%install
mkdir -p %{buildroot}%_bindir
install -m 755 HelloUniverse %{buildroot}%_bindir

%files
%_bindir/%name

%changelog
* Wed May 01 2024 Some One <someone@altlinux.org> 2.0-alt2
- Instructions for build by release tag added.

* Wed May 01 2024 Some One <someone@altlinux.org> 2.0-alt1
- New source fiture added.

* Mon Apr 01 2024 Some One <someone@altlinux.org> 1.0-alt1
- Init Build

\end{verbatim}

\noindent\underline{\Sys{.gear/rules}:}
\begin{verbatim}
tar: v@version@:HelloUniverse/
spec: .gear/HelloUniverse.spec
diff: v@version@:HelloUniverse/ .:HelloUniverse/ exclude=.gear/**
copy?: .gear/*.patch

\end{verbatim}

\noindent\underline{\Sys{.gear/tags/list}:}
\begin{verbatim}
397124774839a4aeede25b281238085716af1b16 v2.0

\end{verbatim}

\noindent\underline{\Sys{HelloUniverse/HelloUniverse.cpp}:}
\begin{verbatim}
#include <iostream>
int main()
{
    std::cout << "Hello New Universe!\n";
    return 0;
}

\end{verbatim}

\noindent\underline{\Sys{HelloUniverse/Makefile}:}
\begin{verbatim}
HelloUniverse: HelloUniverse.cpp
        g++ ./HelloUniverse.cpp -o HelloUniverse
clear:
        rm ./HelloUniverse

\end{verbatim}

\textbf{Исходные данные:} GEAR \textbf{Пример №3}

\noindent
Пример №3  использует исходные данные аналогичные примеру №2.\\
\par
\textbf{Исходные данные:} GEAR обновление пакета \textbf{Пример №4}

\noindent Структура:
\begin{verbatim}
.
├── .gear
│   ├── HelloUniverse.spec
│   ├── rules
│   ├── tags
│   │   └── list
│   └── upstream
│       └── remotes
└── HelloUniverse
    ├── HelloUniverse.cpp
    └── Makefile

\end{verbatim}

\noindent\underline{\Sys{.gear/HelloUniverse.spec}:}
\begin{verbatim}
%define _unpackaged_files_terminate_build 1

Name:    HelloUniverse
Version: 3.0
Release: alt1
Summary: Most simple RPM package
License: GPL-1.0
Group:   Development/Other
Source:  %name-%version.tar
Patch:   %name-%version-%release.patch
BuildRequires: gcc-c++

%description
This is my first RPM

%prep
%setup
%autopatch -p1

%build
%make_build HelloUniverse

%install
mkdir -p %{buildroot}%_bindir
install -m 755 HelloUniverse %{buildroot}%_bindir

%files
%_bindir/%name

%changelog
* Wed May 01 2024 Some One <someone@altlinux.org> 2.0-alt2
- Instructions for build by release tag added.

* Wed May 01 2024 Some One <someone@altlinux.org> 2.0-alt1
- New source fiture added.

* Mon Apr 01 2024 Some One <someone@altlinux.org> 1.0-alt1
- Init Build.

\end{verbatim}


\noindent\underline{\Sys{.gear/rules}:}
\begin{verbatim}
tar: v@version@:HelloUniverse/
spec: .gear/HelloUniverse.spec
diff: v@version@:HelloUniverse/ .:HelloUniverse/ exclude=.gear/**
copy?: .gear/*.patch

\end{verbatim}

\noindent\underline{\Sys{.gear/tags/list}:}
\begin{verbatim}
397124774839a4aeede25b281238085716af1b16 v2.0

\end{verbatim}

\noindent\underline{\Sys{.gear/upstream/remotes}:}
\begin{verbatim}
[remote "upstream"]
        url = https://gitlab.basealt.space/alt/ExampleFirstProject.git
        fetch = +refs/heads/*:refs/remotes/origin/*

\end{verbatim}

\noindent\underline{\Sys{HelloUniverse/HelloUniverse.cpp}:}
\begin{verbatim}
#include <iostream>
int main()
{
    std::cout << "Hello Universe!\n";
    std::cout << "Hello New Universe!\n";
    return 0;
}

\end{verbatim}

\noindent\underline{\Sys{HelloUniverse/Makefile}:}
\begin{verbatim}
HelloUniverse: HelloUniverse.cpp
        g++ ./HelloUniverse.cpp -o HelloUniverse
clear:
        rm ./HelloUniverse
\end{verbatim}

\textbf{Исходные данные:} Сборка на различных процессорных архитектурах на примере Эльбрус \textbf{Пример №5}

\noindent Структура:
\begin{verbatim}
.
├── .gear
│   ├── HelloUniverse.spec
│   ├── rules
│   ├── tags
│   │   └── list
│   └── upstream
│       └── remotes
└── HelloUniverse
    ├── HelloUniverse.cpp
    └── Makefile

\end{verbatim}

\noindent\underline{\Sys{.gear/HelloUniverse.spec}:}
\begin{verbatim}
%define _unpackaged_files_terminate_build 1

Name:    HelloUniverse
Version: 4.0
Release: alt1
Summary: Most simple RPM package
License: GPL-1.0
Group:   Development/Other
Source:  %name-%version.tar
Patch:   %name-%version-%release.patch
BuildRequires: gcc-c++

%description
This is my first RPM

%prep
%setup
%autopatch -p1

%build
%make_build HelloUniverse

%install
mkdir -p %{buildroot}%_bindir
install -m 755 HelloUniverse %{buildroot}%_bindir

%files
%_bindir/%name

%changelog
* Sat Jun 15 2024 Some One <someone@altlinux.org> 4.0-alt1
- Updated to new release.

* Wed May 01 2024 Some One <someone@altlinux.org> 2.0-alt2
- Instructions for build by release tag added.

* Wed May 01 2024 Some One <someone@altlinux.org> 2.0-alt1
- New source fiture added.

* Mon Apr 01 2024 Some One <someone@altlinux.org> 1.0-alt1
- Init Build

\end{verbatim}


\noindent\underline{\Sys{.gear/rules}:}
\begin{verbatim}
tar: v@version@:HelloUniverse/
spec: .gear/HelloUniverse.spec
diff: v@version@:HelloUniverse/ .:HelloUniverse/ exclude=.gear/**
copy?: .gear/*.patch

\end{verbatim}

\noindent\underline{\Sys{.gear/tags/list}:}
\begin{verbatim}
2ab75a02f89deedcf48bf86c2246ac31a88303e4 v4.0

\end{verbatim}

\noindent\underline{\Sys{.gear/upstream/remotes}:}
\begin{verbatim}
[remote "upstream"]
        url = https://gitlab.basealt.space/alt/ExampleFirstProject.git
        fetch = +refs/heads/*:refs/remotes/origin/*

\end{verbatim}

\noindent\underline{\Sys{HelloUniverse/HelloUniverse.cpp}:}
\begin{verbatim}
#include <iostream>
#if defined(__e2k__)
int main()
{
    std::cout << "Hello Elbrus!\n";
    return 0;
}
#else
int main()
{
    std::cout << "Hello Universe!\n";
    std::cout << "Hello New Universe!\n";
    return 0;
}
#endif

\end{verbatim}

\noindent\underline{\Sys{HelloUniverse/Makefile}:}
\begin{verbatim}
HelloUniverse: HelloUniverse.cpp
        g++ ./HelloUniverse.cpp -o HelloUniverse
clear:
        rm ./HelloUniverse
\end{verbatim}
