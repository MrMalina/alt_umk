\chapter{Инструмент Hasher}\label{chapter-hasher}
\Emph{Hasher} --- инструмент для сборки пакетов с использованием минимальной контролируемой среды в \Sys{chroot}. \Sys{Hasher} опирается на системный вызов \Sys{chroot} и создает изолированную среду для сборки отдельного пакета утилитой \Sys{rpm-build}.

\Sys{Hasher} облегчает поддержание сборочных зависимостей и позволяет собирать пакеты для разных дистрибутивов. \Sys{Hasher} проверяет пакеты с помощью утилиты \Sys{sisyphus\_check} и создает локальный \Sys{APT}-репозиторий с результатами сборки.

Для подготовки сборочного окружения \Sys{hasher} берет пакеты как из удалённых репозиториев, настроенных в основной системы, так и из локального репозитория (\Sys{$\sim$/hasher/repo/}), в который попадают ранее собранные пакеты.

Так же \Sys{hasher} удобен для отладки процесса сборки. Если сборка пакета прервалась, выполните команду \Sys{hsh-shell}, чтобы попасть в терминал \Sys{chroot}, исправьте ошибку и продолжайте сборку с прерванного этапа.


\section{Описание системы Hasher}
Опишем структуру каталогов в \Sys{hasher}.

\begin{itemize}
	\item \Sys{hasher}
	
	\begin{itemize}
		\item \Sys{chroot} --- само минимальное окружение. в этом каталоге находится корневое дерево содержащее минимальный набор пакетов, \Sys{rpmbuild} и сборочные зависимости.
		
		\item \Sys{aptbox} --- набор утилит для установки, обновления и удаления пакетов \Sys{chroot}. Например, тут лежит модифицированный \Sys{apt-get}, с помощью которого происходит установка пакетов в \Sys{chroot}.
		
		\item \Sys{cache} --- в этом каталоге кэшируются файлы, необходимые для создания \Sys{chroot}.
		\begin{itemize}
			\item каталог \Sys{$\sim$/hasher/chroot/usr/src/RPM}, который содержит подкаталоги для \Sys{rpmbuild}:
			\begin{itemize}
				\item \Sys{BUILD}
				\item \Sys{RPMS}
				\item \Sys{SOURCES}
				\item \Sys{SPECS}
				\item \Sys{SRPMS}
			\end{itemize} 
		\end{itemize}
		
		\item \Sys{repo}, который содержит подкаталоги:
		\begin{itemize}
			\item \Sys{SRPMS.hasher} --- исходники (sources) пакета.
			\item \Sys{<платформа>/RPMS.hasher/} --- каталог с пакетами, собранными под конкретную архитектуру.
			В этот каталог можно добавить необходимые сборочные зависимости, когда пакета ещё нет в репозитории и оценить сборку.
		\end{itemize}
	\end{itemize}
	
	\item \Sys{$\sim$/.hasher}
	\begin{itemize}
		\item \Sys{apt.config} --- конфигурация для \Sys{apt-get} из \Sys{$\sim$/hasher/aptbox/}
		
		\item \Sys{config} --- конфигурация самого \Sys{hasher}
	\end{itemize}
	
	\item \Sys{/etc/hasher-priv/} каталог с конфигурацией для вспомогательной утилиты \Sys{hasher-priv}
	\begin{itemize}
		\item \Sys{./user.d}
		\item \Sys{fstab} --- информация о точках монтирования вспомогательной программы \Sys{hasher-priv}
		\Sys{system} --- конфигурация вспомогательной программы \Sys{hasher-priv}
	\end{itemize} 
\end{itemize}


\section{Принцип действия}
В каталоге \Sys{$\sim$/hasher/chroot} создаётся сборочная среда. В неё устанавливается минимальный набор пакетов, \Sys{rpmbuild} и сборочные зависимости. Вся работа, связанная непосредственно со сборкой пакетов, происходит в этом \Sys{chroot}. \Sys{Hasher} не использует одну и ту же созданную среду. У каждого пакета свои зависимости и поэтому для каждого пакета его изолированное сборочное окружение будет уникальным. Для этого в \Sys{Hasher} заложено создание новой сборочной среды под каждый собираемый пакет. При успешном завершении сборки каталог \Sys{chroot} очищается\footnote{\href{https://alt-packaging-guide.github.io/\#_hasher_start}{https://alt-packaging-guide.github.io/\#\_hasher\_start}}.

Сборочное окружение можно сохранить. Для этого используют флаг \Sys{--lazy-cleanup}.

Сборочное окружение каждый раз удаляется при инициализации нового \Sys{chroot}.

Из принципа действия \Sys{Hasher} следует:
\begin{itemize}
	\item cборка не пройдёт корректно, если в пакете не указаны все зависимости;
	\item сборку в \Sys{hasher} можно воспроизвести на другом компьютере, поскольку \Sys{hasher} не использует индивидуальные пользовательские настройки;
	\item в настройках можно определить, для какого репозитория будет проходить сборка. Поэтому не меняя пользовательское окружение, можно собрать пакет под разные дистрибутивы.
\end{itemize}


\section{Настройка Hasher}
В разделе \hyperlink{1.3}{1.3 <<Установка необходимых пакетов для процесса сборки>>}  упоминалась установка пакета \Sys{hasher}. Для отдельной установки используйте команду: 
\begin{verbatim}
	# apt-get install hasher
\end{verbatim}

Добавьте пользователя в группу \Sys{rpm}:
\begin{itemize}
	\item для локального пользователя:
	\begin{verbatim}
		# usermod -aG rpm <user>
	\end{verbatim}
	
	\item для доменного пользователя: 
	\begin{verbatim}
		# gpasswd -a <username> rpm
	\end{verbatim}
\end{itemize} 

Запустите демон \Sys{hasher-priv}:
\begin{verbatim}
	# systemctl enable --now hasher-privd.service
\end{verbatim} 

Укажите учетную запись для работы с \Sys{hasher}. Добавьте учетные записи для работы с \Sys{hasher} в группу \Sys{hashman}:
\begin{verbatim}
	# hasher-useradd <user>
\end{verbatim} 

Заново авторизуйтесь, чтобы изменения вступили в силу. 

Укажите расположение сборочной среды (каталоги настроек и их инициализации):
\begin{verbatim}
	$ mkdir ~/hasher
	$ mkdir ~/.hasher
\end{verbatim}}

При создании каталога \Sys{.hasher} следует учитывать два правила: 
\begin{enumerate}
	\item права доступа соответствуют \Sys{drwxr-xr-x}, то есть каталог доступен на запись;
	
	\item на файловой системе, смонтированной с \Sys{noexec} или \Sys{nodev}, каталог располагать нельзя\footnote{\href{
			https://serverfault.com/questions/547237/explanation-of-nodev-and-nosuid-in-fstab}{
			https://serverfault.com/questions/547237/explanation-of-nodev-and-nosuid-in-fstab}};
	\begin{itemize}
		\item \Sys{noexec} устанавливается, если в системе есть файлы с правами \Sys{rwsr-xr-x} (запустить исполняемые файлы с правами владельца или группы) и владельцем \Sys{root}. Запустить файл \Sys{rwxr-xr-x} на такой файловой системы невозможно. а следовательно, и создать корректное сборочное окружение. \Sys{Hasher} не зависит от пользовательского окружения.
		
		\item \Sys{nodev} говорит о том, что на файловой системе не будут созданы файлы устройств. Это не соответствует функциональности \Sys{hasher} (см.~\hyperlink{5.7}{раздел 6.7 <<Монтирование файловых систем внутри \Sys{Hasher}>>}). 
	\end{itemize} 
\end{enumerate}

Создайте конфигурационный файл: 
\begin{verbatim}
	packager="`rpm --eval %packager`"
	no_sisyphus_check="packager,buildhost,gpg"
	allowed_mountpoints=/dev/pts,/proc,/dev/shm
	lazy_cleanup=1
	# share_ipc=1
	# Необходимо в случае, когда требуется разрешить доступ в сеть из Hasher chroot.
	# Данная опция не должна применяться без особой необходимости.
	# share_network=1
	# нужен для сборки с репозиториями отличающимися от системного:
	# apt_config="$HOME/.hasher/apt.conf"
\end{verbatim}}

Создадим сборочное окружение: 
\begin{verbatim}
	$ hsh --initroot-only 
	#путь ~/hasher является параметром по умолчанию. Полностью команда выглядит так:
	$ hsh --initroot-only ~/hasher
	$ ls -al ~/hasher
\end{verbatim}

Если сборочное окружение не создано, оно построится при первой сборке пакета. 

Чем больше пакет, тем больше времени понадобится на его сборку. В конфигурационном файле \Sys{hasher-priv} прописываются ограничения на ресурсы, выделяемые на сборку (CPU, память, общее время исполнения и другие). Чтобы увеличить время бездействия утилиты \Sys{hasher-priv}, требуется отредактировать файл \Sys{/etc/hasher-priv/system/}:
\begin{verbatim}
	wlimit_time_elapsed=360000
	wlimit_time_idle=360000
\end{verbatim} 

Сборочная среда строится по умолчанию на источниках для основной системы (пользователя/хост-системы).


\section{Справочная страница Hasher}
Возможности \Sys{Hasher} задокументированы в инструкциях\footnote{\href{http://uneex.ru/static/AltlinuxOrg_Hasher/}{http://uneex.ru/static/AltlinuxOrg\_Hasher/}} к пакетам \Sys{hasher} и \Sys{hasher-priv}. Для вызова справочной информации по \Sys{hasher} наберите в консоли \Sys{man package}:
\begin{verbatim}
	$ man hsh
	
	HASHER(7)                          ALT Linux                         HASHER(7)
	
	NAME
	hasher - modern safe package building technology
	
	SYNOPSIS
	hsh [options] <path-to-workdir> <package>...
	...
\end{verbatim} 

\begin{verbatim}
	$ man hasher-priv
	
	HASHER-PRIV(8)          System Administration Utilities         HASHER-PRIV(8)
	
	NAME
	hasher-priv - privileged helper for the hasher project
	
	SYNOPSIS
	hasher-priv [options] <args>
	...
\end{verbatim}

Это достаточно подробная инструкция по использованию утилиты \Sys{hasher}. Здесь вы найдёте полное описание, опции и флаги, содержимое рабочего каталога. 


\hypertarget{1.3}{\section{Монтирование файловых систем внутри Hasher}}
\Sys{Hasher} умеет монтировать внутрь \Sys{hasher}-контейнера точки монтирования и виртуальные файловые системы из основной машины\footnote{\href{https://www.altlinux.org/Hasher/\%D0\%A0\%D1\%83\%D0\%BA\%D0\%BE\%D0\%B2\%D0\%BE\%D0\%B4\%D1\%81\%D1\%82\%D0\%B2\%D0\%BE\#\%D0\%9C\%D0\%BE\%D0\%BD\%D1\%82\%D0\%B8\%D1\%80\%D0\%BE\%D0\%B2\%D0\%B0\%D0\%BD\%D0\%B8\%D0\%B5_\%D1\%84\%D0\%B0\%D0\%B9\%D0\%BB\%D0\%BE\%D0\%B2\%D1\%8B\%D1\%85_\%D1\%81\%D0\%B8\%D1\%81\%D1\%82\%D0\%B5\%D0\%BC_\%D0\%B2\%D0\%BD\%D1\%83\%D1\%82\%D1\%80\%D0\%B8_hasher}{https://www.altlinux.org/Hasher/Руководство}}. Этот механизм применяется в тех случаях, когда собираемому приложению для сборки требуется доступ к ресурсам основной машины, которые \Sys{Hasher} не предоставляет по умолчанию. Например, виртуальная файловая система \Sys{/proc} или \Sys{/dev/pts}, которых по умолчанию нет в \Sys{hasher}-контейнере. Файловая система \Sys{/proc} получает информацию о состоянии и конфигурации ядра и системы.

Для монтирования файловой системы следует:
\begin{enumerate}
	\item В файле \Sys{/etc/hasher-priv/fstab} описать файловую систему.
	\item В файле \Sys{/etc/hasher-priv/system} указать файловую систему с помощью опции \Sys{allowed\_mointpoints}.
	\item Указать файловую систему либо при запуске \Sys{Hasher} в опции \Sys{--mountpoints}, либо в конфигурационном файле \Sys{$\sim$/.hasher/config} в ключе \Sys{known\_mountpoints}.
	\item Прописать необходимую файловую систему в \Sys{SPEC}-файле в теге \Sys{BuildReq}, либо в списке зависимостей.
\end{enumerate} 