\chapter{Заключение}%\addcontentsline{toc}{chapter}{Заключение}

Учебно-методическое пособие «Инфраструктура разработки программных пакетов и 
сборки программного обеспечения» содержит систематизированные сведения 
научно-практического и прикладного характера, изложенные в удобной с методической 
точки зрения форме для самостоятельного изучения и усвоения соответствующей рабочей 
программы магистратуры и дополнительного профессионального образования «Информационная 
культура цифровой трансформации». 

Пособие поможет студентам овладеть основами и логикой современного подхода к разработке 
свободного программного обеспечения на основе методологии сборки программных пакетов 
по оригинальным технологиям ALT Linux Team, команды разработчиков российского 
свободного программного обеспечения, которая поддерживает и развивает проект 
репозитория Сизиф (Sisyphus). 

Для дальнейшего профессионального развития, рекомендуем изучить курсы Георгия Владимировича Курячего, 
преподавателя ВМК МГУ, давнего члена ALT Linux Team, разработчика ООО <<Базальт СПО>>:
\begin{itemize}
	\item \href{http://uneex.ru/LecturesCMC/LinuxSoftware2017}{«Программное обеспечение GNU/Linux»}
	\item \href{http://uneex.ru/LecturesCMC/Distro2016}{Архитектура дистрибутивов Linux}
	\item \href{http://uneex.ru/LecturesCMC/LinuxApplicationDevelopment2022}{«Разработка программного обеспечения для GNU/Linux»}
	\item \href{http://uneex.ru/LecturesCMC/PackageMaintaining2013}{«Сопровождение пакетов GNU/Linux»}
\end{itemize}

Каждый пакет в \Sys{Sisyphus} имеет как минимум одного сопровождающего --- мантейнера. 
Мантейнер --- это специалист, который собирает пакет для установки программы из централизованного 
репозитория исходного кода, исправляет обнаруженные в программе ошибки, общается 
с разработчиками программы (upstream-ом), старается реагировать на нужды пользователей --- 
изучает багрепорты, отвечает на вопросы. 
Мантейнер должен быть членом ALT Linux Team (команды ALT), потому, что пакет в Сизиф может 
попасть только от члена команды. 

\href{https://packages.altlinux.org/ru/sisyphus/maintainers/}{Состав} ALT Linux Team, 
а также пакеты, которые кажды из них собирает публично доступны на сайте \url{https://packages.altlinux.org}.
ALT Linux Team организационно не зависит от компании «Базальт СПО», 
хотя многие её сотрудники являются членами Team. 

Если вы освоите технологию сборки пакетов на достаточном уровне, например, собрали несколько нужных вам, 
но отсутствующих в репозитории Сизиф свободных программ, и почувствуете потребность внести свою лепту 
в общий труд команды ALT Linux Team, --- смело вступайте в команду. Процедура вступления описана на 
wiki-страничке \href{https://altlinux.org/Join}{Join}.

Результатом процедуры Join является возможность непосредственно участвовать в разработке репозитория Сизиф. 
После прохождения Join кандидат (человек, вступающий в ALT Linux Team) становится мантейнером.
