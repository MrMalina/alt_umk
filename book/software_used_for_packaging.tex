\chapter{Основные команды пакетного менеджера}\label{software-used-for-packaging}

В разделе \hyperlink{1.3}{1.3 <<Установка необходимых пакетов для процесса сборки>>} упоминался перечень пакетов для работы со сборкой:
\begin{verbatim}
	gcc, rpm-build, rpmlint, make, python, git, gear, hasher, patch, rpmdevtools.
\end{verbatim}

Опишем основные инструменты для управления пакетами и их сборки. Технологическую базу репозитория Sisyphus составляют адаптированные к нуждам команды разработчиков программы и специально разработанные решения\footnote{\href{https://wiki.altlinux.ru/QuickStart/PkgManagment\#\%D0\%9E\%D1\%81\%D0\%BD\%D0\%BE\%D0\%B2\%D0\%BD\%D1\%8B\%D0\%B5_\%D0\%B8\%D0\%BD\%D1\%81\%D1\%82\%D1\%80\%D1\%83\%D0\%BC\%D0\%B5\%D0\%BD\%D1\%82\%D1\%8B_\%D0\%B4\%D0\%BB\%D1\%8F_\%D1\%83\%D0\%BF\%D1\%80\%D0\%B0\%D0\%B2\%D0\%BB\%D0\%B5\%D0\%BD\%D0\%B8\%D1\%8F_\%D0\%BF\%D0\%B0\%D0\%BA\%D0\%B5\%D1\%82\%D0\%B0\%D0\%BC\%D0\%B8}{https://wiki.altlinux.ru/QuickStart/PkgManagment}}. К ним можно отнести\footnote{\href{https://www.altlinux.org/\%D0\%A0\%D0\%B5\%D0\%BF\%D0\%BE\%D0\%B7\%D0\%B8\%D1\%82\%D0\%BE\%D1\%80\%D0\%B8\%D0\%B9_\%D0\%A1\%D0\%9F\%D0\%9E\#APT_\%D0\%B8_\%D1\%80\%D0\%B5\%D0\%BF\%D0\%BE\%D0\%B7\%D0\%B8\%D1\%82\%D0\%BE\%D1\%80\%D0\%B8\%D0\%B9_\%D0\%BF\%D0\%B0\%D0\%BA\%D0\%B5\%D1\%82\%D0\%BE\%D0\%B2}{https://www.altlinux.org/Репозиторий\_СПО\#APT\_и\_репозиторий\_пакетов.}}:
\begin{itemize}
	\item \Emph{RPM (менеджер пакетов).} Используется для просмотра, сборки, установки, инспекции, проверки, обновления и удаления отдельных программных пакетов. Каждый такой пакет состоит из набора файлов и информации о пакете, включающей название, версию, описание пакета и т.д. Отличительными особенностями \Sys{RPM} в Sisyphus являются: удобное поведение <<по умолчанию>> для уменьшения количества шаблонного кода в \Sys{.spec}-файлах, обширный модульный набор макросов для упаковки различных типов пакетов, развитые механизмы автоматического вычисления межпакетных зависимостей при сборке пакетов, поддержка set-версий в зависимостях на разделяемые библиотеки, автоматическое создание пакетов с отладочной информацией с поддержкой зависимостей между такими пакетами. В контексте данной темы \Sys{RPM} рассматривается не только как менеджер пакетов, но и как набор инструментов для их сборки (\Sys{rpmbuild}). 
	
	\item \Emph{APT} (усовершенствованная система управления программными пакетами). Умеет автоматически разрешать зависимости при установке, обеспечивает установку из нескольких источников и целый ряд других уникальных возможностей, включая получение последней версии списка пакетов из репозитория и обновление системы.
	
	\item \textbf{Hasher} --- инструмент для безопасной, воспроизводимой и высокопроизводительной сборки \Sys{RPM}-пакетов в контролируемой среде.
	
	\item \textbf{Gear} --- набор инструментов для поддержки совместной разработки \Sys{RPM}-пакетов в системе контроля версий \Sys{git}. \Sys{Gear} интегрирован с \Sys{hasher} и \Sys{rpmbuild}. \Sys{Gear} подготавливает \Sys{SRPM} из \Sys{git} репозитория, распаковывает его и запускает \Sys{hasher} или \Sys{rpmbuild}.
\end{itemize}


\section{Описание RPM-пакета}
\Emph{RPM-пакет} --- это специальный архив с файлами. Сам файл пакета состоит из четырех секций --- начального идентификатора, сигнатуры, бинарного заголовка и \Sys{cpio}-архива с файлами проекта и деревом каталога\footnote{\href{https://www.opennet.ru/docs/RUS/rpm_guide/13.html}{https://www.opennet.ru/docs/RUS/rpm\_guide/13.html}}.

\Sys{RPM}-пакеты делятся на несколько категорий --- пакеты с исходным кодом, бинарные пакеты и платформо-независимые бинарные пакеты.
\begin{itemize}
	\item \textbf{RPM-пакет} (бинарные) --- это архив с расширением \Sys{.rpm}. Такой пакет содержит скомпилированные под определённую процессорную архитектуру исполняемые файлы и библиотеки. На системах с разной процессорной архитектурой не получится использовать один и тот же скомпилированный бинарный \Sys{rpm}-пакет.
	
	\item \textbf{\Sys{noarch}-пакет} --- платформо-независимый бинарный пакет.
	
	\item \textbf{SRPM-пакет} (source RPM, пакет с исходным кодом) --- это архив с расширением \Sys{.src.rpm}. \Sys{SRPM} содержит исходный код, патчи, если необходимо, и \Sys{SPEC}-файл. Эти пакеты содержат всю информацию для сборки пакета.
\end{itemize}

По инструкциям из \Sys{SPEC} файла собирается бинарный \Sys{RPM}-пакет\footnote{\href{https://uneex.ru/static/RedHatRPMGuideBook/rpm_guide-linux.html\#16_html}{https://uneex.ru/static/RedHatRPMGuideBook/rpm\_guide-linux.html\#16\_html}}. На основе бинарных пакетов строится база данных в \Sys{/var/lib/rpm}. Вся информация о пакетах хранится в базе данных \Sys{Packages}. Инструкции содержат также информацию о правах доступа и их применении в процессе установки, скрипты, запускаемые при установке или удалении пакета.

Принято называть пакеты \Sys{RPM} по типу:

\begin{verbatim}
	имя-версия-релиз.процессорная_архитектура.rpm
\end{verbatim}

При этом в системе имя установленного пакета будет отличаться: в командной строке можно обратиться за информацией по пакету, указывая только его имя, если установлена одна версия пакета, и указывая \Sys{имя-версию-релиз}, если установлено больше версий\footnote{\href{https://www.opennet.ru/docs/RUS/rpm_guide/13.html}{https://www.opennet.ru/docs/RUS/rpm\_guide/13.html}}.


\section{Описание SPEC-файла}
\Emph{SPEC-файл} --- RPM Specification File --- это текстовый файл, который описывает процесс сборки и конфигурацию пакета, служит инструкцией для утилиты \Sys{rpmbuild}. Он содержит метаданные, такие как имя пакета, версию, лицензию, а также разделы с инструкциями для сборки, установки и упаковки программного обеспечения, журнал изменений пакета \footnote{\href{https://wiki.mageia.org.ru/index.php?title=\%D0\%A1\%D1\%82\%D1\%80\%D1\%83\%D0\%BA\%D1\%82\%D1\%83\%D1\%80\%D0\%B0_RPM_SPEC-\%D1\%84\%D0\%B0\%D0\%B9\%D0\%BB\%D0\%B0}{https://wiki.mageia.org.ru/index.php?title=Структура\_RPM\_SPEC-файла}}. \Sys{SPEC}-файл можно рассматривать как <<инструкцию>>, которую утилита \Sys{rpmbuild} использует для сборки \Sys{RPM}-пакета.

\Sys{Spec}-файл состоит из трех разделов: Header (Заголовок/Преамбула), Body (Тело) и Сhangelog(Журнал изменений).
\begin{enumerate}
	\item \textbf{Header} (Заголовок) --- этот раздел содержит метаданные о пакете, такие как его имя (Name), версия (Version), релиз (Release), краткое описание (Summary), лицензия (License) и другие параметры, которые идентифицируют и характеризуют пакет.
	
	\item \textbf{Body} (Тело) --- этот раздел содержит инструкции для процесса сборки пакета. В нем определяются различные секции, такие как \Sys{BuildRequires} (зависимости для сборки), \Sys{\%build} (инструкции для сборки), \Sys{\%install} (инструкции для установки), \Sys{\%files} (список файлов, включенных в пакет), и другие.
	
	\item \textbf{Changelog} (Журнал изменений) --- этот раздел содержит историю изменений пакета. Он включает записи о внесённых изменениях, включая дату изменения, автора и краткое описание того, что было изменено.
\end{enumerate}

\paragraph{Преамбула (Заголовок)}
Заголовок \Sys{SPEC}-файла содержит информацию о пакете: версию, исходный код, патчи, зависимости.

Рекомендуемый порядок заголовочных тэгов:
\begin{itemize}
	\item Name, Version, Release, Serial
	\item далее Summary, License, Group, Url, Packager, BuildArch
	\item потом Source*, Patch*
	\item далее PreReqs, Requires, Provides, Conflicts
	\item и, наконец, Prefix, BuildPreReqs, BuildRequires.
\end{itemize}

Ниже приведен пример части \Sys{SPEC}-файла \Sys{notepadqq}:
\begin{verbatim}
	Summary:	A Linux clone of Notepad++
	Name:		notepadqq
	Version:	1.4.8
	Release:	alt2
	License:	GPLv3
	Group:		Editors
	URL:		http://notepadqq.altervista.org/wp/
	Source0:	%name-%version.tar
	Source1:	codemirror.tar
\end{verbatim}