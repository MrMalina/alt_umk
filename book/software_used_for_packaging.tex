\chapter{Основные команды пакетного менеджера}\label{software-used-for-packaging}

В разделе \hyperlink{1.3}{1.3 <<Установка необходимых пакетов для процесса сборки>>} упоминался перечень пакетов для работы со сборкой:
\begin{verbatim}
	gcc, rpm-build, rpmlint, make, python, git, gear, hasher, patch, rpmdevtools.
\end{verbatim}

Опишем основные инструменты для управления пакетами и их сборки. Технологическую базу репозитория Sisyphus составляют адаптированные к нуждам команды разработчиков программы и специально разработанные решения\footnote{\href{https://wiki.altlinux.ru/QuickStart/PkgManagment\#\%D0\%9E\%D1\%81\%D0\%BD\%D0\%BE\%D0\%B2\%D0\%BD\%D1\%8B\%D0\%B5_\%D0\%B8\%D0\%BD\%D1\%81\%D1\%82\%D1\%80\%D1\%83\%D0\%BC\%D0\%B5\%D0\%BD\%D1\%82\%D1\%8B_\%D0\%B4\%D0\%BB\%D1\%8F_\%D1\%83\%D0\%BF\%D1\%80\%D0\%B0\%D0\%B2\%D0\%BB\%D0\%B5\%D0\%BD\%D0\%B8\%D1\%8F_\%D0\%BF\%D0\%B0\%D0\%BA\%D0\%B5\%D1\%82\%D0\%B0\%D0\%BC\%D0\%B8}{https://wiki.altlinux.ru/QuickStart/PkgManagment}}. К ним можно отнести\footnote{\href{https://www.altlinux.org/\%D0\%A0\%D0\%B5\%D0\%BF\%D0\%BE\%D0\%B7\%D0\%B8\%D1\%82\%D0\%BE\%D1\%80\%D0\%B8\%D0\%B9_\%D0\%A1\%D0\%9F\%D0\%9E\#APT_\%D0\%B8_\%D1\%80\%D0\%B5\%D0\%BF\%D0\%BE\%D0\%B7\%D0\%B8\%D1\%82\%D0\%BE\%D1\%80\%D0\%B8\%D0\%B9_\%D0\%BF\%D0\%B0\%D0\%BA\%D0\%B5\%D1\%82\%D0\%BE\%D0\%B2}{https://www.altlinux.org/Репозиторий\_СПО\#APT\_и\_репозиторий\_пакетов.}}:
\begin{itemize}
	\item \Emph{RPM (менеджер пакетов).} Используется для просмотра, сборки, установки, инспекции, проверки, обновления и удаления отдельных программных пакетов. Каждый такой пакет состоит из набора файлов и информации о пакете, включающей название, версию, описание пакета и т.д. Отличительными особенностями \Sys{RPM} в Sisyphus являются: удобное поведение <<по умолчанию>> для уменьшения количества шаблонного кода в \Sys{.spec}-файлах, обширный модульный набор макросов для упаковки различных типов пакетов, развитые механизмы автоматического вычисления межпакетных зависимостей при сборке пакетов, поддержка set-версий в зависимостях на разделяемые библиотеки, автоматическое создание пакетов с отладочной информацией с поддержкой зависимостей между такими пакетами. В контексте данной темы \Sys{RPM} рассматривается не только как менеджер пакетов, но и как набор инструментов для их сборки (\Sys{rpmbuild}). 
	
	\item \Emph{APT} (усовершенствованная система управления программными пакетами). Умеет автоматически разрешать зависимости при установке, обеспечивает установку из нескольких источников и целый ряд других уникальных возможностей, включая получение последней версии списка пакетов из репозитория и обновление системы.
	
	\item \textbf{Hasher} --- инструмент для безопасной, воспроизводимой и высокопроизводительной сборки \Sys{RPM}-пакетов в контролируемой среде.
	
	\item \textbf{Gear} --- набор инструментов для поддержки совместной разработки \Sys{RPM}-пакетов в системе контроля версий \Sys{git}. \Sys{Gear} интегрирован с \Sys{hasher} и \Sys{rpmbuild}. \Sys{Gear} подготавливает \Sys{SRPM} из \Sys{git} репозитория, распаковывает его и запускает \Sys{hasher} или \Sys{rpmbuild}.
\end{itemize}